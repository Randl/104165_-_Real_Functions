\begin{definition}[Almost uniform convergence]
Let $\left\{  f_n\right\}$, $f$ be real functions measurable on $A$. We say that $f_n\to f$ almost uniformly, if for all $\epsilon>0$ exists measurable $E$ such that $\lambda(A\setminus E) \leq \epsilon$ and on $E$ $f_n \to f$ uniformly.
\end{definition}
\begin{theorem}[Egorov's theorem]
	Let $A\subseteq \mathbb{R}^n$ be a measurable set with finite measure. Let $\left\{  f_n\right\}$, $f$ real functions measurable on $A$ and $f_n\to f$ a.e. on $A$. Then for all $\epsilon>0$ exists $E$ such that $\lambda(A\setminus E) \leq \epsilon$ and on $E$ $f_n \to f$ uniformly a.e.
	\begin{proof}
		$f_n\to f$, then exists $n$ such that for $j\geq n$ $\abs{f_j-f} < \frac{1}{k}$.
	
	Denote 
	$$E_n^k = \left\{ c: \abs{f_j(x) -f(x)} < \frac{1}{k} \quad \forall k\geq n \right\}  $$
	Then for all $k$ $x\in \bigcup_{n=1}^\infty E_n^k$.
	
	From the assumption we get $\lambda\qty(A\setminus \bigcup_{n=1}^\infty E_n^k)=0$. Note that $E_{n}^k \subseteq E_{n+1}^k$ thus
	$$\forall k \: A\setminus E_n^k \supseteq A\setminus E_{n+1}^k$$
	$$ A\setminus \bigcup_{n=1}^\infty E_n^k = \bigcap_{n=1}^\infty A\setminus E_{n}^k$$
	$$0=\lambda\qty(A\setminus \bigcup_{n=1}^\infty E_n^k) = \lambda\qty(\bigcap_{n=1}^\infty A\setminus E_{n}^k) = \lim_{n\to\infty} \lambda(A\setminus A_n^k)$$
	
	Thus for all $k$ $\lambda(A\setminus A_n^k) \to 0$. For all $\epsilon>0$ exists $n_k$ such that 
	$$n\geq n_k \Rightarrow \lambda\qty(A\setminus E_n^k) < \frac{\epsilon}{2^k}$$
	Denote $E = \bigcup_{k=1}^\infty$ (it is measurable). 
	$$\lambda(A\setminus E) = \lambda\qty(\bigcup_{k} A\setminus E_{n_k}^k) \leq \sum \lambda(A\setminus E_{n_k}^k) < \sum \frac{\epsilon}{2^k} = \epsilon$$
	
	For all $x\in E\subseteq E_{n_k}^k \subseteq E_n^k$, thus for all $k$ exists $n_k$ such that
	$$\abs{f_k(x) - f(x) } \leq \frac{1}{k}$$
	Thus $f_n \to f$ uniformly on $E$.
	\end{proof}
\end{theorem}

\paragraph{Example}
The condition $\lambda(A)<\infty$ is necessary. Take $A=\mathbb{R}$ and 
$$f_n = \chi_{[n,\infty]}$$
$f_n\to 0$ for all $x$.


\section{Lebesgue integral}
\subsection{Stage 1}
Let $s$ simple measurable function we can write
$$s(x) = \sum_{i=1}^m a_i \chi_{A_i} (x)$$
Such that $\left\{  A_i\right\}$ are measurable and disjoint and $\mathbb{R}^n = \bigcup_{i=1}^n A_i$ and $0\leq a_i \in \mathbb{R}$. We define
$$\int s\dd{\lambda} = \sum_{i=1}^m a_i \lambda(A_i)$$
where $0\cdot \infty =0$
\begin{prop}
	$\int s\dd{\lambda}$ is well-defined
	\begin{proof}
	 Directly from \ref{simpints}
	\end{proof}
\end{prop} 
\begin{prop}
$$0\leq \int s\dd{\lambda} \leq \infty$$
\end{prop} 
\begin{prop}
For all $0\leq c \in \mathbb{R}$, $\int cs\dd{\lambda} = c\int s\dd{\lambda}$
\end{prop} 
\begin{prop}
$$\int (s+t) \dd{\lambda} = \int s \dd{\lambda} + \int t \dd{\lambda}$$
\end{prop} 
\begin{prop} \label{simpints}
$s\leq t$ a.e. $\Rightarrow$ $\int s\dd{\lambda} \leq \int t \dd{\lambda}$
\begin{proof}
	
	Denote $N = \left\{ x: s(x)> t(t) \right\}$
	$$s(x) = \sum_{i=1}^{m} a_i \chi_{A_i}(x)$$
	$$t(x) = \sum_{i=1}^{k} b_i \chi_{B_i}(x)$$
	For all $i$
	$$A_i = \bigcup_{j=1}^k (A_i \cap B_j) $$
	Then
	$$\int s \dd{\lambda} = \sum a_i \lambda(A_i) $$
	$$\lambda(A_i) = \lambda\qty(\bigcup_{j=1}^k A_i \cap B_j) = \sum_{j=1}^k \lambda(A_i \cap B_j)$$
	$$\int s\dd{\lambda} = \sum_{i,j} a_i \lambda(A_i \cap B_j)$$
	$$\int t\dd{\lambda} = \sum_{i,j} b_j \lambda(A_i \cap B_j)$$
	
	For all $i,j$, $$a_i\lambda(A_i\cap B_j) \leq b_j \lambda(A_i\cap B_j)$$
	If $\lambda(A_i\cap B_j)=0$ it's obvious. Else $\exists x \in A_i\cap B_j \setminus N$, and for that $x$ $s(x) \leq t(x)$ and thus $a_i\leq b_j$
	
	and thus $\int s\dd{\lambda} \leq \int t \dd{\lambda}$.
\end{proof}
\end{prop} 
\begin{prop}
	$s= t$ a.e. $\Rightarrow$ $\int s\dd{\lambda} = \int t \dd{\lambda}$
	
	\begin{proof}
		Directly from \ref{simpints}
	\end{proof}
\end{prop} 
\begin{prop}
If $\alpha \in \mathbb{R}$ and $s'(x) = s(x+\alpha)$, then $s'$ is simple and 
$$\int s' \dd{\lambda} = \int s \dd{\lambda}$$
\end{prop} 


\subsection{Stage 2}

Let $f: \mathbb{R}^n \to [0,\infty]$ be a measurable function. We define
$$\int f \dd{\lambda } = \sup \left\{ \int s\dd{\lambda} : s<\infty, s\leq f, \text{ simple masurable} \right\}$$
\begin{prop}
	$\int f\dd{\lambda}$ is well-defined
\end{prop} 
\begin{prop}
	$$0\leq \int f\dd{\lambda} \leq \infty$$
\end{prop} 
\begin{prop}
	For all $0\leq c \in \mathbb{R}$, $\int cf\dd{\lambda} = c\int f\dd{\lambda}$
\end{prop} 
\begin{prop}
	If $f\leq g$ a.e.
	$$\int f \dd{\lambda} \leq \int g \dd{\lambda}$$
	\begin{proof}
		Denote $N = \left\{ x ; f(x) > g(x) \right\}$ and $I = \int f\dd{\lambda}$.
		
		For all $\epsilon>0$  exists $s$ such that
		$$\int s \dd{\lambda} \geq I-\epsilon$$
		$$\tilde{s} = \begin{cases}
		s(x) & x\notin N\\
		0 &  x\in N
		\end{cases}$$
		Then $\tilde{s} \leq g$.
		$$I-\epsilon = \int s \dd{\lambda}  = \int \tilde{s} \dd{\lambda} \leq \int g \dd{\lambda}$$
		Thus
		$$\int \tilde{s} \dd{\lambda} \geq I$$
	\end{proof}
\end{prop} 
\begin{prop}
If $f= g$ a.e.
$$\int f \dd{\lambda} = \int g \dd{\lambda}$$
\end{prop} 
\begin{prop}
If $f'(x)= f(x+\alpha)$ a.e.
$$\int f \dd{\lambda} = \int f' \dd{\lambda}$$
\end{prop} 


\begin{theorem}[Monotone convergence theorem]
Let $\left\{ f_n\right\}$ sequence of measurable functions such that $0\leq f_1\leq \dots $
Let $f(x) = \lim_{n\to\infty} f_n(x)$ then $\int f\dd{\lambda} = \lim_{n\to\infty} \int f_n \dd{\lambda}$.
\begin{proof}
	For all $n$
	$$\int f_n \dd{\lambda} \leq \int f\dd{\lambda}$$
	Denote $I = \lim_{n\to\infty} \infty f_n$.
	
	Assume $I<\int f\dd{\lambda}$. Then $\exists c\in \mathbb{R}$ such that
	$$I<c<\int f\dd{\lambda}$$
	From definition of $\int f\dd{\lambda}$ exists simple $t$ such that $0\leq t\leq f$ such that $\int t \dd{\lambda} > c$. Exists $0<q<1$ such that $\int t \dd{\lambda} > \frac{c}{q}$
	
	Define $s=qt$, then $c<\int s \dd{\lambda}$ and $f(x).s(x)$.
	
	Define $$E_k = \left\{ x: f_k(x) \geq s(x) \right\}$$, then
	$$E_1 \subseteq E_2 \subseteq \dots$$
	$$\mathbb{R}^n = \bigcup_{k=1}^\infty E_k$$
	
	For all $k$
	$$f_k \geq f_k \chi_{E_k} \geq s(x) \chi_{E_k}$$
	By writing $s(x) = \sum_{i=1}^m a_i \chi_{A_i}$ we get
	$$ s(x) \chi_{E_k} = \sum a_i \chi_{A_i \cap E_k}$$
	
	Then
	$$\int f_k \dd{\lambda} \geq \int s(x) \chi_{E_k} \dd{\lambda} = \sum a_i \lambda(A_i \cap E_k)$$
	
	But
	$$\lambda(A_i)  = \lim_{k\to \infty} \lambda(A_i\cap E_k)$$
	and
	$$I = \lim_{k\to \infty} \int f_k \dd{x} = \lim_{k\to \infty} \sum_{i=1}^m a_i \lambda(A_i \cap E_k) = \sum_{i=1}^m a_i \lambda(A_i) =\int s \dd{x} > c$$
	
\end{proof}
\begin{coll}
	$$\int (f+g) \dd{\lambda} = \int f\dd{\lambda} + \int g \dd{\lambda}$$
	\begin{proof}
		Find $\left\{s_k\right\} \to f$, $\left\{t_k\right\} \to g$, then
		$$\int f+g \dd{\lambda} = \lim_{k\to \infty} \int s_k+t_k \dd{\lambda}= \lim_{k\to \infty} \int s_k\dd{\lambda}+\lim_{k\to \infty} \int  t_k \dd{\lambda} = \int s\dd{\lambda} + \int t \dd{\lambda}$$
	\end{proof}
\end{coll}

\begin{coll}
	Let $\left\{ f_k \right\}$ sequence of measurable functions, $f_k\geq 0$ on $\mathbb{R}^n$ then
	$$\int \qty[\sum_{k=1}^\infty f_k ] \dd{\lambda} = \sum_{k=1}^\infty \int f_k \dd{\lambda}$$
\end{coll}
\end{theorem}