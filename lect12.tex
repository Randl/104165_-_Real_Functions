\begin{theorem}[Fatou's lemma]
	Let $\left\{ f_k \right\}_{k=1}^\infty$ a sequence of nonegative measurable functions.
	
	$$\int \liminf f_k \dd{\lambda} \leq \liminf \int f_k \dd{\lambda}$$
	
	\begin{proof}
		$$\liminf f_k(x)  =\sup\limits_m \inf\limits_{j\geq m} f_j(x)$$
		$$g_m(x) = \inf\limits_{j\geq m} f_j(x) $$
		Since $0\leq g_m \leq g_{m+!}$ and $g_m \leq f_m$,
		$$\liminf f_k(x) = \sup\limits_m g_m(x) = \lim g_m(x)$$
		$$\int \liminf f_k \dd{\lambda} =  \int \lim_{m\to \infty} g_m \dd{\lambda} = \int \lim_{m\to \infty}  g_m \dd{\lambda} \leq \liminf \int f_m \dd{\lambda}$$
	\end{proof}
\end{theorem}

\subsection{Stage 3}
\begin{definition}
	Let $f : \mathbb{R}^n \to [-\infty, \infty]$ measurable. $f$ is called integrable if $\int f^+(x) \dd{\lambda} <\infty$ and $\int f^-(x) \dd{\lambda} <\infty$ and in this case
	$$\int f\dd{\lambda} = \int f^+(x) \dd{\lambda}  - \int f^-(x) \dd{\lambda}$$
	
	Set of all integrable functions is denoted $\mathcal{L}^!(\mathbb{R}^n)$.
\end{definition}

\begin{prop}
	$$\int \abs{f} \dd{\lambda} \leq \infty \iff f\in \mathcal{L}^1$$
	\begin{proof}
		$f\in \mathcal{L}^1$, then
		$$\int \abs{f}  \dd{\lambda}  = \int \qty(f_++f_-) \dd{\lambda} = \int f_+ \dd{\lambda} + \int f_- \dd{\lambda} < \infty$$
		
		$f_+, f_- \leq \abs{f}$, then $\int f_+ \dd{\lambda} ,\int f_- \dd{\lambda} \leq  \int \abs{f} \dd{\lambda} < \infty$.
	\end{proof}
\end{prop}
\begin{prop}
$$\int \abs{f} \dd{\lambda} \geq \abs{\int f \dd{\lambda}}$$
\begin{proof}
	$$\abs{\int f \dd{\lambda} } = \abs{\int f_+ \dd{\lambda} +\int f_- \dd{\lambda}  } \leq \int f_+ \dd{\lambda} + \int f_- \dd{\lambda}  = \int \abs{f}$$
\end{proof}
\end{prop}

\begin{prop}
	If $f\in \mathcal{L}^1$ then
	$$\lambda(f^{-1}(-\infty)) = \lambda(f^{-1}(\infty)) = 0$$
	\begin{proof}
		Let $A = \lambda(f^{-1}(\infty))$ and $\lambda(A) > 0$. 
		$f_+ \geq \infty \cdot \chi_A$.
		
		Then $\int f_+ \dd{\lambda} \geq \int \infty \cdot \chi_A \dd{\lambda} = \infty$.
	\end{proof}
\end{prop}


\begin{prop}
	If $f=g$ a.e. and $f\in \mathcal{L}_1$, then $g\in \mathcal{L}_1$ and $\int f\dd{\lambda}  =\int g\dd{\lambda}$.
\end{prop}


\begin{prop}
	If $f\in \mathcal{L}_1$ and $\int \abs{f} \dd{\lambda} = 0$, then $f=0$ a.e.
	\begin{proof}
		Define 
		$$A_k = \left\{ x: \abs{f(x) } > \frac{1}{k}  \right\}$$
		$$\abs{f} \geq \abs{f} \chi_{A_k} \geq \frac{1}{k} \chi_{A_k}$$
		$$0 =\int \abs{f} \dd{\lambda} \geq \frac{1}{k} \int \chi_{A_k} $$
		Thus $\lambda(A_k) = 0$ and thus
		$$\lambda(\bigcup A_k) = 0$$
	\end{proof}
\end{prop}

\begin{prop}
	$\mathcal{L}^1$ is vector space.
	\begin{proof}
		For $a\geq 0$,
		$$af = af_+ + af_-$$
		$$\int af \dd{\lambda} = \int af_+ - \int af_- = a\int f_+ - a\int f_- = a\int f$$
		For $a<0$, we note that $(af)_\pm = -af_\mp$.
		
		Let $h=f+g$
		$$h_+-h_- = h = (f_+-f_-) + (g_+ - g_-)$$
		$$h_+ + f_- + g_- = h_- + f_+ + g_+$$
		$$\int h_+ +\int f_- + \int g_-=  \int h_- + \int f_+ + \int g_+$$
		Then $h_+ \leq f_++ g_+$ and $\int h_+ < \infty$, i.e. $h\in \mathcal{L}^1$.
	\end{proof}
\end{prop}

\begin{theorem}[Monotonic convergence theorem]
	Let $\left\{ f_n\right\}$ sequence of integrable functions such that $ f_1\leq f_2 \leq  \dots $ and $\sup\limits_n \int f_n \dd{\lambda} <\infty$
	Let $f(x) = \lim_{n\to\infty} f_n(x)$ then $\int f\dd{\lambda} = \lim_{n\to\infty} \int f_n \dd{\lambda}$.
	
	\begin{proof}
		Define $g_n = f_n-f_1$ then $g_n$ are non-negative and thus converge as we've shown.
	\end{proof}
	
\end{theorem}

\begin{theorem}[Dominated convergence theorem]
	Let $\left\{  f_n \right\}$ be a sequence of measurable functions on $\mathbb{R}^n$ such that exists $0\leq g \in \mathcal{L}^1$ such that
	$\abs{f_k(x)} \leq g_x$ a.e.
	
	Let
	$$f(x) = \lim_{k\to \infty} f_k(x)$$
	then
	$$\int f \dd{\lambda} = \lim_{k\to \infty} \int f_k \dd{\lambda}$$
	
	\begin{proof}
		
	\end{proof}
\end{theorem}