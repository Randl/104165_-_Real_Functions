\section{Introduction}
If $\forall x \quad f_n(x) \to f(x)$ (pointwise) does $\int_0^1 f_n(x) \dd{x} \to \int_0^1 f(x) \dd{x}$?

Define $f_n(x) = \chi_{r_1, r_2, \dots r_n}$, where $\left\{ r_i \right\} = \mathbb{Q} \cap [0,1]$, i.e., first $n$ rational numbers. Those functions are integrable since they are non-zero in finite number of points. However, $f(x) = \chi_{\mathbb{Q} \cap [0,1]}$ is not integrable.
\paragraph{Riemann integral: limit}
We defined Riemann integral as limit of Riemann sum:
$$\int_a^b f(x) \dd{x} = \lim \sum f(x'_i)(x_{i+1}-x_i)$$

\begin{center}
	\includesvg{lect1/l1pic1.svg}
\end{center}

By dividing on $y$, we bound the error by the size of each interval, $\epsilon$:
$$g(x) = s\chi_{A_1} + \qty(s+\epsilon)\chi_{A_2} + \dots$$
$$\forall x \quad \abs{g(x)-f(x)} \leq \epsilon
$$
\section{Measure}
For $A\subseteq \mathbb{R}$ we want to define size of $A$ which we will denote $\lambda(A)$. What do we require from $\lambda$?
\begin{enumerate}
	\item $\lambda(\qty[a,b]) = b- a$
	\item $0\leq \lambda(A) \leq \infty$
	\item $\lambda(\emptyset) = 0$
	\item If $A = \bigcup_{k=1}^\infty A_k$ and $\forall i,j \quad A_i \cap A_j = \emptyset$, then $\lambda(A) = \sum_{i=1}^\infty \lambda(A_k)$.
	\item $\lambda(A+x) = \lambda(A)$, where $A+x = \left\{s+x: a\in A \right\}$.
\end{enumerate}

From those properties we get additional properties:
\begin{itemize}
	\item Additivity:
	$$A = \bigcupdot_{i=1}^n A_i  \Rightarrow \lambda(A) = \sum_{i=1}^n \lambda(A_i)$$
	\item  If $A \subseteq B$, then $\lambda(A) \leq \lambda(B)$.
\end{itemize}
\begin{theorem}Function $\lambda$ fulfilling 1-5 and defined on every subset of $\mathbb{R}$ doesn't exist.
	\begin{proof}
		
		Suppose there exists such $\lambda$.
		
		Define equivalence relation $x\sim y$ iff $x-y \in \mathbb{Q}$. Define $E$ choose from each equivalence class one representative from $\qty[0, \frac{1}{2}]$. Note that if $q_1\neq q_2$, then $q_1+E \cap q_2 + E = \emptyset$, since else $e_1-e_2=q_1-q_2$ and $e_1\sim e_2$, in contradiction.
		
		From definition $E \subset \qty[0, \frac{1}{2}]$. Take a look at
		$$\bigcup_{k=2}^\infty \qty(\frac{1}{k}+E) \subseteq \qty[0,1]$$
		Thus
		$$\lambda \qty(\bigcup_{k=2}^\infty \qty(\frac{1}{k}+E) ) \leq \lambda\qty([0,1])=1$$
		On the other hand
		$$\lambda \qty(\bigcup_{k=2}^\infty \qty(\frac{1}{k}+E) ) = \sum_{k=2}^\infty \lambda\qty(\frac{1}{k}+E) ) = \lambda\qty(E) )$$
		Thus $\lambda(E)=0$. However, 
		$$\mathbb{R} = \bigcupdot_{r\in \mathbb{Q}} r+E$$
		From sigma-additivity
		$$\lambda(\mathbb{E}) = \sum_{r\in \mathbb{Q}} \lambda(r+E) = 0$$
		But $\lambda(\mathbb{R}) \geq \lambda([0,1])$, in contradiction.
	\end{proof}
\end{theorem}