\documentclass[]{article}
\usepackage{amsmath}
\usepackage{amsfonts}
\usepackage{amssymb}
\usepackage{hyperref}
\usepackage{gensymb}
\usepackage{graphicx}
\usepackage{svg}
\usepackage{bbding}
\usepackage{mathtools}
\usepackage{centernot} % not parallel, etc.
\usepackage{lmodern}
\usepackage{morewrites}
\usepackage{xcolor,sectsty} % colorful sections
\usepackage[left=10mm, top=10mm, right=10mm, bottom=20mm, nohead]{geometry}
%\usepackage{bigints}
\usepackage{dsfont} %mathbb 1
\usepackage{esint} % beatiful integrals
\usepackage[arrowdel]{physics}
\usepackage{amsthm}

\usepackage[T1]{fontenc}
% Nicer default font (+ math font) than Computer Modern for most use cases
% \usepackage{mathpazo} % problems with greek vectors
\usepackage[utf8x]{inputenc} % Allow utf-8 characters in the tex document
% Prevent overflowing lines due to hard-to-break entities
\sloppy 
% Colors for the hyperref package
\definecolor{urlcolor}{rgb}{0,.145,.698}
\definecolor{linkcolor}{rgb}{.71,0.21,0.01}
\definecolor{citecolor}{rgb}{.12,.54,.11}
% Setup hyperref package
\hypersetup{
	breaklinks=true,  % so long urls are correctly broken across lines
	colorlinks=true,
	urlcolor=urlcolor,
	linkcolor=linkcolor,
	citecolor=citecolor,
}


\DeclareFontFamily{OMX}{lmex}{}
\DeclareFontShape{OMX}{lmex}{m}{n}{<-> lmex10}{}


%colors of sections
\definecolor{secfont}{RGB}{46,116,181}
\definecolor{subfont}{RGB}{146,23,57}
\definecolor{parfont}{RGB}{19,127,43}
\definecolor{subparfont}{RGB}{7,11,100}

\subsectionfont{\color{subfont}}
\sectionfont{\color{secfont}}
\paragraphfont{\color{parfont}}
\subparagraphfont{\color{subparfont}}



% declare a new theorem style
\newtheoremstyle{bluestyle}%
{3pt}% Space above
{3pt}% Space below 
{}% Body font
{}% Indent amount
{\bfseries\color{blue}}% Theorem head font
{.}% Punctuation after theorem head
{.5em}% Space after theorem head
{}% Theorem head spec (can be left empty, meaning ‘normal’)
% declare a new theorem style
\newtheoremstyle{redstyle}{3pt}{3pt}{}{}{\bfseries\color{red}}{.}{.5em}{}
\newtheoremstyle{olivestyle}{3pt}{3pt}{}{}{\bfseries\color{olive}}{.}{.5em}{}
\newtheoremstyle{orangestyle}{3pt}{3pt}{}{}{\bfseries\color{orange}}{.}{.5em}{}
\newtheoremstyle{magentastyle}{3pt}{3pt}{}{}{\bfseries\color{magenta}}{.}{.5em}{}

\theoremstyle{bluestyle}
\newtheorem{theorem}{Theorem}[section]
\theoremstyle{redstyle}
\newtheorem{definition}{Definition}[section]
\theoremstyle{magentastyle}
\newtheorem{coll}{Collary}[section]
\theoremstyle{olivestyle}
\newtheorem{lemma}{Lemma}[section]
\theoremstyle{olivestyle}
\newtheorem{prop}[theorem]{Proposition}
%\usepackage{babel}[english]
%opening
\title{104165 - Real functions}
\author{Baruch Solel}
% Johns Lebugue Integration
% Roiden Rudin


% disjoint union
\makeatletter
\def\moverlay{\mathpalette\mov@rlay}
\def\mov@rlay#1#2{\leavevmode\vtop{%
		\baselineskip\z@skip \lineskiplimit-\maxdimen
		\ialign{\hfil$\m@th#1##$\hfil\cr#2\crcr}}}
\newcommand{\charfusion}[3][\mathord]{
	#1{\ifx#1\mathop\vphantom{#2}\fi
		\mathpalette\mov@rlay{#2\cr#3}
	}
	\ifx#1\mathop\expandafter\displaylimits\fi}
\makeatother

\newcommand{\cupdot}{\charfusion[\mathbin]{\cup}{\cdot}}
\newcommand{\bigcupdot}{\charfusion[\mathop]{\bigcup}{\cdot}}
\DeclareMathOperator{\inter}{int}



%\newtheorem{proof}{Proof}[theorem]


\parindent=0em
\begin{document}


\maketitle

\begin{abstract}

\end{abstract}

%\tableofcontents
\section{Introduction}
If $\forall x \quad f_n(x) \to f(x)$ (pointwise) does $\int_0^1 f_n(x) \dd{x} \to \int_0^1 f(x) \dd{x}$?

Define $f_n(x) = \chi_{r_1, r_2, \dots r_n}$, where $\left\{ r_i \right\} = \mathbb{Q} \cap [0,1]$, i.e., first $n$ rational numbers. Those functions are integrable since they are non-zero in finite number of points. However, $f(x) = \chi_{\mathbb{Q} \cap [0,1]}$ is not integrable.
\paragraph{Riemann integral: limit}
We defined Riemann integral as limit of Riemann sum:
$$\int_a^b f(x) \dd{x} = \lim \sum f(x'_i)(x_{i+1}-x_i)$$

\begin{center}
	\includesvg{lect1/l1pic1.svg}
\end{center}

By dividing on $y$, we bound the error by the size of each interval, $\epsilon$:
$$g(x) = s\chi_{A_1} + \qty(s+\epsilon)\chi_{A_2} + \dots$$
$$\forall x \quad \abs{g(x)-f(x)} \leq \epsilon
$$
\section{Measure}
For $A\subseteq \mathbb{R}$ we want to define size of $A$ which we will denote $\lambda(A)$. What do we require from $\lambda$?
\begin{enumerate}
	\item $\lambda(\qty[a,b]) = b- a$
	\item $0\leq \lambda(A) \leq \infty$
	\item $\lambda(\emptyset) = 0$
	\item If $A = \bigcup_{k=1}^\infty A_k$ and $\forall i,j \quad A_i \cap A_j = \emptyset$, then $\lambda(A) = \sum_{i=1}^\infty \lambda(A_k)$.
	\item $\lambda(A+x) = \lambda(A)$, where $A+x = \left\{s+x: a\in A \right\}$.
\end{enumerate}

From those properties we get additional properties:
\begin{itemize}
	\item Additivity:
	$$A = \bigcupdot_{i=1}^n A_i  \Rightarrow \lambda(A) = \sum_{i=1}^n \lambda(A_i)$$
	\item  If $A \subseteq B$, then $\lambda(A) \leq \lambda(B)$.
\end{itemize}
\paragraph{Theorem} Function $\lambda$ fulfilling 1-5 and defined on every subset of $\mathbb{R}$ doesn't exist.
\subparagraph{Proof}
Suppose there exists such $\lambda$.

Define equivalence relation $x\sim y$ iff $x-y \in \mathbb{Q}$. Define $E$ choose from each equivalence class one representative from $\qty[0, \frac{1}{2}]$. Note that if $q_1\neq q_2$, then $q_1+E \cap q_2 + E = \emptyset$, since else $e_1-e_2=q_1-q_2$ and $e_1\sim e_2$, in contradiction.

From definition $E \subset \qty[0, \frac{1}{2}]$. Take a look at
$$\bigcup_{k=2}^\infty \qty(\frac{1}{k}+E) \subseteq \qty[0,1]$$
Thus
$$\lambda \qty(\bigcup_{k=2}^\infty \qty(\frac{1}{k}+E) ) \leq \lambda\qty([0,1])=1$$
On the other hand
$$\lambda \qty(\bigcup_{k=2}^\infty \qty(\frac{1}{k}+E) ) = \sum_{k=2}^\infty \lambda\qty(\frac{1}{k}+E) ) = \lambda\qty(E) )$$
Thus $\lambda(E)=0$. However, 
$$\mathbb{R} = \bigcupdot_{r\in \mathbb{Q}} r+E$$
From sigma-additivity
$$\lambda(\mathbb{E}) = \sum_{r\in \mathbb{Q}} \lambda(r+E) = 0$$
But $\lambda(\mathbb{R}) \geq \lambda([0,1])$, in contradiction.
\paragraph{Reqirements for measure in $\mathbb{R}$}

\begin{enumerate}
	\item $0\leq \lambda(E) \leq \infty$
	\item $\lambda(\emptyset) = 0$
	\item $\lambda(\qty[a_1,b_1]\cross\qty[a_2,b_2]\cross\dots\cross \qty[a_n,b_n]) = \prod_{i=1}^n (b_i-a_i)$
	\item If $A = \bigcupdot_{k=1}^\infty A_k$, then $\lambda(A) = \sum_{i=1}^\infty \lambda(A_k)$.
	\item If $C$ is acquired from $A$ by rotation or translation $\lambda(C) = \lambda(A)$.
\end{enumerate}
\paragraph{Note} In $\mathbb{R}^3$ it is impossible to define measure that fulfills those requirements eve if we replace sigma-additivity with additivity.
\paragraph{Banach–Tarski paradox} Denote $B$ -- unit ball in $\mathbb{R}^3$. We can write 
$$B = \bigcupdot_{i=1}^5 A_i$$
and find $C_i$ by rotation or translation of $A_i$ such that $\bigcupdot_{i=1}^5 C_i$ is two unit balls.

\begin{center}
	\includesvg{lect2/l2pic1.svg}
\end{center}
\subsection{Construction of $\lambda$}
\paragraph{Special boxes} 
Let $E$ box with edges parallel to axes:
$$E = \qty[a_1,b_1]\cross\qty[a_2,b_2]\cross\dots\cross \qty[a_n,b_n]$$

For $E$ we define 
 $$\lambda(E) = \prod_{i=1}^n (b_i-a_i)$$
 
 \paragraph{Special polygons} is a finite union of special boxes. 
 \subparagraph{Note} Each special polygon is a finite union of special boxes with disjoint interior.
 
 Let $P$ is special polygon written as $P = \bigcap_{i=1}^k A_i$ where $A_i$ is special box and their interior is disjoint.
 $$\lambda(P) = \sum_{i=1}^k \lambda(A_i)$$
 \paragraph{Claim}
 \begin{enumerate}
 	\item The definition is independent on choice of $A_i$.
 	\item If $P_1$, $P_2$ are special polygons and $P_1\subseteq P_2$ then $\lambda(P_1)\leq \lambda(P_2)$.
 	\item If $P_1$, $P_2$ are  special polygons with disjoint interior then
 	$$\lambda(P_1\cup P_2) =  \lambda(P_1)+\lambda(P_2)$$
 	\item For all $x\in \mathbb{R}^n$ 
 	$$\lambda(x+P) = \lambda(P)$$
 \end{enumerate}
 \subparagraph{Proof}
  \begin{enumerate}
 	\item Let $P = \bigcap A_i = \bigcap B_i$.
 	
 	If we continue edges of both $A_i$ and $B_i$ we'll get net which divides $P$ into $C_i$ which refines both $A_i$ and $B_i$ and thus
 	$$\lambda(P) = \sum_i \lambda(A_i) = \sum_i \lambda(B_i) = \sum_i \lambda(C_i)$$ 
 	\item Let $P_2= \bigcap A_i$ and choose the refinement which divides $P_1$. 
 	\item Find $A_i$ which divides both $P_1$ and $P_2$.
 	\item $\dots$
 \end{enumerate}
\subparagraph{Alternative proof}
For special boxes
$$\lambda(E) = \lim_{N\to \infty} \frac{1}{N^n} \abs{E \cap \frac{1}{N}\mathbb{Z}^n}$$

For $n=1$, $I=[a,b] \subseteq \mathbb{R}$. We claim
$$b-a = \lim_{N\to \infty} \frac{1}{N} \abs{E \cap \frac{1}{N}\mathbb{Z}}$$
First of all
$$b-a-1 \leq \abs{[a,b] \cap \mathbb{Z}} \leq b-a+1$$
To find $\abs{[a,b] \cap \frac{1}{2}\mathbb{Z}}$, we can use $\abs{[2a,2b] \cap \mathbb{Z}}$, which means

$$2b-2a-1 \leq \abs{E \cap \frac{1}{2}\mathbb{Z}} \leq 2b-2a+1$$

And for any $N$:
$$Nb-Na-1 \leq \abs{[a,b] \cap \frac{1}{N}\mathbb{Z}} \leq Nb-Na+1$$
$$b-a-\frac{1}{N} \leq \frac{1}{N}\abs{[a,b] \cap \frac{1}{N}\mathbb{Z}} \leq b-a+\frac{1}{N}$$
By sandwich rule, we get the equality.

We can do the same for higher dimension and for open sets, and then we can easily proof the claim.

If $P$ is special polygon and we take $\lim_{N\to \infty} \frac{1}{N^n} \abs{P \cap \frac{1}{N}\mathbb{Z}^n} = \sum \lambda(A_i) $
when $P=\bigcap A_i$

\paragraph{Open sets}
\subparagraph{Definition} $G$ is open if $\forall x\in G$ exists ball $B(x,r)$ such that $B\subset G$. Alternatively we can replace ball with special box.

Thus for any open $G\neq \emptyset$ 
$$G = \bigcup \left\{ P \text{ special polygon}  \right\}$$

And we can define
$$\lambda(G) = \sup \left\{ \lambda(P) | P\subseteq G \right\}$$

\paragraph{Claim}
\begin{enumerate}
	\item $$0 \leq \lambda(G) \leq \infty$$
	\item $$\lambda(G) = 0 \iff G =\emptyset$$
	\item $$\lambda(\mathbb{R}^n) = \infty$$
	\item $$G_1 \subseteq G_2 \Rightarrow \lambda(G_1) \leq \lambda(G_2)$$
	\item $$\lambda\qty(\bigcup_{k=1}^\infty G_k) \leq \sum \lambda(G_k)$$
	\item $$\lambda\qty(\bigcupdot_{k=1}^\infty G_k) = \sum \lambda(G_k)$$
	\item $$\lambda(P) = \lambda(\inter P ) =\inf \left\{ \lambda(G) : P\subseteq G\right\}$$
	\item $$\lambda(x+G) = \lambda(G)$$
\end{enumerate}

\begin{lemma}
Let $K\subseteq \mathbb{R}^n$ compact set and $\left\{ G_i \right\}_{i\in I}$ open cover ($K\subseteq \bigcup G_i$). Then exists $\epsilon>0$ such that $\forall x\in K$ exists $i\in I$ such that $B(x,\epsilon) \subseteq G_i$.
\end{lemma}
\begin{lemma} \label{lemma_poly_inf}
	For all polygon of dimension $P$ 
	$$\lambda(P) = \inf \left\{ \lambda(G) : P\subset G  \right\}$$

\end{lemma}
	\begin{proof}
	$$P \subseteq G \Rightarrow \lambda(P) \leq \lambda(G)$$
	Infimum would give
	$$\lambda(P) \leq \inf \left\{ \lambda(G) : P\subset G  \right\}$$
	
	
	Write $P = \bigcup_{k=1}^N I_k$. Then
	$$\lambda(P)  =\sum_{k=1}^N \lambda(I_k)$$
	For $\epsilon$ find $I_k^\epsilon$ such that
	$$\begin{cases}
	\inter I_k^\epsilon \supseteq I_k\\
	\lambda(I_k^\epsilon) \leq \lambda(I_k) + \frac{\epsilon}{N}
	\end{cases}$$
	Denote $G = \bigcup_{k=1}^N \inter (I_k^\epsilon)$, then, from subadditivity
	$$\lambda(G) \leq \sum_{k=1}^N \lambda (\inter I_k^\epsilon) = \sum_{k=1}^N \lambda (I_k^\epsilon) \leq \epsilon + \sum_{k=1}^N \lambda(I_k) $$
	
	In addition,
	$$\inf \lambda(G) \leq \lambda(P) $$
\end{proof}
\subparagraph{Proof} %TODO
\begin{enumerate}
	\item Obvious
	\item If $G$ is not empty, exists $x\in G$ and special box around $x$ such that $P\subseteq G$and thus $\lambda(G) \leq \lambda(P) > 0$
	\item Any box is subset of $\mathbb{R}^n$ thus $\lambda(\mathbb{R}^n) = \infty$
	\item Obvious
	\item Let $P$ special polygon, $P\subseteq \bigcup_{k=1}^\infty G_k$. We'll show that it's possible to write 
	$$P = \bigcup_{j=1}^N I_j$$
	finite union of special boxes with disjoint interior and for each $j$ exists $k$ such that $I_j \subset G_k$. Let $\epsilon$ from lemma for $K=P$. Write $P=\bigcup_{j=1}^N = I_j$ such that diameter of each $I_j<\epsilon$. If $x_j$ is center of $I_j$, then $I_j \subseteq B(x_j,\epsilon) \subseteq G_k$.
	
	If this is possible, for such $P$ denote $$P_k =  \bigcup_{j=1}^\infty I_j | I_j \subset G_k, \forall i<k \quad I_j \not\subset G_i$$
	
	Obviously $\bigcup P_k = P$ and union is finite since for some $m$, for every $k>m$ $P_m=\emptyset$, because there is finite number of $I_j$, and also internals of $P_k$ are disjoint.
	
	Thus $\lambda(P)=\sum \lambda(P_k)\leq\sum \lambda(G_k)$.
	This is right for any $P$, thus
	$$ \lambda\qty(\bigcup (G_k)) = \sup \left\{ \lambda(P) | P\subseteq \bigcup (G_k) \right\} \leq \sum_{k=1}^\infty \lambda(G_k)$$
	\item Since we have sigma-subadditivity, we need only one direction of inequality:
	$$\lambda(G_k) = \sup\left\{ \lambda(P) : P\subseteq G_k \right\}$$
	For any $N$
	$$\sum_{k=1}^N \lambda(G_k) = \sup\left\{ \sum_{k=1}^N \lambda(P_k) : P_k\subseteq G_k \right\} = \sup\left\{ \lambda\qty(\bigcupdot_{k=1}^N P_k) : P_k\subseteq G_k \right\} \leq \lambda\qty(\bigcupdot_{k=1}^N G_k) \leq \lambda\qty(\bigcupdot_{k=1}^\infty G_k) $$
	i.e.,
	$$\sum_{k=1}^\infty \lambda(G_k)  \leq \lambda\qty(\bigcupdot_{k=1}^\infty G_k) $$
	\item First, proof that $\lambda(P) = \lambda(\inter P)$. If $I=P$ is non-empty special box $I = [a_1,b_1]\times [a_2,b_2]\times \dots \times [a_n,b_n]$. For any $\epsilon>0$, $I_\epsilon = [a_1+\epsilon,b_1-\epsilon]\times [a_2+\epsilon,b_2-\epsilon]\times \dots \times [a_n+\epsilon,b_n-\epsilon]$. $I_\epsilon \subseteq \inter I$.
	
	That means that $\lambda(I_\epsilon) \leq \lambda(\inter I)$. Obviously, $\lambda(I_\epsilon)\to \lambda(I)$, i.e. $\lambda(I) \leq \lambda(\inter I)$.
	
	Generally, for $P = \bigcup_{k=1}^N I_k$, 
	$$\inter P \geq \bigcup_{k=1}^N \inter I_k$$
	thus
	$$\lambda(\inter P) \geq \lambda \qty(\bigcup_{k=1}^N \inter I_k) = \sum_{k=1}^N \lambda(\inter I_k) \geq \sum_{k=1}^N \lambda(I_k) = \lambda(P)$$
	For any $P$ 
	$$\lambda(\inter P) \geq \lambda P$$
	
	However
	$$\lambda(\inter P) = \sum \left\{ \lambda(Q) : Q \subseteq \inter P  \right\}$$
	$$Q \subseteq P \Rightarrow \lambda(Q) \leq \lambda(P) \Rightarrow \lambda(\inter P) \leq \lambda(P)$$
	
	Second part is obvious from Lemma \ref{lemma_poly_inf}.
	\item Obvious since it's right for polygons
\end{enumerate}
\subsection{Compact sets}
\begin{definition}
	For compact $K \subseteq \mathbb{R}^n$
	$$\lambda(K) = \inf \left\{ \lambda(G) : K\subseteq G \quad G \text{ is open} \right\}$$
\end{definition}
\begin{prop}
	$$0 \leq \lambda(K) < \infty$$
\end{prop}
\begin{proof}
	Each $K$ is subset of open box $A$ and $\lambda(A) < \infty$
\end{proof}
\begin{prop}
	$$K_1 \subseteq K_2 \Rightarrow \lambda(K_1) \leq \lambda(K_2)$$
\end{prop}
\begin{proof}
	Obvious
\end{proof}
\begin{prop}
	Subadditivity
	$$\lambda(K_1 \cup K_2) \leq \lambda(K_1) + \lambda(K_2)$$
\end{prop}
\begin{proof}
$$K_i \subseteq G_i$$
$$K_1 \cup K_2 \subseteq G_1 \cup G_2$$
$$\lambda(K_1 \cup K_2) \leq \lambda(G_1 \cup G_2) \leq \lambda(G_1) + \lambda( G_2)$$
Thus
$$\lambda(K_1 \cup K_2) \leq \lambda(K_1) + \lambda(K_2)$$
\end{proof}
\begin{prop}
	$$K_1 \cup K_2 = \emptyset \Rightarrow \lambda(K_1 \cup K_2) = \lambda(K_1) + \lambda(K_2)$$
\end{prop}
\begin{proof}
For $K_1$, $K_2$ exists $\epsilon>0$ such that $\forall \: x \in K_1 \: y \in K_2$, $d(x,y)\geq\epsilon$. Denote 
$$U_i = \bigcup_{x\in K_i} B\qty(x, \frac{\epsilon}{2}) \supset K_i$$
Let $K_1 \cup K_2 \subset G_i$, since  $K_i \subset U_i$, 
$$K_i \subset G \cap U_i$$
i.e.,
$$\forall i \quad \lambda(K_i) \leq \lambda(G\cap U_i)$$

Since $U_1 \cap U_2 = \emptyset$ (from construction)
$$(G\cap U_1) \cap (G \cap U_2) = \emptyset$$
$$\lambda(G\cap U_1) + \lambda (G \cap U_2) = \lambda \qty\big((G\cap U_1) \cap (G \cap U_2)) \leq \lambda(G)$$
Thus
$$\lambda(G) \geq \lambda(G\cap U_1) + \lambda (G \cap U_2) \geq \lambda(K_1)+\lambda(K_2)$$
i.e.,
$$\lambda(K_1\cup K_2) \geq \lambda(K_1)+\lambda(K_2)$$
\end{proof}

\subsection{General sets}
Define outer and inner measure similar to Darboux sums:
$$\lambda^*(A) = \inf \left\{ \lambda(G) : A\subset G \text{, open} \right\}$$
$$\lambda_*(A) = \sup \left\{ \lambda(K) : A\supset G \text{, compact} \right\}$$

\begin{prop}
	$$\lambda_*(A) \leq \lambda^*(A)$$
\end{prop}
\begin{proof}
	If $G$ is open and $K$ compact and $K\subset A\subset G$ then $K\subset G$, i.e. $\lambda(K) \leq \lambda(G)$. From that, taking supremum on $K$ and infimum on $G$, we get the required result.
\end{proof}
\begin{prop}
$$A\subset B \Rightarrow \lambda^*(A) \leq \lambda^*(B) \quad \lambda_*(A) \leq \lambda_*(B)$$
\end{prop}
\begin{proof}
Obvious.
\end{proof}
\begin{prop}
$$\lambda^*\qty(\bigcup_{k=1}^\infty A_k) \leq \sum_{k=1}^\infty \lambda^*(A_k)$$
\end{prop}

\begin{proof}
	
	$$\lambda^*(A) = \inf \left\{ \lambda(G) : A\subset G \text{, open} \right\}$$
	Thus exists $G_k$ such that
	$$\lambda(G_k) < \lambda^*(A_k) + \frac{\epsilon}{2^k}$$
	$$\lambda^*\qty(\bigcup_{k=1}^\infty A_k) \leq \lambda\qty(\bigcup_{k=1}^\infty G_k) \leq \sum_{k=1}^\infty \lambda\qty(G_k) < \sum_{k=1}^\infty \qty(\lambda^*\qty(A_k) + \frac{\epsilon}{2^k}) =\lambda^*\qty(A_k) + \epsilon $$
\end{proof}
\begin{prop}
For disjoint $A_k$
$$\lambda^*\qty(\bigcupdot_{k=1}^\infty A_k) \geq \sum_{k=1}^\infty \lambda^*(A_k)$$
\end{prop}


\begin{proof}
	For all $i$ choose $K_i \subseteq A_i$. Choose some $N$, then
	$$\bigcup_{k=1}^N K_k \subseteq \bigcup_{k=1}^\infty A_k$$
	Since $\bigcup_{k=1}^N K_k$ is compact, 
	$$\lambda_*\qty(\bigcup_{k=1}^\infty A_n) \geq \lambda\qty(\bigcup_{k=1}^N K_k)  = \sum_{k=1}^N \lambda(K_k)$$ 
	By taking supremum on $K_i$, we get
	$$\lambda_*\qty(\bigcup_{k=1}^\infty A_n) \geq \sum_{k=1}^N \lambda_*\qty(A_n)$$
\end{proof}
\begin{prop}
	If $A$ is open or compact then
	$$\lambda(A) = \lambda^*(A) = \lambda_*(A)$$
\end{prop}


\begin{proof}
	If $A$ is compact, obviously $\lambda_*(A) = \lambda(A)$, and $\lambda^*(A) = \lambda(A)$ by definition.
	
	For open $A$, obviously $\lambda(A)=\lambda^*(A)$. In addition, for any special polygon $P \subset A$, $\lambda(P) \leq \lambda_*(A)$. However
	$$\lambda^*(A)  = \lambda(A) = \sup \left\{ \lambda(P) : P\subseteq A \right\} \leq \lambda_*(A)$$
	meaning
	$$\lambda^*(A)  = \lambda(A) =  \lambda_*(A)$$
\end{proof}
Denote $$\mathcal{L}_0 = \left\{ A\subset\mathbb{R}^n :: \lambda^*)A_=\lambda_*(A) < \infty  \right\}$$
All compact sets and all open set with finite measure are in $\mathcal{L}_0$.
\begin{prop}
	$$\lambda_*(A) = \lambda_*(A+x)$$
	$$\lambda^*(A) = \lambda^*(A+x)$$
\end{prop}

\begin{definition}
	For set in $\mathcal{L}_0$, $\lambda(A)=\lambda^*(A)=\lambda_*(A)$.
\end{definition}
\begin{lemma}
	If $A,B \in \mathcal{L}_0$ and $A\cap B = \emptyset$ then $A\cup B \in \mathcal{L}_0$ and 
	$$\lambda(A\cup B) = \lambda(A)+\lambda(B)$$
\end{lemma}
\begin{proof}
	$$\lambda^*(A\cup B) \leq \lambda^*(A)+ \lambda^*(B) = \lambda(A) + \lambda(B) = = \lambda_*(A) + \lambda_*(B) \leq \lambda_*(A\cup B)  \leq \lambda^*(A\cup B) $$
\end{proof}
\begin{theorem}
	$A\subseteq \mathbb{R}^n$ with $\lambda^*(A)<\infty$. $A \in \mathcal{L}_0$ iff for all $\epsilon > 0 $ exists compact $K$ and open $G$, $K\subseteq A \subseteq G$ and $\lambda(G\setminus K) < \epsilon$
	
	
	\begin{proof}
		$\Rightarrow$:
		
		Let  $A \in \mathcal{L}_0$ . We can find compact $K$ and open $G$, $K\subseteq A \subseteq G$ such that
		$$\lambda(G) < \lambda^*(A) +\frac{\epsilon}{2}$$
		$$\lambda(K) > \lambda_*(A) - \frac{\epsilon}{2}$$
		Note that, by lemma
		$$\lambda(G) = \lambda(K) + \lambda(G\setminus K)$$
		$$\lambda(G\setminus K) = \lambda(G) -\lambda(K) < \epsilon$$
		
		
		$\Leftarrow$:
		
		$$\lambda^*(A) \leq \lambda(G) = \lambda(K) + \lambda(G\setminus K) <\lambda(K)+\epsilon \leq \lambda_*(A) +\epsilon $$
		Thus $\lambda^*(A)  = \lambda_*(A) $ and $A \in \mathcal{L}_0$.
	\end{proof}
\begin{coll}
	If $A,B\in \mathcal{L}_0$, then $A\cup B, A\cap B, A\setminus B  \in \mathcal{L}_0$
	\begin{proof}
		First, show that $A\setminus B \in \mathcal{L}_0$.
		Take $K_1\subseteq A\subseteq G_1$ and $K_2 \subseteq A\subseteq G_2$.
		$$\lambda(G_1\setminus K_1) < \frac{\epsilon}{2}$$
		$$\lambda(G_2\setminus K_2) < \frac{\epsilon}{2}$$
		
		Denote $K = K_1\setminus G_2$ and $G = G_1 \setminus K_2$.
		$$K\subseteq A\setminus B \setminus G$$
		$$G\setminus K = (G_1\setminus K_1) \cup (G_2\setminus K_2)$$
		$$\lambda(G\setminus K)  \leq \lambda(G_1\setminus K_1) + \lambda(G_2\setminus K_2) < \epsilon$$
		
		Now
		$$A\cup B = (A\setminus B) \cup B \in \mathcal{L}_0$$
		$$A\cap B = A\setminus (A\setminus B)   \in \mathcal{L}_0$$
	\end{proof}
\end{coll}
\end{theorem}

\begin{theorem}
	Let $\left\{ A_k \right\}$ set in $\mathcal{L}_0$ and $A \bigcup_{k=1}^\infty A_k$ such that $\lambda^*(A) < \infty$ then $A\in \mathcal{L}_0$ and 
	$$\lambda(A) \leq \sum_{k=1}^\infty \lambda(A_k)$$
		In addition, if $A_i\cap A_j = \emptyset$,
		$$\lambda(A) = \sum_{k=1}^\infty \lambda(A_k)$$
	\begin{proof}
		Suppose $\left\{ A_k \right\}$ are disjoint.
		$$\lambda^*(A) \leq \sum_{k=1}^\infty \lambda^*(A_k) = \sum_{k=1}^\infty \lambda_*(A_k) \leq \lambda_*(A)$$
		
		Thus $A\in \mathcal{L}_0$ and
		$$\lambda(A) = \lambda^*(A) =  \sum_{k=1}^\infty \lambda^*(A_k) =  \sum_{k=1}^\infty \lambda(A_k)$$
		
		Now generally, define $$B_1 = A_1 \in \mathcal{L}_0$$
		$$B_2 = A_2 \setminus A_1$$
		and so on:
		$$B_k = A_k \setminus \bigcup_{i=1}^{k-1} A_i \in \mathcal{L}_0 $$
		
		Now $\left\{ B_k \right\} $ are disjoint and $A = \bigcup_{k=1}^\infty A_k = \bigcup_{k=1}^\infty B_k \in \mathcal{L}_0 $. Thus
		$$\lambda(A) = \lambda\qty(\bigcup_{k=1}^\infty A_k)=\lambda\qty(\bigcup_{k=1}^\infty B_k) = \sum_{k=1}^\infty \lambda(B_k) \leq \sum_{k=1}^\infty \lambda(A_k)  $$
	\end{proof}
\end{theorem}

\paragraph{Note}Any ball $B(0,R)$ is in $\mathcal{L_0}$, since it is inside special box large enough.
\begin{definition}
	Let $A\subseteq \mathbb{R}^n$, we say $A$ is Lebesgue measurable if $\forall M\in \mathcal{L}_0 \quad A\cap M \in \mathcal{L}_0$. It's measure equals
	$$\lambda(A) = \sup \left\{ \lambda(A\cap M), M \in \mathcal{L}_0 \right\}$$
	
	Denote a set of all such sets as $\mathcal{L}$.
\end{definition}
\begin{prop}
	If $\lambda^*(A) < \infty$, $A\in \mathcal{L} \iff A \in \mathcal{L}_0$. For those sets $\lambda$ definitions are equivalent.
	\begin{proof}
		If $A \in \mathcal{L}_0$ in, then $\forall M\in \mathcal{L}_0 \quad A\cap M \in \mathcal{L}_0$, thus $A\in \mathcal{L}$.
		
		Now, if $A\in \mathcal{L}$ and $\lambda^*(A) < \infty$. For all $N \in \mathbb{N}$,
		$$A\cap B(0,N) \in \mathcal{L}_0$$
		
		However
		$$A = \bigcup_{N=1}^\infty \qty[A\cap B(0,N)]$$
		And $\lambda^*(A) < \infty$, thus  $A \in \mathcal{L}_0$.
		
		Denote
		$$\tilde{\lambda}(A) = \sup \left\{ \lambda(A\cap M), M \in \mathcal{L}_0 \right\}$$
		Obviously, $\tilde{\lambda}(A)\geq\lambda(A) $ (take $M=A$). On the other side,
		$$\forall M\in \mathcal{L}_0 \quad \lambda(A\cap M) \leq \lambda(A)$$
		thus $\tilde{\lambda}(A)=\lambda(A) $
		
	\end{proof}
\end{prop}

\begin{prop}
	$$\emptyset \in \mathcal{L}$$
	\begin{proof}
		$$\emptyset \in \mathcal{L}_0 \Rightarrow \emptyset \in \mathcal{L}$$
	\end{proof}
\end{prop}
\begin{prop}
$$A \in \mathcal{L} \Rightarrow \mathbb{R}^n \setminus A \in \mathcal{L} $$
\begin{proof}
	Take $M\in \mathcal{L}_0$. 
	$$\qty(\mathbb{R}^n \cap A) \cap M = M \setminus A = M \setminus (A\cap M) \in \mathcal{L}_0$$
\end{proof}
\end{prop}
\begin{prop}
$$\left\{ A_i \right\}_{i=1}^\infty \in \mathcal{L} \Rightarrow A = \bigcup A_i \in \mathcal{L} $$
\begin{proof}
	Take $M\in \mathcal{L}_0$. 
	$$A\cap M = \bigcup_{i=1}^\infty (A_k\cap M) $$
	$$\lambda^*(A\cap M) \leq \lambda(M) $$
	Thus
	$$A\cap M \in \in \mathcal{L}_0$$
\end{proof}
\end{prop}

\begin{prop}
	If $\forall N\in \mathbb{N}$, $A\cap B(0,N) \in \mathcal{L}_0$, then $A\in \mathcal{L}$.
\end{prop}

\begin{definition}
	For some set $X$, set $M$ of its subsets is called $\sigma$-algebra if
	\begin{enumerate}
		\item $\emptyset \in M$
		\item $A\in M \Rightarrow X\setminus A\in M$
		\item $\left\{ A_i \right\}_{i=1}^\infty \in M \Rightarrow A = \bigcup A_i \in M $
	\end{enumerate}
\end{definition}

\paragraph{Examples}
\begin{enumerate}
	\item $2^X$ for any $X$ is $\sigma$-algebra
	\item All subsets of $\mathbb{R}$ that are countable or their complement is countable.
	\item All open sets in $\mathbb{R}$ is not $\sigma$-algebra.
\end{enumerate}
\begin{prop}
	If $M$ is $\sigma$-algebra and  $\left\{ A_k \right\}_{k=1}^\infty \subset M$, then
	$$\bigcap_{k=1}^\infty A_k \in M$$
	\begin{proof}
		$$X \setminus \bigcap_{k=1}^\infty A_k = \bigcup_{k=1}^\infty \qty(X\setminus A_k) \in M$$
	\end{proof}
\end{prop}

\begin{prop}
	All open and closed sets are in $\mathcal{L}$
	\begin{proof}
		Let $A$ some open set. Then $A\cap B(0,N)\in \mathcal{L}_0$. Since $\mathcal{L}$ is closed for complementation, also closed sets are in $\mathcal{L}$.
	\end{proof}
\end{prop}

\begin{prop}
	If $\left\{ A_k \right\}_{k=1}^\infty \subset \mathcal{L}$ then
	$$\lambda\qty(\bigcup_{k=1}^\infty A_k) \leq \sum_{k=1}^\infty \lambda(A_k)$$
	\begin{proof}
		Denote $A=\bigcup_{k=1}^\infty A_k$. For $M\in \mathcal{L}_0$
		$$\lambda(A\cap M) = \lambda\qty(\bigcup_{k=1}^\infty \qty(A_k \cap M)) \leq \sum_{k=1}^\infty \lambda(A_k\cap M) \leq \sum_{k=1}^\infty \lambda(A_k)$$
		Since it right for any $M$, 
		$$\lambda\qty(A) \leq \sum_{k=1}^\infty \lambda(A_k)$$
	\end{proof}
\end{prop}

\begin{prop}
If $\left\{ A_k \right\}_{k=1}^\infty \subset \mathcal{L}$ and $A_i\cap A_j=0$ then
$$\lambda\qty(\bigcup_{k=1}^\infty) = \sum_{k=1}^\infty \lambda(A_k)$$
\begin{proof}
	For some $N \in \mathbb{N}$, choose $\left\{ M_p \in \mathcal{L}_0 \right\}_{p=1}^N$. Define $\mathcal{L}_0 \ni M = \bigcup_{p=1}^N M_p$. 
	$$\lambda(A) \geq \lambda(A\cap M) = \sum_{k=1}^\infty \lambda (A_k \cap M) \geq \sum_{k=1}^N \lambda (A_k \cap M) \geq \sum_{k=1}^N \lambda (A_k \cap M_k)  $$
	Thus
	$$\lambda_A \geq \sup \left\{ \sum_{k=1}^N \lambda (A_k \cap M_k), M_k \in \mathcal{L}_0 \right\} = \sum_{k=1}^N \sup \left\{  \lambda (A_k \cap M_k), M_k \in \mathcal{L}_0 \right\} = \sum_{k=1}^N \lambda(A_k) $$
	Since it's right for any $N$,
	$$\lambda_A \geq \sum_{k=1}^\infty \lambda(A_k) $$
\end{proof}
\end{prop}

\begin{theorem}
	The defined $\lambda$ fulfills properties of measure.
	\begin{enumerate}
	\item $0\leq \lambda(A) \leq \infty$
\item $\lambda(\emptyset) = 0$
\item $\lambda(\qty[a_1,b_1]\cross\qty[a_2,b_2]\cross\dots\cross \qty[a_n,b_n]) = \prod_{i=1}^n (b_i-a_i)$
\item If $A = \bigcupdot_{k=1}^\infty A_k$, then $\lambda(A) = \sum_{i=1}^\infty \lambda(A_k)$.
\item If $C$ is acquired from $A$ by rotation or translation $\lambda(C) = \lambda(A)$.
	\end{enumerate}
\end{theorem}

\begin{definition}[Measure]
	For some set $X$, measure of $X$ is function $\mu$ defined on $\sigma$-algebra $M$ of subsets of $X$ and fulfills
	\begin{enumerate}
		\item $0\leq \mu(A) \leq \infty$
		\item $\mu(\emptyset) = 0$
		\item If $A = \bigcupdot_{k=1}^\infty A_k$, then $\lambda(A) = \sum_{i=1}^\infty \lambda(A_k)$.
	\end{enumerate}
\end{definition}
We denote measure space as $\qty(X,\mu, M)$.
\begin{theorem}
	Let  $\qty(X,\mu, M)$ measure space. 
	\begin{enumerate}
		\item If $\left\{ A_k \right\}_{k=1}^\infty \subset M$ and $\forall k \: A_k \subset A_{k+1}$, then
		$$\mu\qty(\bigcup_{k=1}^\infty A_k) = \lim_{k\to \infty} \mu(A_k)$$
		\item If $\left\{ A_k \right\}_{k=1}^\infty \subset M$ and $\forall k \: A_k \supset A_{k+1}$ and $\mu(A_1) < \infty$, then
		$$\mu\qty(\bigcap_{k=1}^\infty A_k) = \lim_{k\to \infty} \mu(A_k)$$
	\end{enumerate}

\begin{proof}
	$$\bigcup_{k=1}^\infty A_k = A_1 \cup \qty[\bigcup_{k=1}^\infty A_{k+1}\setminus A_k]$$
	Since those sets are disjoint
	$$\mu\qty(\bigcup_{k=1}^\infty A_k ) = \mu(A_1) + \sum_{k=1}^\infty \mu(A_{k+1} \setminus A_k)= \lim_{N\to \infty} \mu(A_1) + \sum_{k=1}^N \mu(A_{k+1} \setminus A_k) = \lim_{N\to \infty} \mu\qty(A_1 \cup \qty[\bigcup_{k=1}^N A_{k+1}\setminus A_k])  = \lim_{N\to \infty} \mu\qty(A_{N+1})$$
\end{proof}
\end{theorem}

\begin{prop}[]
	If $\lambda^*(A)=0 $, $A\in \mathcal{L}$ and for any $B\subset A$, $B\in \mathcal{L}$ and $\lambda(B)=0$.
	\begin{proof}
		$$\lambda_*(A) \leq \lambda^*(A) = 0 \Rightarrow A\in \mathcal{L}_0$$
		Monotonity of upper measure
	\end{proof}
\end{prop}
\begin{theorem}
	$A$ is measurable iff $\forall \epsilon>0$ exist open $G$ and closed $F$ such that $$F\subseteq A\subseteq G$$ and $$\lambda(G\setminus F) \leq \epsilon$$
	\begin{proof}
		$\Leftarrow$:
		
		Suppose exist such $G$ and $K$. For all $k$ choose 
		$G_k$ and $F_k$ such that
		$$\lambda(G_k\setminus F_k) < \frac{1}{k}$$
		Denote 
		$$B = \bigcup_{k=1}^\infty F_k$$
		$$\lambda^* (A\setminus B) = 0$$
		and
		$$A\setminus B \subseteq G_k \setminus B \subseteq G_k \setminus F_k$$
		Thus
		$$\lambda^*(A\setminus B) \leq \lambda(G_k \setminus F_k) <\frac{1}{k}$$
		Thus $\lambda^*(A\setminus B) = 0$ and $A\setminus B \in \mathcal{L}$.
		
		However $B \in \mathcal{L}$ and $A = B\cup (A\setminus B)$, thus $A\in \mathcal{L}$.
		
		
		$\Rightarrow$:
		
		Suppose $A\in \mathcal{L}$. Denote $E_k = B(0,k) \setminus B(0,k-1)$. This is partition of $\mathbb{R}^n$. $E_k \in \mathcal{L}_0$ and so is $A\cap E_k \in \mathcal{L}$. 	
		Thus for all $k$ there is
		$$K_k \subseteq A\cap E_k \subseteq G_k$$
		such that $\lambda(G_k\setminus K_k) < \frac{\epsilon}{2^k}$.
		Denote 
		$$F = \bigcup_{k=1}^\infty K_k$$
		$$G = \bigcup_{k=1}^\infty G_k$$
		
		$$\lambda(G\setminus F)= \lambda\qty(\bigcup_{k=1}^\infty (G_k\setminus F)) \leq \lambda\qty(\bigcup_{k=1}^\infty (G_k\setminus K_k)) \leq \sum_{k=1}^\infty\lambda\qty( G_k\setminus K_k) < \epsilon$$
		
		Now, $F$ is closed. Let $F \ni x_k \to x$. The sequence converges and thus bounded, and thus exists $N$ such that $\left\{ x_k \right\} \cup \left\{ x \right\} \in B(0,N)$.
		
		Thus $\left\{ x_k \right\} \subseteq \qty(\bigcup_{i=1}^N E_i) \cap F$ and $\left\{ x_k \right\} \subseteq \bigcup_{i=1}^N K_i$ and thus  $\left\{ x_k \right\} \cup \left\{ x \right\} \in F$.
		
		
		
	\end{proof}
\end{theorem}
\end{document}
