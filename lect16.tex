
\section{Question 1}
$L^1_C(\mathbb{R}^n)$ - functions in $L^1$ with compact support a.e.

$\mathcal{C}^m$ - function with continuous partial derivatives up to order $m$.

$\mathcal{C}^m_c$ - function in $\mathcal{C}^m$ with constant support.

$\mathcal{C}^\infty = \bigcap_{k=1}^\infty \mathcal{C}^k$.


Convergence in $L^1$:
$$f_n\stackrel{L^1}{\to} f  \Rightarrow \int \abs{f_n-f} \dd{\lambda} \to 0$$

If $Y_1\subset Y_2$ then $\bar{Y}_1 \subset \bar{Y}_2$. Also $\bar{\bar{Y}} = \bar{Y}$.

\begin{lemma}
	If $f_n\to f$ a.e. and $\abs{f_n}<\abs{f}$ and $\left\{  f_n, f\right\} \subseteq L^1$ then $f_n\stackrel{L^1}{\to} f  $.
	\begin{proof}
		Since
		$$\int \abs{f_n-f} \dd{\lambda} \stackrel{a.e.}{\to} 0$$
		$$\abs{f_n-f} \leq \abs{f_n}+\abs{f} \leq 2\abs{f} \in L^1$$
		From dominated convergence,
		$$\int\abs{f_n-f} \dd{\lambda} = 0$$
	\end{proof} 
\end{lemma}


\begin{lemma}[Urysohn's lemma ] \label{urysohn}
	Let $B$,$C$ closed disjoint sets, then exists continuous $f:\mathbb{R}^n \to [0,1]$ such that $\eval{f}_C = 1$ and $\eval{f}_B=0$.
\end{lemma}
\begin{lemma}
	Simple function are dense in $L^1$.
	\begin{proof}
		First lets show that $L_c^1$ is dense in $L^1$. Let $f\in L^1$ and denote
		$$f_n = f\chi_{B_0(n)}$$
		Since $f_n \stackrel{a.e.}{\to} f$ and $\abs{f_n} \leq \abs{f}$ we get 
		$$L_c^1 \ni f_n \stackrel{L^1}{\to} f $$
		Thus $L_c^1$ is dense in $L^1$.
		
		Now lets show that simple functions are dense in $L_c^1$. It's enough to show that for nonnegative functions. For each function $f\in L^1_c$ exists sequence $s_k\geq 0$ such that $s_k \stackrel{a.e.}{\to} f$ and $\abs{s_k} \leq \abs{f}$ we get 
		$$s_k \stackrel{L^1}{\to} f $$
		
	\end{proof} 
\end{lemma}


\begin{lemma}
	$C_c$ is dense in $L^1$.
	\begin{proof}
		We want to show that any nonnegative simple function with compact support can be approximated with functions from $C_c$. It's enough to show that for $\chi_A$, where $A$ is closed and measurable. Suppose $A\subseteq B_0(n)$. $\forall \epsilon >0$ exists open $G$ and compact $K$ such that 
		$$K\subseteq A \subseteq G$$
		$G\subseteq B_0(n)$.
		
		From \ref{urysohn}, exists $g: \mathbb{R}^n \to [0,1]$ such that 
		$\eval{g}_{K} = 1$ and $\eval{g}_{\mathbb{R}^n \setminus G}=0$. $g\in C_c$.
		$$g(x) - \chi_A(x) = \begin{cases}
		0&x\in K\\
		0& x \in \mathbb{R}^n \setminus G\\
		\leq 1 & x\in G\setminus K
		\end{cases}$$
		$$\int \abs{g(x) - \chi_A(x)} \dd{\lambda} = \int\limits_{G\setminus K} \abs{g(x) - \chi_A(x)} \dd{\lambda} \leq \lambda(G\setminus K) < \epsilon$$
	\end{proof} 
\end{lemma}

\begin{theorem}
	Let $f\in L_1$ and $f_y(x) = f(x+y)$
	$$\norm{f_y-f} \stackrel{y\to0}{\to} 0$$
	\begin{proof}
		Let $f\in L^1$ and $\epsilon>0$. Exists $g\in C_c$ such that $\norm{f-g} <\frac{\epsilon}{3}$. Suppose $g=0$ outside of $B_0(r)$.
		
		$$\int \abs{f-g} \dd{\lambda} < \frac{\epsilon}{3} $$
		But
		$$\int \abs{f_y-g_y} \dd{\lambda} < \frac{\epsilon}{3} $$
		i.e., $\norm{f_y-g_y} <\frac{\epsilon}{3}$.
		$g$ is uniformly continuous. Thus exists $1>\delta>0$ such that if $\norm{y}<\delta$ then
		$$\abs{g(x+y)-g(x)} <\frac{\epsilon}{3\lambda(B_0(r+1)}$$
		For $\norm{x} > r+1$ 
		$$\abs{g(x+1)-g(x) } = 0$$
		Thus for $\norm{y}<\delta$
		$$\int \abs{g_y(x) -g(x)} \dd{\lambda} = \int\limits_{B_0(r+1)} \abs{g(x+y)-g(x)} \leq \lambda(B_0(r+1)) \frac{\epsilon}{3\lambda(B_0(r+1)} = \frac{\epsilon}{3}$$
		Thus
		$$\norm{f_y-f} \leq \norm{f_y-g_y} + \norm{g_y-g} + \norm{g_y-f} <\epsilon$$		
	\end{proof}
\end{theorem}

\begin{definition}
	We say that $f_n\to f$ in measure if 
	$$\lambda\qty(\left\{ x: \abs{f_n(x) - f(x)} \geq \epsilon  \right\}) = 0$$
\end{definition}

Note that $f_n\stackrel{L^1}{\to} f \Rightarrow f_n\stackrel{\text{measure}}{\to} f$.

\begin{definition}
	$\left\{x_n\right\}$ in metric space $X$ is called Cauchy sequence if for all $\epsilon>0$ exists $N$ such that for all $n,m>N$
	$$\norm{x_n-x_m} < \epsilon$$
\end{definition}

\begin{definition}
	Metric space is called complete space if each Cauchy sequence converges.
\end{definition}
\begin{definition}
	Complete normed space is called Banach space.
\end{definition}
\begin{theorem}
	$L^1$ is Banach space.
	\begin{proof}
		Let $\left\{ f_n \right\}$ Cauchy sequence in $L^1$. For all $\epsilon>0$ exists $N(\epsilon)$ such that $\norm{f_n-f_m}<\epsilon$ for $n,m>N$. We can require that $\left\{ N\qty(\frac{1}{2^k}) \right\}$ is non-decreasing sequence.
		Denote $n_k = N\qty(\frac{1}{2^k})$ and $g_k = f_{n_k}$.
		
		For all $k$:
		$$\int \abs{g_k-g_{k+1} } = \norm{g_k-g_{k+1}} = \abs{f_{n_k} - f_{n_{k+1}}} < \frac{1}{2^k}$$
		Thus
		$$\sum_{k=1}^\infty \int \abs{g_k-g_{k+1} } < \infty$$
		
		We've shown that from that we can conclude that
		$$h(x) = \sum_{k=1}^\infty (g_{k+1}-g_k)$$
		exists and $h\in L^1$.
		
		$$\int h\dd{\lambda} = \sum_{k=1}^\infty g_{k+1}-g_k \dd{\lambda}$$
		$$h(x) = \lim_{N\to \infty} \sum_{k=1}^N  g_{k+1}-g_k = \lim_{N\to \infty} g_N(x) - g_1(x)$$
		
		Thus
		$$\lim_{N\to \infty} g_N(x) = h(x) + g_1(x)$$
		
		On the other hand
		$$\sum_{k=m}^\infty g_{k+1}(x)- g_k(x) =\lim_{N\to \infty} g_N(x) - g_m(x) = g_1(x) + h(x) - g_m(x) $$
		$$\norm{g_1+h-g_m} = \norm{\sum_{k=m}^\infty g_{k+1} - g_k} \leq \sum \norm{g_k-g_{k+1}}\leq \sum_{k=m}^\infty \frac{1}{2^k} \stackrel{m\to \infty}{\to} 0$$
		
		Thus $g_m \stackrel{L_1}{\to} g_1+h$ meaning partial sequence of $\left\{f_n\right\}$ converges and thus  $\left\{f_n\right\}$ converges.
	\end{proof}

\begin{coll}
	If $f_n \stackrel{L_1}{\to}  f$ then exists subsequence $\left\{ f_{n_k} \right\}$ such that $f_n \stackrel{\text{a.e.}}{\to}  f$ 
	\begin{proof}
		$g_k = f_{n_{k}} \to g_1+h$ a.e. and $g_k = f_{n_k} \stackrel{L_1}{\to}  g_1+h$  but $g_k \stackrel{L_1}{\to}  f_1$ thus $f=g_1+h$ a.e. and thus $g_k\to f$ a.e.
	\end{proof}
\end{coll}
\end{theorem}

\begin{prop}
	In normed space if exists $\sum x_n$ then
	$$\norm{\sum x} \leq \sum \norm{x}$$
\end{prop}
