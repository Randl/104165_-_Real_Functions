\begin{lemma}
	Let $A\subseteq \mathbb{R}$ with positive measure, and let $\epsilon>0$ then there exists an interval $J\subseteq \mathbb{R}$ $\frac{\lambda(A\cap J)}{\lambda(J)} = 1-\epsilon$
	\begin{proof}
		Denote $C=\lambda(A)>0$.
		$$\lambda(A) = \lambda^*(A) = C$$
		Thus exists open $G\supseteq A$ such that $\lambda(G) < \qty(1+ \frac{\epsilon}{2})C$.
		
		Since $G$ is open, it is disjoint union of open intervals:
		$$G = \bigcup_{i=1}^\infty J_i$$
		$$\qty(1+ \frac{\epsilon}{2})C > \lambda(G) = \sum \lambda(J_i)$$
		
		Assume that $\forall i \: \lambda(A\cap J) leq (1-\epsilon) \lambda(J)$. Then
		$$C = \lambda(A) = \lambda\qty(A \cap \qty(\bigcup_{i=1}^\infty J_i)) = \sum_{i=1}^\infty \lambda(A\cap J_i) \leq (1-\epsilon) \sum \lambda_J = (1-\epsilon)\lambda(G) = (1-\epsilon)\qty(1+\frac{\epsilon}{2})C < C$$
	\end{proof}
\end{lemma}

\begin{theorem}
	Let $A\subset \mathbb{R}$ measurable set with positive measure.
	$A-A = \left\{ x-y|x,y\in A \right\}$.
	
	\begin{proof}
		If $A$ has non-empty interior, the theorem is obvious. since there exists $a\in A$, $(a-\delta, a+\delta) \subset A$ and thus $(-\delta, \delta) \subset A-A$.
		
		$$t\in A-A \iff A+t\cap A \neq \emptyset$$
		Let $J=(a,b)$ from previous lemma with $\epsilon =\frac{1}{3}$. Assume $t\notin A-A$, i.e. $A\cap(A+t) =\emptyset$. And thus
		$$(A\cap J) \cap \qty[(A+t) \cap (J+t)] = \emptyset$$
		$$\lambda(A\cap J) \geq \frac{2}{3} \lambda(J)$$
		$$\frac{2}{3}\lambda(J) + \frac{2}{3}\lambda(J) \leq \lambda(A\cap J) + \lambda\qty\big((A+t) \cap (J+t)) =\lambda\qty((A\cap J) \cup \qty\big[(A+t) \cap (J+t)]) \leq \lambda(J \cup (J+t))$$
		
		Now, if $t\geq 0$, $J \cup (J+t) \subseteq (a,b+t)$, and if $t<0$, $J \cup (J+t) \subseteq (a+t,b)$. Anyway
		$$\frac{4}{3}\lambda(J) \leq \lambda(J \cup (J+t)) \leq \lambda(J) + \abs{t} $$
		i.e.,
		$$\abs{t} \geq \frac{1}{3}\lambda(J)$$
		Thus $\forall\: 0<t < \frac{1}{3}\lambda(J)$, $(-t,t)\subseteq A-A$.
	\end{proof}
\end{theorem}

Let $a$ set of subsets in $\mathbb{R}^n$. Exists $\sigma$-algebra that is superset of $a$, and also
$$\bigcup \left\{ m: \quad a\subset m  |; \sigma\text{-algebra} \right\}$$
is $\sigma$-algebra and is called $\sigma$-algebra  generated by $a$.

Denote $\mathcal{B}$  $\sigma$-algebra  generated by all open sets in $\mathbb{R}^n$. $\mathcal{B}$ is Borel $\sigma$-algebra. Since all open sets are in $\mathcal{L}$, $\mathcal{B} \subseteq \mathcal{L}$.

\begin{theorem}
	Let measurable $A\subseteq \mathbb{R}^n$, we can write $A=E\cup N$, such that
	\begin{enumerate}
		\item $E\cap N=0$
		\item $E\in \mathcal{B}$
		\item $\lambda(N)=0$ 
	\end{enumerate}
\begin{proof}
	For all $k\in \mathbb{N}$, find
	$$F_k\subseteq A\subseteq G_k$$
	$G_k$ open and $F_k$ closed, and
	$$\lambda(G_k\setminus F_k) \leq \frac{1}{k}$$
	Denote $E = \bigcup_{k=1}^\infty F_k \in \mathcal{E}$. $N=A\setminus E \in \mathcal{L}$.
	
	$$\lambda(N) = \lambda(A\setminus E) \leq \lambda(G_k\setminus F_k) <\frac{1}{k}$$
	i.e., $\lambda(N)=0$.
\end{proof}

\end{theorem}

\paragraph{Reminder}
$f: E\to \mathbb{R}^n$ is continuous iff $\forall \quad G\subseteq \mathbb{R}^n$, $f^{-1}(G) $ is open in $E$.

\begin{theorem}
	Let $f: E \to \mathbb{R}^n$ be continuous for Borel set $E\subseteq \mathbb{R}^n$. Then $f^{-1}(\mathcal{B}) \subseteq \mathcal{B}$.
	\begin{proof}
		Let
		$$m = \left\{ A\subseteq \mathbb{R}^n : f^{-1}(A) \in \mathcal{B} \right\}$$
		We need to show that $\mathcal{B} \subseteq m$, i.e., that $m$ is $\sigma$-algebra containing all open sets.
		
		$\emptyset \in m$, since $\emptyset = f^{-1}\qty(\emptyset)$.
		
		If $\left\{ A_k \right\} \subseteq m$, then $f^{-1}(A_k) \in \mathcal{B}$ and
		$$f^{-1}\qty(\bigcup_k A_k) = \bigcup_k f^{-1}(A_k) \in \mathcal{B}$$
		
		If $A\in m$, then
		$$f^{-1}(\mathbb{R}^n \setminus A) = E\setminus f^{-1}(A) \in \mathcal{B} $$
		
		
		Now lets show that all open sets are in $m$. If $G$ is open,
		$$f^{-1}(G) = E\cap U_G \in \mathcal{B}$$
	\end{proof}
\end{theorem}

\begin{theorem}
	There exists measurable set in $\mathbb{R}$ which is not Borel.
	\begin{proof}
		Define $f: [0,1]:\mathbb{R}$. Let $x$ in ternary basis $0.a_1a_2\dots$. Then
		$$f(x) = \frac{1}{2^N}+\sum_{1}^{N-1} \frac{1}{2^n} \frac{a_n}{2} $$
		where $N$ is first index such that $a_N=1$.
		
		Note that $f$ is constant on $I\subset [0,1]$ such that $I \not\subset C$ (Cantor set).
		
		$f$ is monotonous and onto, and thus continuous.
		
		Define also $g(x) = x+f(x)$, which is one-to-one and onto, thus it is homeomorphism.
	\end{proof}
\end{theorem}
