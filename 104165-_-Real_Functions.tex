\documentclass[]{article}
\usepackage{amsmath}
\usepackage{amsfonts}
\usepackage{amssymb}
\usepackage{hyperref}
\usepackage{gensymb}
\usepackage{graphicx}
\usepackage{svg}
\usepackage{bbding}
\usepackage{mathtools}
\usepackage{centernot} % not parallel, etc.
\usepackage{lmodern}
\usepackage{morewrites}
\usepackage{xcolor,sectsty} % colorful sections
\usepackage[left=10mm, top=10mm, right=10mm, bottom=20mm, nohead]{geometry}
%\usepackage{bigints}
\usepackage{dsfont} %mathbb 1
\usepackage{esint} % beatiful integrals
\usepackage[arrowdel]{physics}
\usepackage{amsthm}

\usepackage[T1]{fontenc}
% Nicer default font (+ math font) than Computer Modern for most use cases
% \usepackage{mathpazo} % problems with greek vectors
\usepackage[utf8x]{inputenc} % Allow utf-8 characters in the tex document
% Prevent overflowing lines due to hard-to-break entities
\sloppy 
% Colors for the hyperref package
\definecolor{urlcolor}{rgb}{0,.145,.698}
\definecolor{linkcolor}{rgb}{.71,0.21,0.01}
\definecolor{citecolor}{rgb}{.12,.54,.11}
% Setup hyperref package
\hypersetup{
	breaklinks=true,  % so long urls are correctly broken across lines
	colorlinks=true,
	urlcolor=urlcolor,
	linkcolor=linkcolor,
	citecolor=citecolor,
}


\DeclareFontFamily{OMX}{lmex}{}
\DeclareFontShape{OMX}{lmex}{m}{n}{<-> lmex10}{}


%colors of sections
\definecolor{secfont}{RGB}{46,116,181}
\definecolor{subfont}{RGB}{146,23,57}
\definecolor{parfont}{RGB}{19,127,43}
\definecolor{subparfont}{RGB}{7,11,100}

\subsectionfont{\color{subfont}}
\sectionfont{\color{secfont}}
\paragraphfont{\color{parfont}}
\subparagraphfont{\color{subparfont}}



% declare a new theorem style
\newtheoremstyle{bluestyle}%
{3pt}% Space above
{3pt}% Space below 
{}% Body font
{}% Indent amount
{\bfseries\color{blue}}% Theorem head font
{.}% Punctuation after theorem head
{.5em}% Space after theorem head
{}% Theorem head spec (can be left empty, meaning ‘normal’)
% declare a new theorem style
\newtheoremstyle{redstyle}{3pt}{3pt}{}{}{\bfseries\color{red}}{.}{.5em}{}
\newtheoremstyle{olivestyle}{3pt}{3pt}{}{}{\bfseries\color{olive}}{.}{.5em}{}
\newtheoremstyle{orangestyle}{3pt}{3pt}{}{}{\bfseries\color{orange}}{.}{.5em}{}
\newtheoremstyle{magentastyle}{3pt}{3pt}{}{}{\bfseries\color{magenta}}{.}{.5em}{}

\theoremstyle{bluestyle}
\newtheorem{theorem}{Theorem}[section]
\theoremstyle{redstyle}
\newtheorem{definition}{Definition}[section]
\theoremstyle{magentastyle}
\newtheorem{coll}{Collary}[theorem]
\theoremstyle{olivestyle}
\newtheorem{lemma}{Lemma}[section]
\theoremstyle{olivestyle}
\newtheorem{prop}[theorem]{Proposition}
%\usepackage{babel}[english]
%opening
\title{104165 - Real functions}
\author{Baruch Solel}
% Johns Lebugue Integration
% Roiden Rudin


% disjoint union
\makeatletter
\def\moverlay{\mathpalette\mov@rlay}
\def\mov@rlay#1#2{\leavevmode\vtop{%
		\baselineskip\z@skip \lineskiplimit-\maxdimen
		\ialign{\hfil$\m@th#1##$\hfil\cr#2\crcr}}}
\newcommand{\charfusion}[3][\mathord]{
	#1{\ifx#1\mathop\vphantom{#2}\fi
		\mathpalette\mov@rlay{#2\cr#3}
	}
	\ifx#1\mathop\expandafter\displaylimits\fi}
\makeatother

\newcommand{\cupdot}{\charfusion[\mathbin]{\cup}{\cdot}}
\newcommand{\bigcupdot}{\charfusion[\mathop]{\bigcup}{\cdot}}
\DeclareMathOperator{\inter}{int}



%\newtheorem{proof}{Proof}[theorem]


\parindent=0em
\begin{document}


\maketitle

\begin{abstract}

\end{abstract}

%\tableofcontents
\section{Introduction}
If $\forall x \quad f_n(x) \to f(x)$ (pointwise) does $\int_0^1 f_n(x) \dd{x} \to \int_0^1 f(x) \dd{x}$?

Define $f_n(x) = \chi_{r_1, r_2, \dots r_n}$, where $\left\{ r_i \right\} = \mathbb{Q} \cap [0,1]$, i.e., first $n$ rational numbers. Those functions are integrable since they are non-zero in finite number of points. However, $f(x) = \chi_{\mathbb{Q} \cap [0,1]}$ is not integrable.
\paragraph{Riemann integral: limit}
We defined Riemann integral as limit of Riemann sum:
$$\int_a^b f(x) \dd{x} = \lim \sum f(x'_i)(x_{i+1}-x_i)$$

\begin{center}
	\includesvg{lect1/l1pic1.svg}
\end{center}

By dividing on $y$, we bound the error by the size of each interval, $\epsilon$:
$$g(x) = s\chi_{A_1} + \qty(s+\epsilon)\chi_{A_2} + \dots$$
$$\forall x \quad \abs{g(x)-f(x)} \leq \epsilon
$$
\section{Measure}
For $A\subseteq \mathbb{R}$ we want to define size of $A$ which we will denote $\lambda(A)$. What do we require from $\lambda$?
\begin{enumerate}
	\item $\lambda(\qty[a,b]) = b- a$
	\item $0\leq \lambda(A) \leq \infty$
	\item $\lambda(\emptyset) = 0$
	\item If $A = \bigcup_{k=1}^\infty A_k$ and $\forall i,j \quad A_i \cap A_j = \emptyset$, then $\lambda(A) = \sum_{i=1}^\infty \lambda(A_k)$.
	\item $\lambda(A+x) = \lambda(A)$, where $A+x = \left\{s+x: a\in A \right\}$.
\end{enumerate}

From those properties we get additional properties:
\begin{itemize}
	\item Additivity:
	$$A = \bigcupdot_{i=1}^n A_i  \Rightarrow \lambda(A) = \sum_{i=1}^n \lambda(A_i)$$
	\item  If $A \subseteq B$, then $\lambda(A) \leq \lambda(B)$.
\end{itemize}
\paragraph{Theorem} Function $\lambda$ fulfilling 1-5 and defined on every subset of $\mathbb{R}$ doesn't exist.
\subparagraph{Proof}
Suppose there exists such $\lambda$.

Define equivalence relation $x\sim y$ iff $x-y \in \mathbb{Q}$. Define $E$ choose from each equivalence class one representative from $\qty[0, \frac{1}{2}]$. Note that if $q_1\neq q_2$, then $q_1+E \cap q_2 + E = \emptyset$, since else $e_1-e_2=q_1-q_2$ and $e_1\sim e_2$, in contradiction.

From definition $E \subset \qty[0, \frac{1}{2}]$. Take a look at
$$\bigcup_{k=2}^\infty \qty(\frac{1}{k}+E) \subseteq \qty[0,1]$$
Thus
$$\lambda \qty(\bigcup_{k=2}^\infty \qty(\frac{1}{k}+E) ) \leq \lambda\qty([0,1])=1$$
On the other hand
$$\lambda \qty(\bigcup_{k=2}^\infty \qty(\frac{1}{k}+E) ) = \sum_{k=2}^\infty \lambda\qty(\frac{1}{k}+E) ) = \lambda\qty(E) )$$
Thus $\lambda(E)=0$. However, 
$$\mathbb{R} = \bigcupdot_{r\in \mathbb{Q}} r+E$$
From sigma-additivity
$$\lambda(\mathbb{E}) = \sum_{r\in \mathbb{Q}} \lambda(r+E) = 0$$
But $\lambda(\mathbb{R}) \geq \lambda([0,1])$, in contradiction.
\paragraph{Reqirements for measure in $\mathbb{R}$}

\begin{enumerate}
	\item $0\leq \lambda(E) \leq \infty$
	\item $\lambda(\emptyset) = 0$
	\item $\lambda(\qty[a_1,b_1]\cross\qty[a_2,b_2]\cross\dots\cross \qty[a_n,b_n]) = \prod_{i=1}^n (b_i-a_i)$
	\item If $A = \bigcupdot_{k=1}^\infty A_k$, then $\lambda(A) = \sum_{i=1}^\infty \lambda(A_k)$.
	\item If $C$ is acquired from $A$ by rotation or translation $\lambda(C) = \lambda(A)$.
\end{enumerate}
\paragraph{Note} In $\mathbb{R}^3$ it is impossible to define measure that fulfills those requirements eve if we replace sigma-additivity with additivity.
\paragraph{Banach–Tarski paradox} Denote $B$ -- unit ball in $\mathbb{R}^3$. We can write 
$$B = \bigcupdot_{i=1}^5 A_i$$
and find $C_i$ by rotation or translation of $A_i$ such that $\bigcupdot_{i=1}^5 C_i$ is two unit balls.

\begin{center}
	\includesvg{lect2/l2pic1.svg}
\end{center}
\subsection{Construction of $\lambda$}
\paragraph{Special boxes} 
Let $E$ box with edges parallel to axes:
$$E = \qty[a_1,b_1]\cross\qty[a_2,b_2]\cross\dots\cross \qty[a_n,b_n]$$

For $E$ we define 
 $$\lambda(E) = \prod_{i=1}^n (b_i-a_i)$$
 
 \paragraph{Special polygons} is a finite union of special boxes. 
 \subparagraph{Note} Each special polygon is a finite union of special boxes with disjoint interior.
 
 Let $P$ is special polygon written as $P = \bigcap_{i=1}^k A_i$ where $A_i$ is special box and their interior is disjoint.
 $$\lambda(P) = \sum_{i=1}^k \lambda(A_i)$$
 \paragraph{Claim}
 \begin{enumerate}
 	\item The definition is independent on choice of $A_i$.
 	\item If $P_1$, $P_2$ are special polygons and $P_1\subseteq P_2$ then $\lambda(P_1)\leq \lambda(P_2)$.
 	\item If $P_1$, $P_2$ are  special polygons with disjoint interior then
 	$$\lambda(P_1\cup P_2) =  \lambda(P_1)+\lambda(P_2)$$
 	\item For all $x\in \mathbb{R}^n$ 
 	$$\lambda(x+P) = \lambda(P)$$
 \end{enumerate}
 \subparagraph{Proof}
  \begin{enumerate}
 	\item Let $P = \bigcap A_i = \bigcap B_i$.
 	
 	If we continue edges of both $A_i$ and $B_i$ we'll get net which divides $P$ into $C_i$ which refines both $A_i$ and $B_i$ and thus
 	$$\lambda(P) = \sum_i \lambda(A_i) = \sum_i \lambda(B_i) = \sum_i \lambda(C_i)$$ 
 	\item Let $P_2= \bigcap A_i$ and choose the refinement which divides $P_1$. 
 	\item Find $A_i$ which divides both $P_1$ and $P_2$.
 	\item $\dots$
 \end{enumerate}
\subparagraph{Alternative proof}
For special boxes
$$\lambda(E) = \lim_{N\to \infty} \frac{1}{N^n} \abs{E \cap \frac{1}{N}\mathbb{Z}^n}$$

For $n=1$, $I=[a,b] \subseteq \mathbb{R}$. We claim
$$b-a = \lim_{N\to \infty} \frac{1}{N} \abs{E \cap \frac{1}{N}\mathbb{Z}}$$
First of all
$$b-a-1 \leq \abs{[a,b] \cap \mathbb{Z}} \leq b-a+1$$
To find $\abs{[a,b] \cap \frac{1}{2}\mathbb{Z}}$, we can use $\abs{[2a,2b] \cap \mathbb{Z}}$, which means

$$2b-2a-1 \leq \abs{E \cap \frac{1}{2}\mathbb{Z}} \leq 2b-2a+1$$

And for any $N$:
$$Nb-Na-1 \leq \abs{[a,b] \cap \frac{1}{N}\mathbb{Z}} \leq Nb-Na+1$$
$$b-a-\frac{1}{N} \leq \frac{1}{N}\abs{[a,b] \cap \frac{1}{N}\mathbb{Z}} \leq b-a+\frac{1}{N}$$
By sandwich rule, we get the equality.

We can do the same for higher dimension and for open sets, and then we can easily proof the claim.

If $P$ is special polygon and we take $\lim_{N\to \infty} \frac{1}{N^n} \abs{P \cap \frac{1}{N}\mathbb{Z}^n} = \sum \lambda(A_i) $
when $P=\bigcap A_i$

\paragraph{Open sets}
\subparagraph{Definition} $G$ is open if $\forall x\in G$ exists ball $B(x,r)$ such that $B\subset G$. Alternatively we can replace ball with special box.

Thus for any open $G\neq \emptyset$ 
$$G = \bigcup \left\{ P \text{ special polygon}  \right\}$$

And we can define
$$\lambda(G) = \sup \left\{ \lambda(P) | P\subseteq G \right\}$$

\paragraph{Claim}
\begin{enumerate}
	\item $$0 \leq \lambda(G) \leq \infty$$
	\item $$\lambda(G) = 0 \iff G =\emptyset$$
	\item $$\lambda(\mathbb{R}^n) = \infty$$
	\item $$G_1 \subseteq G_2 \Rightarrow \lambda(G_1) \leq \lambda(G_2)$$
	\item $$\lambda\qty(\bigcup_{k=1}^\infty G_k) \leq \sum \lambda(G_k)$$
	\item $$\lambda\qty(\bigcupdot_{k=1}^\infty G_k) = \sum \lambda(G_k)$$
	\item $$\lambda(P) = \lambda(\inter P ) =\inf \left\{ \lambda(G) : P\subseteq G\right\}$$
	\item $$\lambda(x+G) = \lambda(G)$$
\end{enumerate}

\begin{lemma}
Let $K\subseteq \mathbb{R}^n$ compact set and $\left\{ G_i \right\}_{i\in I}$ open cover ($K\subseteq \bigcup G_i$). Then exists $\epsilon>0$ such that $\forall x\in K$ exists $i\in I$ such that $B(x,\epsilon) \subseteq G_i$.
\end{lemma}
\begin{lemma} \label{lemma_poly_inf}
	For all polygon of dimension $P$ 
	$$\lambda(P) = \inf \left\{ \lambda(G) : P\subset G  \right\}$$

\end{lemma}
	\begin{proof}
	$$P \subseteq G \Rightarrow \lambda(P) \leq \lambda(G)$$
	Infimum would give
	$$\lambda(P) \leq \inf \left\{ \lambda(G) : P\subset G  \right\}$$
	
	
	Write $P = \bigcup_{k=1}^N I_k$. Then
	$$\lambda(P)  =\sum_{k=1}^N \lambda(I_k)$$
	For $\epsilon$ find $I_k^\epsilon$ such that
	$$\begin{cases}
	\inter I_k^\epsilon \supseteq I_k\\
	\lambda(I_k^\epsilon) \leq \lambda(I_k) + \frac{\epsilon}{N}
	\end{cases}$$
	Denote $G = \bigcup_{k=1}^N \inter (I_k^\epsilon)$, then, from subadditivity
	$$\lambda(G) \leq \sum_{k=1}^N \lambda (\inter I_k^\epsilon) = \sum_{k=1}^N \lambda (I_k^\epsilon) \leq \epsilon + \sum_{k=1}^N \lambda(I_k) $$
	
	In addition,
	$$\inf \lambda(G) \leq \lambda(P) $$
\end{proof}
\subparagraph{Proof} %TODO
\begin{enumerate}
	\item Obvious
	\item If $G$ is not empty, exists $x\in G$ and special box around $x$ such that $P\subseteq G$and thus $\lambda(G) \leq \lambda(P) > 0$
	\item Any box is subset of $\mathbb{R}^n$ thus $\lambda(\mathbb{R}^n) = \infty$
	\item Obvious
	\item Let $P$ special polygon, $P\subseteq \bigcup_{k=1}^\infty G_k$. We'll show that it's possible to write 
	$$P = \bigcup_{j=1}^N I_j$$
	finite union of special boxes with disjoint interior and for each $j$ exists $k$ such that $I_j \subset G_k$. Let $\epsilon$ from lemma for $K=P$. Write $P=\bigcup_{j=1}^N = I_j$ such that diameter of each $I_j<\epsilon$. If $x_j$ is center of $I_j$, then $I_j \subseteq B(x_j,\epsilon) \subseteq G_k$.
	
	If this is possible, for such $P$ denote $$P_k =  \bigcup_{j=1}^\infty I_j | I_j \subset G_k, \forall i<k \quad I_j \not\subset G_i$$
	
	Obviously $\bigcup P_k = P$ and union is finite since for some $m$, for every $k>m$ $P_m=\emptyset$, because there is finite number of $I_j$, and also internals of $P_k$ are disjoint.
	
	Thus $\lambda(P)=\sum \lambda(P_k)\leq\sum \lambda(G_k)$.
	This is right for any $P$, thus
	$$ \lambda\qty(\bigcup (G_k)) = \sup \left\{ \lambda(P) | P\subseteq \bigcup (G_k) \right\} \leq \sum_{k=1}^\infty \lambda(G_k)$$
	\item Since we have sigma-subadditivity, we need only one direction of inequality:
	$$\lambda(G_k) = \sup\left\{ \lambda(P) : P\subseteq G_k \right\}$$
	For any $N$
	$$\sum_{k=1}^N \lambda(G_k) = \sup\left\{ \sum_{k=1}^N \lambda(P_k) : P_k\subseteq G_k \right\} = \sup\left\{ \lambda\qty(\bigcupdot_{k=1}^N P_k) : P_k\subseteq G_k \right\} \leq \lambda\qty(\bigcupdot_{k=1}^N G_k) \leq \lambda\qty(\bigcupdot_{k=1}^\infty G_k) $$
	i.e.,
	$$\sum_{k=1}^\infty \lambda(G_k)  \leq \lambda\qty(\bigcupdot_{k=1}^\infty G_k) $$
	\item First, proof that $\lambda(P) = \lambda(\inter P)$. If $I=P$ is non-empty special box $I = [a_1,b_1]\times [a_2,b_2]\times \dots \times [a_n,b_n]$. For any $\epsilon>0$, $I_\epsilon = [a_1+\epsilon,b_1-\epsilon]\times [a_2+\epsilon,b_2-\epsilon]\times \dots \times [a_n+\epsilon,b_n-\epsilon]$. $I_\epsilon \subseteq \inter I$.
	
	That means that $\lambda(I_\epsilon) \leq \lambda(\inter I)$. Obviously, $\lambda(I_\epsilon)\to \lambda(I)$, i.e. $\lambda(I) \leq \lambda(\inter I)$.
	
	Generally, for $P = \bigcup_{k=1}^N I_k$, 
	$$\inter P \geq \bigcup_{k=1}^N \inter I_k$$
	thus
	$$\lambda(\inter P) \geq \lambda \qty(\bigcup_{k=1}^N \inter I_k) = \sum_{k=1}^N \lambda(\inter I_k) \geq \sum_{k=1}^N \lambda(I_k) = \lambda(P)$$
	For any $P$ 
	$$\lambda(\inter P) \geq \lambda P$$
	
	However
	$$\lambda(\inter P) = \sum \left\{ \lambda(Q) : Q \subseteq \inter P  \right\}$$
	$$Q \subseteq P \Rightarrow \lambda(Q) \leq \lambda(P) \Rightarrow \lambda(\inter P) \leq \lambda(P)$$
	
	Second part is obvious from Lemma \ref{lemma_poly_inf}.
	\item Obvious since it's right for polygons
\end{enumerate}
\subsection{Compact sets}
\begin{definition}
	For compact $K \subseteq \mathbb{R}^n$
	$$\lambda(K) = \inf \left\{ \lambda(G) : K\subseteq G \quad G \text{ is open} \right\}$$
\end{definition}
\begin{prop}
	$$0 \leq \lambda(K) < \infty$$
\end{prop}
\begin{proof}
	Each $K$ is subset of open box $A$ and $\lambda(A) < \infty$
\end{proof}
\begin{prop}
	$$K_1 \subseteq K_2 \Rightarrow \lambda(K_1) \leq \lambda(K_2)$$
\end{prop}
\begin{proof}
	Obvious
\end{proof}
\begin{prop}
	Subadditivity
	$$\lambda(K_1 \cup K_2) \leq \lambda(K_1) + \lambda(K_2)$$
\end{prop}
\begin{proof}
$$K_i \subseteq G_i$$
$$K_1 \cup K_2 \subseteq G_1 \cup G_2$$
$$\lambda(K_1 \cup K_2) \leq \lambda(G_1 \cup G_2) \leq \lambda(G_1) + \lambda( G_2)$$
Thus
$$\lambda(K_1 \cup K_2) \leq \lambda(K_1) + \lambda(K_2)$$
\end{proof}
\begin{prop}
	$$K_1 \cup K_2 = \emptyset \Rightarrow \lambda(K_1 \cup K_2) = \lambda(K_1) + \lambda(K_2)$$
\end{prop}
\begin{proof}
For $K_1$, $K_2$ exists $\epsilon>0$ such that $\forall \: x \in K_1 \: y \in K_2$, $d(x,y)\geq\epsilon$. Denote 
$$U_i = \bigcup_{x\in K_i} B\qty(x, \frac{\epsilon}{2}) \supset K_i$$
Let $K_1 \cup K_2 \subset G_i$, since  $K_i \subset U_i$, 
$$K_i \subset G \cap U_i$$
i.e.,
$$\forall i \quad \lambda(K_i) \leq \lambda(G\cap U_i)$$

Since $U_1 \cap U_2 = \emptyset$ (from construction)
$$(G\cap U_1) \cap (G \cap U_2) = \emptyset$$
$$\lambda(G\cap U_1) + \lambda (G \cap U_2) = \lambda \qty\big((G\cap U_1) \cap (G \cap U_2)) \leq \lambda(G)$$
Thus
$$\lambda(G) \geq \lambda(G\cap U_1) + \lambda (G \cap U_2) \geq \lambda(K_1)+\lambda(K_2)$$
i.e.,
$$\lambda(K_1\cup K_2) \geq \lambda(K_1)+\lambda(K_2)$$
\end{proof}

\subsection{General sets}
Define outer and inner measure similar to Darboux sums:
$$\lambda^*(A) = \inf \left\{ \lambda(G) : A\subset G \text{, open} \right\}$$
$$\lambda_*(A) = \sup \left\{ \lambda(K) : A\supset G \text{, compact} \right\}$$

\begin{prop}
	$$\lambda_*(A) \leq \lambda^*(A)$$
\end{prop}
\begin{proof}
	If $G$ is open and $K$ compact and $K\subset A\subset G$ then $K\subset G$, i.e. $\lambda(K) \leq \lambda(G)$. From that, taking supremum on $K$ and infimum on $G$, we get the required result.
\end{proof}
\begin{prop}
$$A\subset B \Rightarrow \lambda^*(A) \leq \lambda^*(B) \quad \lambda_*(A) \leq \lambda_*(B)$$
\end{prop}
\begin{proof}
Obvious.
\end{proof}
\begin{prop}
$$\lambda^*\qty(\bigcup_{k=1}^\infty A_k) \leq \sum_{k=1}^\infty \lambda^*(A_k)$$
\end{prop}

\begin{proof}
	
	$$\lambda^*(A) = \inf \left\{ \lambda(G) : A\subset G \text{, open} \right\}$$
	Thus exists $G_k$ such that
	$$\lambda(G_k) < \lambda^*(A_k) + \frac{\epsilon}{2^k}$$
	$$\lambda^*\qty(\bigcup_{k=1}^\infty A_k) \leq \lambda\qty(\bigcup_{k=1}^\infty G_k) \leq \sum_{k=1}^\infty \lambda\qty(G_k) < \sum_{k=1}^\infty \qty(\lambda^*\qty(A_k) + \frac{\epsilon}{2^k}) =\lambda^*\qty(A_k) + \epsilon $$
\end{proof}
\begin{prop}
For disjoint $A_k$
$$\lambda^*\qty(\bigcupdot_{k=1}^\infty A_k) \geq \sum_{k=1}^\infty \lambda^*(A_k)$$
\end{prop}


\begin{proof}
	For all $i$ choose $K_i \subseteq A_i$. Choose some $N$, then
	$$\bigcup_{k=1}^N K_k \subseteq \bigcup_{k=1}^\infty A_k$$
	Since $\bigcup_{k=1}^N K_k$ is compact, 
	$$\lambda_*\qty(\bigcup_{k=1}^\infty A_n) \geq \lambda\qty(\bigcup_{k=1}^N K_k)  = \sum_{k=1}^N \lambda(K_k)$$ 
	By taking supremum on $K_i$, we get
	$$\lambda_*\qty(\bigcup_{k=1}^\infty A_n) \geq \sum_{k=1}^N \lambda_*\qty(A_n)$$
\end{proof}
\begin{prop}
	If $A$ is open or compact then
	$$\lambda(A) = \lambda^*(A) = \lambda_*(A)$$
\end{prop}


\begin{proof}
	If $A$ is compact, obviously $\lambda_*(A) = \lambda(A)$, and $\lambda^*(A) = \lambda(A)$ by definition.
	
	For open $A$, obviously $\lambda(A)=\lambda^*(A)$. In addition, for any special polygon $P \subset A$, $\lambda(P) \leq \lambda_*(A)$. However
	$$\lambda^*(A)  = \lambda(A) = \sup \left\{ \lambda(P) : P\subseteq A \right\} \leq \lambda_*(A)$$
	meaning
	$$\lambda^*(A)  = \lambda(A) =  \lambda_*(A)$$
\end{proof}
Denote $$\mathcal{L}_0 = \left\{ A\subset\mathbb{R}^n :: \lambda^*)A_=\lambda_*(A) < \infty  \right\}$$
All compact sets and all open set with finite measure are in $\mathcal{L}_0$.
\begin{prop}
	$$\lambda_*(A) = \lambda_*(A+x)$$
	$$\lambda^*(A) = \lambda^*(A+x)$$
\end{prop}

\begin{definition}
	For set in $\mathcal{L}_0$, $\lambda(A)=\lambda^*(A)=\lambda_*(A)$.
\end{definition}
\begin{lemma}
	If $A,B \in \mathcal{L}_0$ and $A\cap B = \emptyset$ then $A\cup B \in \mathcal{L}_0$ and 
	$$\lambda(A\cup B) = \lambda(A)+\lambda(B)$$
\end{lemma}
\begin{proof}
	$$\lambda^*(A\cup B) \leq \lambda^*(A)+ \lambda^*(B) = \lambda(A) + \lambda(B) = = \lambda_*(A) + \lambda_*(B) \leq \lambda_*(A\cup B)  \leq \lambda^*(A\cup B) $$
\end{proof}
\begin{theorem}
	$A\subseteq \mathbb{R}^n$ with $\lambda^*(A)<\infty$. $A \in \mathcal{L}_0$ iff for all $\epsilon > 0 $ exists compact $K$ and open $G$, $K\subseteq A \subseteq G$ and $\lambda(G\setminus K) < \epsilon$
	
	
	\begin{proof}
		$\Rightarrow$:
		
		Let  $A \in \mathcal{L}_0$ . We can find compact $K$ and open $G$, $K\subseteq A \subseteq G$ such that
		$$\lambda(G) < \lambda^*(A) +\frac{\epsilon}{2}$$
		$$\lambda(K) > \lambda_*(A) - \frac{\epsilon}{2}$$
		Note that, by lemma
		$$\lambda(G) = \lambda(K) + \lambda(G\setminus K)$$
		$$\lambda(G\setminus K) = \lambda(G) -\lambda(K) < \epsilon$$
		
		
		$\Leftarrow$:
		
		$$\lambda^*(A) \leq \lambda(G) = \lambda(K) + \lambda(G\setminus K) <\lambda(K)+\epsilon \leq \lambda_*(A) +\epsilon $$
		Thus $\lambda^*(A)  = \lambda_*(A) $ and $A \in \mathcal{L}_0$.
	\end{proof}
\begin{coll}
	If $A,B\in \mathcal{L}_0$, then $A\cup B, A\cap B, A\setminus B  \in \mathcal{L}_0$
	\begin{proof}
		First, show that $A\setminus B \in \mathcal{L}_0$.
		Take $K_1\subseteq A\subseteq G_1$ and $K_2 \subseteq A\subseteq G_2$.
		$$\lambda(G_1\setminus K_1) < \frac{\epsilon}{2}$$
		$$\lambda(G_2\setminus K_2) < \frac{\epsilon}{2}$$
		
		Denote $K = K_1\setminus G_2$ and $G = G_1 \setminus K_2$.
		$$K\subseteq A\setminus B \setminus G$$
		$$G\setminus K = (G_1\setminus K_1) \cup (G_2\setminus K_2)$$
		$$\lambda(G\setminus K)  \leq \lambda(G_1\setminus K_1) + \lambda(G_2\setminus K_2) < \epsilon$$
		
		Now
		$$A\cup B = (A\setminus B) \cup B \in \mathcal{L}_0$$
		$$A\cap B = A\setminus (A\setminus B)   \in \mathcal{L}_0$$
	\end{proof}
\end{coll}
\end{theorem}

\begin{theorem}
	Let $\left\{ A_k \right\}$ set in $\mathcal{L}_0$ and $A \bigcup_{k=1}^\infty A_k$ such that $\lambda^*(A) < \infty$ then $A\in \mathcal{L}_0$ and 
	$$\lambda(A) \leq \sum_{k=1}^\infty \lambda(A_k)$$
		In addition, if $A_i\cap A_j = \emptyset$,
		$$\lambda(A) = \sum_{k=1}^\infty \lambda(A_k)$$
	\begin{proof}
		Suppose $\left\{ A_k \right\}$ are disjoint.
		$$\lambda^*(A) \leq \sum_{k=1}^\infty \lambda^*(A_k) = \sum_{k=1}^\infty \lambda_*(A_k) \leq \lambda_*(A)$$
		
		Thus $A\in \mathcal{L}_0$ and
		$$\lambda(A) = \lambda^*(A) =  \sum_{k=1}^\infty \lambda^*(A_k) =  \sum_{k=1}^\infty \lambda(A_k)$$
		
		Now generally, define $$B_1 = A_1 \in \mathcal{L}_0$$
		$$B_2 = A_2 \setminus A_1$$
		and so on:
		$$B_k = A_k \setminus \bigcup_{i=1}^{k-1} A_i \in \mathcal{L}_0 $$
		
		Now $\left\{ B_k \right\} $ are disjoint and $A = \bigcup_{k=1}^\infty A_k = \bigcup_{k=1}^\infty B_k \in \mathcal{L}_0 $. Thus
		$$\lambda(A) = \lambda\qty(\bigcup_{k=1}^\infty A_k)=\lambda\qty(\bigcup_{k=1}^\infty B_k) = \sum_{k=1}^\infty \lambda(B_k) \leq \sum_{k=1}^\infty \lambda(A_k)  $$
	\end{proof}
\end{theorem}

\paragraph{Note}Any ball $B(0,R)$ is in $\mathcal{L_0}$, since it is inside special box large enough.
\begin{definition}
	Let $A\subseteq \mathbb{R}^n$, we say $A$ is Lebesgue measurable if $\forall M\in \mathcal{L}_0 \quad A\cap M \in \mathcal{L}_0$. It's measure equals
	$$\lambda(A) = \sup \left\{ \lambda(A\cap M), M \in \mathcal{L}_0 \right\}$$
	
	Denote a set of all such sets as $\mathcal{L}$.
\end{definition}
\begin{prop}
	If $\lambda^*(A) < \infty$, $A\in \mathcal{L} \iff A \in \mathcal{L}_0$. For those sets $\lambda$ definitions are equivalent.
	\begin{proof}
		If $A \in \mathcal{L}_0$ in, then $\forall M\in \mathcal{L}_0 \quad A\cap M \in \mathcal{L}_0$, thus $A\in \mathcal{L}$.
		
		Now, if $A\in \mathcal{L}$ and $\lambda^*(A) < \infty$. For all $N \in \mathbb{N}$,
		$$A\cap B(0,N) \in \mathcal{L}_0$$
		
		However
		$$A = \bigcup_{N=1}^\infty \qty[A\cap B(0,N)]$$
		And $\lambda^*(A) < \infty$, thus  $A \in \mathcal{L}_0$.
		
		Denote
		$$\tilde{\lambda}(A) = \sup \left\{ \lambda(A\cap M), M \in \mathcal{L}_0 \right\}$$
		Obviously, $\tilde{\lambda}(A)\geq\lambda(A) $ (take $M=A$). On the other side,
		$$\forall M\in \mathcal{L}_0 \quad \lambda(A\cap M) \leq \lambda(A)$$
		thus $\tilde{\lambda}(A)=\lambda(A) $
		
	\end{proof}
\end{prop}

\begin{prop}
	$$\emptyset \in \mathcal{L}$$
	\begin{proof}
		$$\emptyset \in \mathcal{L}_0 \Rightarrow \emptyset \in \mathcal{L}$$
	\end{proof}
\end{prop}
\begin{prop}
$$A \in \mathcal{L} \Rightarrow \mathbb{R}^n \setminus A \in \mathcal{L} $$
\begin{proof}
	Take $M\in \mathcal{L}_0$. 
	$$\qty(\mathbb{R}^n \cap A) \cap M = M \setminus A = M \setminus (A\cap M) \in \mathcal{L}_0$$
\end{proof}
\end{prop}
\begin{prop}
$$\left\{ A_i \right\}_{i=1}^\infty \in \mathcal{L} \Rightarrow A = \bigcup A_i \in \mathcal{L} $$
\begin{proof}
	Take $M\in \mathcal{L}_0$. 
	$$A\cap M = \bigcup_{i=1}^\infty (A_k\cap M) $$
	$$\lambda^*(A\cap M) \leq \lambda(M) $$
	Thus
	$$A\cap M \in \in \mathcal{L}_0$$
\end{proof}
\end{prop}

\begin{prop}
	If $\forall N\in \mathbb{N}$, $A\cap B(0,N) \in \mathcal{L}_0$, then $A\in \mathcal{L}$.
\end{prop}

\begin{definition}
	For some set $X$, set $M$ of its subsets is called $\sigma$-algebra if
	\begin{enumerate}
		\item $\emptyset \in M$
		\item $A\in M \Rightarrow X\setminus A\in M$
		\item $\left\{ A_i \right\}_{i=1}^\infty \in M \Rightarrow A = \bigcup A_i \in M $
	\end{enumerate}
\end{definition}

\paragraph{Examples}
\begin{enumerate}
	\item $2^X$ for any $X$ is $\sigma$-algebra
	\item All subsets of $\mathbb{R}$ that are countable or their complement is countable.
	\item All open sets in $\mathbb{R}$ is not $\sigma$-algebra.
\end{enumerate}
\begin{prop}
	If $M$ is $\sigma$-algebra and  $\left\{ A_k \right\}_{k=1}^\infty \subset M$, then
	$$\bigcap_{k=1}^\infty A_k \in M$$
	\begin{proof}
		$$X \setminus \bigcap_{k=1}^\infty A_k = \bigcup_{k=1}^\infty \qty(X\setminus A_k) \in M$$
	\end{proof}
\end{prop}

\begin{prop}
	All open and closed sets are in $\mathcal{L}$
	\begin{proof}
		Let $A$ some open set. Then $A\cap B(0,N)\in \mathcal{L}_0$. Since $\mathcal{L}$ is closed for complementation, also closed sets are in $\mathcal{L}$.
	\end{proof}
\end{prop}

\begin{prop}
	If $\left\{ A_k \right\}_{k=1}^\infty \subset \mathcal{L}$ then
	$$\lambda\qty(\bigcup_{k=1}^\infty A_k) \leq \sum_{k=1}^\infty \lambda(A_k)$$
	\begin{proof}
		Denote $A=\bigcup_{k=1}^\infty A_k$. For $M\in \mathcal{L}_0$
		$$\lambda(A\cap M) = \lambda\qty(\bigcup_{k=1}^\infty \qty(A_k \cap M)) \leq \sum_{k=1}^\infty \lambda(A_k\cap M) \leq \sum_{k=1}^\infty \lambda(A_k)$$
		Since it right for any $M$, 
		$$\lambda\qty(A) \leq \sum_{k=1}^\infty \lambda(A_k)$$
	\end{proof}
\end{prop}

\begin{prop}
If $\left\{ A_k \right\}_{k=1}^\infty \subset \mathcal{L}$ and $A_i\cap A_j=0$ then
$$\lambda\qty(\bigcup_{k=1}^\infty) = \sum_{k=1}^\infty \lambda(A_k)$$
\begin{proof}
	For some $N \in \mathbb{N}$, choose $\left\{ M_p \in \mathcal{L}_0 \right\}_{p=1}^N$. Define $\mathcal{L}_0 \ni M = \bigcup_{p=1}^N M_p$. 
	$$\lambda(A) \geq \lambda(A\cap M) = \sum_{k=1}^\infty \lambda (A_k \cap M) \geq \sum_{k=1}^N \lambda (A_k \cap M) \geq \sum_{k=1}^N \lambda (A_k \cap M_k)  $$
	Thus
	$$\lambda_A \geq \sup \left\{ \sum_{k=1}^N \lambda (A_k \cap M_k), M_k \in \mathcal{L}_0 \right\} = \sum_{k=1}^N \sup \left\{  \lambda (A_k \cap M_k), M_k \in \mathcal{L}_0 \right\} = \sum_{k=1}^N \lambda(A_k) $$
	Since it's right for any $N$,
	$$\lambda_A \geq \sum_{k=1}^\infty \lambda(A_k) $$
\end{proof}
\end{prop}

\begin{theorem}
	The defined $\lambda$ fulfills properties of measure.
	\begin{enumerate}
	\item $0\leq \lambda(A) \leq \infty$
\item $\lambda(\emptyset) = 0$
\item $\lambda(\qty[a_1,b_1]\cross\qty[a_2,b_2]\cross\dots\cross \qty[a_n,b_n]) = \prod_{i=1}^n (b_i-a_i)$
\item If $A = \bigcupdot_{k=1}^\infty A_k$, then $\lambda(A) = \sum_{i=1}^\infty \lambda(A_k)$.
\item If $C$ is acquired from $A$ by rotation or translation $\lambda(C) = \lambda(A)$.
	\end{enumerate}
\end{theorem}

\begin{definition}[Measure]
	For some set $X$, measure of $X$ is function $\mu$ defined on $\sigma$-algebra $M$ of subsets of $X$ and fulfills
	\begin{enumerate}
		\item $0\leq \mu(A) \leq \infty$
		\item $\mu(\emptyset) = 0$
		\item If $A = \bigcupdot_{k=1}^\infty A_k$, then $\lambda(A) = \sum_{i=1}^\infty \lambda(A_k)$.
	\end{enumerate}
\end{definition}
We denote measure space as $\qty(X,\mu, M)$.
\begin{theorem}
	Let  $\qty(X,\mu, M)$ measure space. 
	\begin{enumerate}
		\item If $\left\{ A_k \right\}_{k=1}^\infty \subset M$ and $\forall k \: A_k \subset A_{k+1}$, then
		$$\mu\qty(\bigcup_{k=1}^\infty A_k) = \lim_{k\to \infty} \mu(A_k)$$
		\item If $\left\{ A_k \right\}_{k=1}^\infty \subset M$ and $\forall k \: A_k \supset A_{k+1}$ and $\mu(A_1) < \infty$, then
		$$\mu\qty(\bigcap_{k=1}^\infty A_k) = \lim_{k\to \infty} \mu(A_k)$$
	\end{enumerate}

\begin{proof}
	$$\bigcup_{k=1}^\infty A_k = A_1 \cup \qty[\bigcup_{k=1}^\infty A_{k+1}\setminus A_k]$$
	Since those sets are disjoint
	$$\mu\qty(\bigcup_{k=1}^\infty A_k ) = \mu(A_1) + \sum_{k=1}^\infty \mu(A_{k+1} \setminus A_k)= \lim_{N\to \infty} \mu(A_1) + \sum_{k=1}^N \mu(A_{k+1} \setminus A_k) = \lim_{N\to \infty} \mu\qty(A_1 \cup \qty[\bigcup_{k=1}^N A_{k+1}\setminus A_k])  = \lim_{N\to \infty} \mu\qty(A_{N+1})$$
\end{proof}
\end{theorem}

\begin{prop}[]
	If $\lambda^*(A)=0 $, $A\in \mathcal{L}$ and for any $B\subset A$, $B\in \mathcal{L}$ and $\lambda(B)=0$.
	\begin{proof}
		$$\lambda_*(A) \leq \lambda^*(A) = 0 \Rightarrow A\in \mathcal{L}_0$$
		Monotonity of upper measure
	\end{proof}
\end{prop}
\begin{theorem}
	$A$ is measurable iff $\forall \epsilon>0$ exist open $G$ and closed $F$ such that $$F\subseteq A\subseteq G$$ and $$\lambda(G\setminus F) \leq \epsilon$$
	\begin{proof}
		$\Leftarrow$:
		
		Suppose exist such $G$ and $K$. For all $k$ choose 
		$G_k$ and $F_k$ such that
		$$\lambda(G_k\setminus F_k) < \frac{1}{k}$$
		Denote 
		$$B = \bigcup_{k=1}^\infty F_k$$
		$$\lambda^* (A\setminus B) = 0$$
		and
		$$A\setminus B \subseteq G_k \setminus B \subseteq G_k \setminus F_k$$
		Thus
		$$\lambda^*(A\setminus B) \leq \lambda(G_k \setminus F_k) <\frac{1}{k}$$
		Thus $\lambda^*(A\setminus B) = 0$ and $A\setminus B \in \mathcal{L}$.
		
		However $B \in \mathcal{L}$ and $A = B\cup (A\setminus B)$, thus $A\in \mathcal{L}$.
		
		
		$\Rightarrow$:
		
		Suppose $A\in \mathcal{L}$. Denote $E_k = B(0,k) \setminus B(0,k-1)$. This is partition of $\mathbb{R}^n$. $E_k \in \mathcal{L}_0$ and so is $A\cap E_k \in \mathcal{L}$. 	
		Thus for all $k$ there is
		$$K_k \subseteq A\cap E_k \subseteq G_k$$
		such that $\lambda(G_k\setminus K_k) < \frac{\epsilon}{2^k}$.
		Denote 
		$$F = \bigcup_{k=1}^\infty K_k$$
		$$G = \bigcup_{k=1}^\infty G_k$$
		
		$$\lambda(G\setminus F)= \lambda\qty(\bigcup_{k=1}^\infty (G_k\setminus F)) \leq \lambda\qty(\bigcup_{k=1}^\infty (G_k\setminus K_k)) \leq \sum_{k=1}^\infty\lambda\qty( G_k\setminus K_k) < \epsilon$$
		
		Now, $F$ is closed. Let $F \ni x_k \to x$. The sequence converges and thus bounded, and thus exists $N$ such that $\left\{ x_k \right\} \cup \left\{ x \right\} \in B(0,N)$.
		
		Thus $\left\{ x_k \right\} \subseteq \qty(\bigcup_{i=1}^N E_i) \cap F$ and $\left\{ x_k \right\} \subseteq \bigcup_{i=1}^N K_i$ and thus  $\left\{ x_k \right\} \cup \left\{ x \right\} \in F$.
		
		
		
	\end{proof}
\end{theorem}
\begin{prop}
	If $A$ is measurable then $\lambda(A)=\lambda^*(A)=\lambda_*(A)$.
	\begin{proof}
		If $\lambda^*(A)<\infty$ we've already seen this.
		
		Suppose $\lambda^(A)=\infty$. Thus $\inf \left\{  \lambda(G) : A\subseteq G \right\} =\infty$. By previous theorem exists closed $F$ such that $F\subseteq A\subseteq G$ and $\lambda(G\setminus A) \leq 1$. 
		
		
		$$\infty = \lambda(G) = \lambda(G\setminus A) + \lambda(A) \leq \aa,bda(G\setminus F) + \lambda(A) \leq 1+\lambda(A)$$
		Thus, $\lambda(A)=\infty$.
		
		Now, take a look at $\left\{ A\cap B(0,N)   \right\}_{N}$. 
		$$\infty = \lambda(A) = \lambda\qty(\bigcup_N (A\cap B(0,N))) = \lim_{N\to \infty} \lambda\qty(A\cap B(0,N)) $$
		$$\infty \leftarrow \lambda(A\cap B(0,N)) = \lambda_*(A\cap B(0,N)) \leq \lambda_*(A)$$
	\end{proof}
\end{prop}

\paragraph{Reminder}
We've built $E\subseteq \qty[0,\frac{1}{2}]$ such that $q+E : q\in \mathbb{Q}$ is disjoint. And 
$$\forall k \in \mathbb{N} \: \frac{1}{k} + E \subseteq[0,1]$$
$$\bigcup_{q\in \mathbb{Q} } q+E=\mathbb{R}$$

\begin{prop}
	$E$ is not measurable
	\begin{proof}
		$$\bigcup_{k=2 }^\infty \qty(\frac{1}{k}+E)\subseteq [0,1]$$
		$$1 = \lambda_*\qty(\bigcup_{k=2 }^\infty \qty(\frac{1}{k}+E)) \geq \sum_{k=2}^\infty \lambda_*\qty(\frac{1}{k}+E) $$
		i.e., $\lambda_*(E)= 0$.
		On the other hand
		
		$$\infty = \lambda^*(\mathbb{R}) = \lambda^*\qty(\bigcup_{q\in \mathbb{Q} } q+E) \leq \sum_q \lambda^*(q+E) = \sum_q \lambda^*(E)$$
		
		Thus $\lambda^*(E) > 0$, i.e., $E$ is not measurable.
	\end{proof}
\end{prop}

\begin{prop}
	For any measurable $A\subseteq \mathbb{R}$ such that $\lambda(A)>0$, exists non-measurable $B\subseteq A$.
	
	\begin{proof}
		We've seen that
		$$\bigcup_{q\in \mathbb{Q} } q+E=\mathbb{R}$$
		thus 
		$$A = \bigcup_{q\in \mathbb{Q} } A\cap (q+E)$$
		
		$$0 \leq \lambda^*(A) = \lambda^*\qty(\bigcup_{q\in \mathbb{Q} } A\cap (q+E)) \leq \sum_q \lambda^*(A \cap (q+E))$$
		
		Thus  exists $q_0$ such that $0 < \lambda^*\qty(A\cap (q+E))$, denote
		$$B = A \cap (q_0+E)$$
		$$\lambda_*(B) \leq \lambda_*(q_0+E) = \lambda_*(E) = 0$$
		i.e. $B\notin \mathcal{L}$.
		
	\end{proof}
\end{prop}

\begin{prop}
	$B$ measurable, $A\subseteq B$, then
	$$\lambda^*(A) + \lambda_*(B\setminus A) = \lambda(B)$$
	\begin{proof}
		$$\lambda^*(A) = \inf\left\{ \lambda(G) : A\cap G   \right\} $$
		$$\lambda(G) + \lambda_*(B\setminus A) \geq \lambda(G) + \lambda_*(B\setminus B) = \lambda(G) + \lambda(B\setminus G) \geq \lambda(B)$$
		
		
		On the other hand, for any $K\subseteq B\setminus A$
		$$\lambda^*(A) + \lambda(K) \leq \lambda(B\setminus K)+ \lambda(K) = \lambda(B)$$
		By taking supremum on $K$, we get
		$$\lambda^*(A) + \lambda(B\setminus A) \leq  \lambda(B)$$
	\end{proof}


\end{prop}

\begin{prop}[Carath\'{e}odory's condition]
	$$A\subseteq \mathbb{R}^n \text{ measurable} \iff \forall E\subseteq \mathbb{R}^n \quad \lambda^*(E) = \lambda^*(E\cap A)+\lambda^*(E\setminus A)$$
	
	\begin{proof}
		$\Rightarrow$:
		
		Let $A$ measurable set. Choose general $E$. For open $G\supset E$,
		$$\lambda(G) = \lambda(G\cap A) + \lambda(G\setminus A) \geq \lambda^*(E\cap A)+ \lambda^*(E\setminus A)$$
		
		Since it's right for any $G$, by taking infimum:
		$$\lambda^*(E) \geq \lambda^*(E\cap A) + \lambda^*(E\setminus A) $$
		And by subadditivity
		$$\lambda^*(E) \leq \lambda^*(E\cap A) + \lambda^*(E\setminus A) $$
		i.e.,
		$$\lambda^*(E) = \lambda^*(E\cap A) + \lambda^*(E\setminus A) $$
		
		$\Leftarrow$:
		
		Suppose the condition is right for $A$. Let $M\in \mathcal{L}_0$, then
		$$\lambda(M) = \lambda^*(M\cap A) + \lambda^*(M\setminus A)$$
		From previous proposition
		$$\lambda(M) = \lambda^*(M\cap A) + \lambda_*(M\setminus A)$$
		
		Thus
		$$\lambda_*(M\setminus A) = \lambda^*(M\setminus A)$$
		and thus $M\setminus A \in \mathcal{L}_0$, i.e. $A\in \mathcal{L}$.
	\end{proof}
\end{prop}
\begin{lemma}
	Let $A\subseteq \mathbb{R}$ with positive measure, and let $\epsilon>0$ then there exists an interval $J\subseteq \mathbb{R}$ $\frac{\lambda(A\cap J)}{\lambda(J)} = 1-\epsilon$
	\begin{proof}
		Denote $C=\lambda(A)>0$.
		$$\lambda(A) = \lambda^*(A) = C$$
		Thus exists open $G\supseteq A$ such that $\lambda(G) < \qty(1+ \frac{\epsilon}{2})C$.
		
		Since $G$ is open, it is disjoint union of open intervals:
		$$G = \bigcup_{i=1}^\infty J_i$$
		$$\qty(1+ \frac{\epsilon}{2})C > \lambda(G) = \sum \lambda(J_i)$$
		
		Assume that $\forall i \: \lambda(A\cap J) leq (1-\epsilon) \lambda(J)$. Then
		$$C = \lambda(A) = \lambda\qty(A \cap \qty(\bigcup_{i=1}^\infty J_i)) = \sum_{i=1}^\infty \lambda(A\cap J_i) \leq (1-\epsilon) \sum \lambda_J = (1-\epsilon)\lambda(G) = (1-\epsilon)\qty(1+\frac{\epsilon}{2})C < C$$
	\end{proof}
\end{lemma}

\begin{theorem}
	Let $A\subset \mathbb{R}$ measurable set with positive measure.
	$A-A = \left\{ x-y|x,y\in A \right\}$.
	
	\begin{proof}
		If $A$ has non-empty interior, the theorem is obvious. since there exists $a\in A$, $(a-\delta, a+\delta) \subset A$ and thus $(-\delta, \delta) \subset A-A$.
		
		$$t\in A-A \iff A+t\cap A \neq \emptyset$$
		Let $J=(a,b)$ from previous lemma with $\epsilon =\frac{1}{3}$. Assume $t\notin A-A$, i.e. $A\cap(A+t) =\emptyset$. And thus
		$$(A\cap J) \cap \qty[(A+t) \cap (J+t)] = \emptyset$$
		$$\lambda(A\cap J) \geq \frac{2}{3} \lambda(J)$$
		$$\frac{2}{3}\lambda(J) + \frac{2}{3}\lambda(J) \leq \lambda(A\cap J) + \lambda\qty\big((A+t) \cap (J+t)) =\lambda\qty((A\cap J) \cup \qty\big[(A+t) \cap (J+t)]) \leq \lambda(J \cup (J+t))$$
		
		Now, if $t\geq 0$, $J \cup (J+t) \subseteq (a,b+t)$, and if $t<0$, $J \cup (J+t) \subseteq (a+t,b)$. Anyway
		$$\frac{4}{3}\lambda(J) \leq \lambda(J \cup (J+t)) \leq \lambda(J) + \abs{t} $$
		i.e.,
		$$\abs{t} \geq \frac{1}{3}\lambda(J)$$
		Thus $\forall\: 0<t < \frac{1}{3}\lambda(J)$, $(-t,t)\subseteq A-A$.
	\end{proof}
\end{theorem}

Let $a$ set of subsets in $\mathbb{R}^n$. Exists $\sigma$-algebra that is superset of $a$, and also
$$\bigcup \left\{ m: \quad a\subset m  |; \sigma\text{-algebra} \right\}$$
is $\sigma$-algebra and is called $\sigma$-algebra  generated by $a$.

Denote $\mathcal{B}$  $\sigma$-algebra  generated by all open sets in $\mathbb{R}^n$. $\mathcal{B}$ is Borel $\sigma$-algebra. Since all open sets are in $\mathcal{L}$, $\mathcal{B} \subseteq \mathcal{L}$.

\begin{theorem}
	Let measurable $A\subseteq \mathbb{R}^n$, we can write $A=E\cup N$, such that
	\begin{enumerate}
		\item $E\cap N=0$
		\item $E\in \mathcal{B}$
		\item $\lambda(N)=0$ 
	\end{enumerate}
\begin{proof}
	For all $k\in \mathbb{N}$, find
	$$F_k\subseteq A\subseteq G_k$$
	$G_k$ open and $F_k$ closed, and
	$$\lambda(G_k\setminus F_k) \leq \frac{1}{k}$$
	Denote $E = \bigcup_{k=1}^\infty F_k \in \mathcal{E}$. $N=A\setminus E \in \mathcal{L}$.
	
	$$\lambda(N) = \lambda(A\setminus E) \leq \lambda(G_k\setminus F_k) <\frac{1}{k}$$
	i.e., $\lambda(N)=0$.
\end{proof}

\end{theorem}

\paragraph{Reminder}
$f: E\to \mathbb{R}^n$ is continuous iff $\forall \quad G\subseteq \mathbb{R}^n$, $f^{-1}(G) $ is open in $E$.

\begin{theorem}
	Let $f: E \to \mathbb{R}^n$ be continuous for Borel set $E\subseteq \mathbb{R}^n$. Then $f^{-1}(\mathcal{B}) \subseteq \mathcal{B}$.
	\begin{proof}
		Let
		$$m = \left\{ A\subseteq \mathbb{R}^n : f^{-1}(A) \in \mathcal{B} \right\}$$
		We need to show that $\mathcal{B} \subseteq m$, i.e., that $m$ is $\sigma$-algebra containing all open sets.
		
		$\emptyset \in m$, since $\emptyset = f^{-1}\qty(\emptyset)$.
		
		If $\left\{ A_k \right\} \subseteq m$, then $f^{-1}(A_k) \in \mathcal{B}$ and
		$$f^{-1}\qty(\bigcup_k A_k) = \bigcup_k f^{-1}(A_k) \in \mathcal{B}$$
		
		If $A\in m$, then
		$$f^{-1}(\mathbb{R}^n \setminus A) = E\setminus f^{-1}(A) \in \mathcal{B} $$
		
		
		Now lets show that all open sets are in $m$. If $G$ is open,
		$$f^{-1}(G) = E\cap U_G \in \mathcal{B}$$
	\end{proof}
\end{theorem}

\begin{theorem}
	There exists measurable set in $\mathbb{R}$ which is not Borel.
	\begin{proof}
		Define $f: [0,1]:\mathbb{R}$. Let $x$ in ternary basis $0.a_1a_2\dots$. Then
		$$f(x) = \frac{1}{2^N}+\sum_{1}^{N-1} \frac{1}{2^n} \frac{a_n}{2} $$
		where $N$ is first index such that $a_N=1$.
		
		Note that $f$ is constant on $I\subset [0,1]$ such that $I \not\subset C$ (Cantor set).
		
		$f$ is monotonous and onto, and thus continuous.
		
		Define also $g(x) = x+f(x)$, which is one-to-one and onto, thus it is homeomorphism.
	\end{proof}
\end{theorem}

Denote $\mathcal{C}$ set of intervals in $[0,1]\setminus C$. Any interval $J\in \mathcal{C}$ exists $r$ such that
$$g(x) = x+r$$
($f$ is constant on $J$). That means $\lambda(g(J))=\lambda(J)$.

We see that
$$\lambda(G) - \lambda\qty([0,2] \setminus \bigcup_{J\in \mathcal{C}} g(J)) = 2 - \sum_{J\in \mathcal{C}} \lambda(g(J)) = 2- \sum_{J\in \mathcal{C}} \lambda(J) = 2-1=1$$

Let $B\subseteq g(C)$ which is not measurable. Denote
$$A = g^{-1}(B)$$
It is obvious that $A\subseteq C$, and since $\lambda(C)=0$, $\lambda(A)=0$.

If $A$ was Borel, then, since
$B=g(A)$ and $g$ is homeomorphism, we get that $B$ is Borel. However, this is impossible, since $B$ is non-measurable.
\section{Measurable functions and integrals}
We want to define integral as the sum of possible values of function times the size of set for which function gets this values:
$$\int f \sim \sum f(t_i \in A_i) \times \lambda(A_i)$$
where
$$A_i = \left\{ x: f(x) \in [a,a+\epsilon] \right\}$$

Let $X$ space with $\sigma$-algebra $M$. We work with functions
$$f: X \to [-\infty, \infty]$$

\begin{definition}
	We say $f$ is $M$-measurable if for all $-\infty\leq t\leq \infty$ 
	$$f^{-1} \qty(-\infty,t) \in M$$
\end{definition}

\begin{prop}
	The following conditions are equivalent:
	\begin{enumerate}
		\item $f$ is $M$-measurable: $$\forall  \: -\infty < t \leq \infty \quad f^{-1}([-\infty, t]) \in M$$
		\item $$\forall  \: -\infty < t \leq \infty \quad f^{-1}([-\infty, t)) \in M$$
		\item $$\forall  \: -\infty \leq t \leq \infty \quad f^{-1}([t, \infty]) \in M$$
		\item $$\forall  \: -\infty \leq t < \infty \quad f^{-1}((t, \infty]) \in M$$
		\item $f^{-1}(\infty) \in M$, $f^{-1}(-\infty) \in M$, and $\forall E\in \mathcal{B}(\mathbb{R})$ $f^{-1}(E) \in M$
		\item $f^{-1}(\infty) \in M$, $f^{-1}(-\infty) \in M$, and $\forall a,b \in \mathbb{R}$ $f^{-1}([a,b]) \in M$
	\end{enumerate}

\begin{proof}
	$1\Rightarrow 2$:
	
	$$f^{-1}([-\infty, t)) = \bigcup_{\mathbb{Q} \ni r <t} f^{-1} ([-\infty, r])$$
	thus $f^{-1}([-\infty, t)) \in M$.
	
	$2\Rightarrow 3$:
	
	If $t=-\infty$, $f^{-1}([-\infty, \infty]) = X \in M$. Otherwise
	$$f^{-1}([t, \infty]) = X \setminus \left(f^{-1}([-\infty, t))\right) $$
	thus $f^{-1}([t, \infty]) \in M$.
	
	$3\Rightarrow 4$: just like $1\Rightarrow 2$
	
	
	$4\Rightarrow 1$: just like $2\Rightarrow 3$
	
	$1-4 \Rightarrow 5$:
	
	Taking $t=\pm \infty$, we get $f^{-1}(\infty)$ and $f^{-1}(-\infty)$. 
	
	Let 
	$$S = \left\{ E\subset \mathbb{R} | f^{-1}(E)\in M \right\}$$
	$S$ is $\sigma$-algebra.
	
	$$f^{-1} \qty\bigg((a,b)) = f^{-1}((a,\infty]) \cap f^{-1} ([-\infty, b)) \in M$$
	Thus open intervals are in $\mathbb{R}$, and thus open sets and thus $\mathcal{B}\subset S$.
	
	
	$5 \Rightarrow 6$: Obvious, since $5$ is stronger
	
	
	$6 \Rightarrow 1$: Lest as an exercise
\end{proof}
\end{prop}
\begin{coll}
	If $f: E\to [-\infty, \infty]$ and $E\in M$, then the definition is conserved.
\end{coll}
\begin{coll}
	$\chi_A(x) = \begin{cases}
	1&x\in A\\0&x\notin A
	\end{cases}$ is measurable iff $A \in M$.
\begin{proof}
		
	$A=\chi_A^{-1}(\left\{ 1 \right\})$, thus one direction is obvious.
	
	Else,
	$$\chi_A^{-1}(A) = \begin{cases}
	X&0,1\in E\\
	A& 1\in E, 0\notin E\\
	X\setminus A & 1\notin E, 0 \in E\\
	\emptyset & 0,1\notin E
	\end{cases}$$
\end{proof}
\end{coll}

\begin{coll}
	$f: E\to \mathbb{R}$, for Borel set $E\subset \mathbb{R}$. If $f$ is continuous then $f$ Borel-measurable and Lebesgue-measurable.
\end{coll}

\begin{theorem}
	Let $f: X \to \mathbb{R}$ M-measurable functions. 
	
		If $\phi: B\to \mathbb{R}$ for Borel set $B\subseteq \mathbb{R}$ and $f(x) \subseteq B$ and $\phi$ Borel-measurable, then $\phi \circ f$ are M-measurable.
		
		\begin{proof}
			We need to show
			$$f^{-1}(\phi^{-1}(E)) = (\phi\circ f)^{-1}(E) \in M$$
			Now, $\phi^{1}(E) \in \mathcal{B}$, since $\phi$ is Borel-measurable. Then $f^{-1}(\phi^{-1}(E)) \in M$.
		\end{proof}
		\begin{coll}
			If $f$ is non-zero, $\frac{1}{f}$ is measurable.
		\end{coll}
	\begin{coll}
	If $0<p<\infty$, $\abs{f}^p$ is measurable.
\end{coll}
\end{theorem}

\begin{prop}
	If $f$ is weaker (for example, Lebesgue-measurable), the theorem is not true, even if $\phi$ is homeomorphism. For example, we've seen $g$ and non measurable $g(A)$ for measurable $A$. Then
	$$\chi_A \circ \phi = \chi_{g(A)}$$
	which is non-measurable.
\end{prop}

\begin{theorem}
	Let $f,g : X \to \mathbb{R}$ $M$-measurable functions. Then
	$f+g$, $cf$, $f\cdot g$ are $M$-measurable.
	\begin{proof}
		$$(f+g)^{-1}(-\infty,t) \bigcup_{r\in \mathbb{Q}} \qty[f^{-1}(-\infty, r) \cap g^{-1} (-\infty, t-r)]$$
		
		That means measurable functions are vector space.
		
		$$f\cdot g = \frac{1}{4} (f+g)^2 - \frac{1}{4} (f-g)^2$$
	\end{proof}
\end{theorem}

\begin{theorem}
	Let $\left\{ f_k\right\}_{k=1}^\infty: X \to [-\infty, \infty]$ sequence of $M$-measurable functions. Then also 
	$\liminf f_k$, $\limsup f_k$, $\sup f_k$, $\inf f_k$ and so is $\lim f_k$ if exists.
	\begin{proof}
		$$\qty(\sup f_k)^{-1} ([-\infty, t]) = \left\{ x: \sup f_k(x) \leq t \right\} = \bigcap \left\{ x: f_k(x) \leq t \right\} = \bigcap f^{-1}_k ([-\infty, t]) \in M$$
		$$\limsup f_k(x) = \inf\limits_n \qty(\sup\limits_n{j\leq n} f_j(x)) $$
	\end{proof}
\end{theorem}

\begin{definition}[Simple function]
	$f: X \to [-\infty, \infty]$ is called simple function if it acquires only finite number of values.
	
	If we denote those values as $\left\{ a_i \right\}_{i=1}^n$ and $A_k = \left\{ x: f(x) = a_k \right\}$. Then we can rewrite function as
	$$f(x) = \sum_{k=1}^na_k \chi_{A_k}(x)$$
	
	In fact, all  functions that can be written as
	$$f(x) = \sum_{k=1}^m b_K \chi_{B_k}(x)$$
	is simple. If $\left\{ B_k\right\}$ ar disjoint and $b_k$ are not equal, this is called canonical representation.
\end{definition}

\begin{prop}
	$f$ is measurable iff $\forall k \: A_k \in M$
	\begin{proof}
		$\chi_A$ measurable $\Rightarrow$ $f$ is measurable.
		
		$A_k = \left\{ x: f(x) = a_k \right\}$ is measurable.
	\end{proof}
\end{prop}

\begin{theorem}
	$f: X\to [-\infty, \infty]$. $f$ $m$-measurable if there is sequence $\left\{s_k \right\}$ of measurable simple functions such that $\forall x \: s_k(x) \to f(x)$. We can choose $s_k$ such that $\abs{s_{k-1}} \leq \abs{s_k}$.
	
	\begin{proof}
		$\Leftarrow$ obvious.
		
		$\Rightarrow$:
		
		Suppose $f\geq 0$. Define
		$$s_k(x) = \begin{cases}
		k & f(x)\geq k\\
		\frac{i-1}{2^k} & \frac{i-1}{2^k} \leq f(x) < \frac{i}{2^k}
		\end{cases}$$
		
		We can rewrite as
		$$s_k(x) = k\chi_{A_kf^{-1}(k,\infty)} + \sum_{i=1}^{k\cdot 2^k} \frac{i-1}{2^k} \chi_{f^{-1}\qty[\frac{i-1}{2^k},\frac{i}{2^k}]}$$
		which is canonical form, and we conclude $s_k$ are measurable.
		
		Obviously, $s_k\leq s_{k+1}$.
		
		If $f(x) = \infty$, $s_k=k \to \infty=f(x) $.
		
		Else, $\exists k_0> f(x)$, and then
		$$s_k(x) \leq f(x) \leq s_k(x) + \frac{1}{2^k}$$
		i.e., $s_k(x) \to f(x)$.
		
		
		In general case we define $f_+ = \max \left\{f(x),0 \right\}$ and $f_- = \max \left\{-f(x),0 \right\}$. Note that $f=f_+-f_-$ and $f_-\cdot f_+ = 0$. Both $f_-, f_+$ are measurable. For $f_\pm$ exist sequences $\left\{ s'_k \right\}$, $\left\{ s''_k \right\}$, we can define $s_k=s'_k-s''_k$. 
		
		For any $x$ either $s'_k(x)$ or $s''_k(x)$ is $0$, thus in any point $s_k=s'_k$ or $s_k=-s''_k$, i.e., $\abs{s_{k-1}} \leq \abs{s_k}$. 
	\end{proof}
\end{theorem}

\begin{definition}
	If some property is fulfilled for all $x$ except, maybe, some set $A$ which is subset of set of measure $0$, we say that property is fulfilled almost everywhere (a.e.). In probability we say the property fulfilled almost surely (a.s.). 
\end{definition}
\begin{theorem}
	Let $f: \mathbb{R}^n \to [-\infty, \infty]$ be Lebesgue-measurable function. Then  $\exists g(x)$, Borel-measurable function, such that $\lambda\qty(\left\{ x: f(x) \neq g(x) \right\}) = 0$, i.e. $f(x)=g(x)$ a.e.
	
	\begin{proof}
		Suppose $f\geq 0$. Let $\left\{ s_k\right\}$ as in previous theorem and thus $f=\sup s_k$. 
		$$s_k = \sum_{j=1}^m a_j \chi_{A_j}$$
		Since $A_j \in \mathcal{L}$ we can rewrite it as $A_j = E_j \cup N_j$.
		
		Define $$h_k = \sum_{j=1}^m a_j \chi_{E_j} \leq s_k$$
		
		Since $h_k = s_k$ except for $\bigcup N_j$, which is of measure $0$, $h_k = s_k$ a.e.
		
		Denote $N = \bigcup_{k=1}^\infty \bigcup_{j=1}^{m_k} N_j$, obviously $\lambda(N) = 0$. Also define $g=\sup\limits_k h_k$.
		
		$g(x)=f(x)$ if $x\notin N$, i.e., a.e. and $g(x)$ is Borel-measurable as supremum of Borel-measurable functions.
		
		For general $f$, we do same with $f_\pm$ and acquire $g_\pm$.
	\end{proof}
\end{theorem}
\begin{lemma}
	If $f$ is Lebesgue measurable, then if $g: \mathbb{R}^n \to [-\infty,\infty]$ fulfilling
	$$\lambda^*\left\{ x : f(x) \neq g(x) \right\} = 0$$ 
	then $g$ is measurable.
	\begin{proof}
		Let $-\infty \leq t \leq \infty$, we need to show that $B=g^{-1}\qty(\qty[-\infty, t])$ is Lebesgue-measurable.
		
		We now that $A=f^{-1}[-\infty, t]$ is Lebesgue-measurable.
		$$B \setminus A \subseteq \left\{x: f(x) \neq g(x)  \right\}$$
		Thus $B\setminus A $ is measurable with measure $0$.
		$$B = (A\cup B) \setminus (A\setminus B) \in \mathcal{L}$$
	\end{proof}
\end{lemma}
\begin{theorem}[Tietze extension theorem] \label{tietze}
	Let $Y \subseteq \mathbb{R}^n$ and $f: Y\to \mathbb{R}$ continuous and bounded ($\abs{f}\leq M$). Exists continuous function $F: \mathbb{R}^n\to \mathbb{R}$ such that $F=f$ on $Y$ and $\abs{F}\leq M$.
\end{theorem}
\begin{theorem}[Lusin's theorem]
	Let $f: \mathbb{R}^n \to \mathbb{R}$ which vanishes outside of measurable set $A$. The for all $\epsilon >0$ exists closed set $E\subseteq A$ and continuous function $g: \mathbb{R}^n \to \mathbb{R}$ such that $f=g$ on $E$ and $\lambda(A\setminus E) < \epsilon$.
	\begin{proof}
		Let $f$ be a simple function in canonical form:
		$$f(x) = \sum_{j=1}^m a_j \chi_{A_j}(x)$$
		and $A = \bigcup_{j=1}^m A_i$. $A_i$ are measurable, thus exists closed set $F_i \subseteq A_i \subseteq A$ such that $\lambda(A_i\setminus F_i) < \frac{\epsilon}{m}$.
		
		Denote $E= \bigcup_{i=1}^m F_i$. 
		$$\lambda(A\setminus E) =\sum_i \lambda(A_i\setminus F_i)  < \leq \epsilon$$
		Define $f_0$ on $E$ such that $\eval{f_0}_{F_i}=a_i$. $f$ is continuous and thus by \ref{tietze} exists $g$ as required.
		
		
		Now, let $f$ be measurable and bounded. Let $\epsilon>0$. We know there exists $\left\{ s_k\right\}_{k=1}^\infty$ such that $s_k \to f$ uniformly. 
		
		For all $k$ exists continuous $g_k$ and $L_k$ such that $\lambda(A\setminus L_k) <\frac{\epsilon}{2^k}$ and $g_k=s_k$ on $L_k$. Denote $E = \bigcap L_k$.
		$$\lambda(E\setminus E) = \lambda\qty(A\setminus \bigcap L_k) = \lambda\qty(\bigcup \qty(A\setminus  L_k)) \leq \sum \lambda(A\setminus L_k) < \epsilon$$
		
		On $E$, $g_k$ converges uniformly to $f$, thus $f$ is continuous of $E$ and from \ref{tietze} we get what we wanted.
		
		Let $f$ measurable function which vanishes outside of measurable set $A$ such that $\lambda(A) < \infty$.
		$$\bigcap_N \underbrace{\left\{  x\in A : \abs{f(x)} \geq N \right\}}_{A_N} = \emptyset$$
		$A_N\subset A_{N+1}$, measurable and $\lambda(A) < \infty$, then $0=\lambda(\bigcap_N A_N) = \lim_{N\to \infty} \lambda(A_N)$.
		
		Thus exists $N_0$ such that $\lambda(A_{N_0}) <\frac{\epsilon}{2}$. Denote
		$$G = \left\{ x\in A : \abs{f(x)} < N_0 \right\}$$
		$$A_{N_0} = \left\{ x\in A : \abs{f(x)} \geq N_0 \right\}$$
		Then $\lambda(A\setminus G) =\lambda( A_{N_0} ) < \frac{\epsilon}{2}$.
		
		Then $\chi_G f: G\to \mathbb{R}$ is bounded and measurable, i.e., exists closed $E\subseteq G$ such that $\chi_G f$ is continuous on $E$ and $\lambda(G\setminus E) <\frac{\epsilon}{2}$. Since $\lambda(A\setminus E) < \epsilon$, once again we use \ref{tietze} and get what we wanted. 
		
		Denote $A_k = A\cap (B(0,k) \setminus B(0,k-1)$. Define $f_k =\eval{f}_{A_k}$ then exists closed $E_k\subseteq A_k$ such that $\eval{f_k}_{E_k}$ such that $\lambda(A_k\setminus E_k)  < \frac{\epsilon}{2^k}$. $E=\bigcup E_k$ is closed and $\eval{f}_E$ is continuous and $\lambda(A\setminus E) < \epsilon$.
		\end{proof}
	
	\begin{coll}
		Let $f$ measurable and vanishing outside measurable $A$. Then there exists sequence of continuous functions $\left\{ g_k\right\}$ such that $g_k\to f$ a.e. on $A$.
		\begin{proof}
			For all natural $K$ exists closed $E_k\subseteq A$ and continuous $g_k$ suxh that $g_k=f$ on $E_k$ and $\lambda(A\setminus E_k) < \frac{1}{2^k}$. Denote $E \bigcup_m \qty(\bigcap_{k\geq m} E_k) \subseteq A$.
			
			
			$\forall x \in E$ exists $m$ such that $\forall k\geq m$ $x\in E_k$. That means $\forall k\geq m \: g_k(x) = f(x)$, which means $g_k(x) \to f(x)$ for every such $x$, i.e. for all $x\in E$.
			
			$$A\setminus E = \bigcap_m \bigcup_{k\geq m} \qty(A\setminus E_k)$$
			$$\lambda(A\setminus E) \leq \lambda\qty(\bigcup_{k\geq m} A\setminus E_k) \leq \sum_{k=m}^\infty \lambda(A\setminus E_k) \leq \sum_{k=m}^\infty\frac{1}{2^k} = \frac{1}{2^{m-1}}$$
		\end{proof}
	\end{coll}
\end{theorem}
\begin{definition}[Almost uniform convergence]
Let $\left\{  f_n\right\}$, $f$ be real functions measurable on $A$. We say that $f_n\to f$ almost uniformly, if for all $\epsilon>0$ exists measurable $E$ such that $\lambda(A\setminus E) \leq \epsilon$ and on $E$ $f_n \to f$ uniformly.
\end{definition}
\begin{theorem}[Egorov's theorem]
	Let $A\subseteq \mathbb{R}^n$ be a measurable set with finite measure. Let $\left\{  f_n\right\}$, $f$ real functions measurable on $A$ and $f_n\to f$ a.e. on $A$. Then for all $\epsilon>0$ exists $E$ such that $\lambda(A\setminus E) \leq \epsilon$ and on $E$ $f_n \to f$ uniformly a.e.
	\begin{proof}
		$f_n\to f$, then exists $n$ such that for $j\geq n$ $\abs{f_j-f} < \frac{1}{k}$.
	
	Denote 
	$$E_n^k = \left\{ c: \abs{f_j(x) -f(x)} < \frac{1}{k} \quad \forall k\geq n \right\}  $$
	Then for all $k$ $x\in \bigcup_{n=1}^\infty E_n^k$.
	
	From the assumption we get $\lambda\qty(A\setminus \bigcup_{n=1}^\infty E_n^k)=0$. Note that $E_{n}^k \subseteq E_{n+1}^k$ thus
	$$\forall k \: A\setminus E_n^k \supseteq A\setminus E_{n+1}^k$$
	$$ A\setminus \bigcup_{n=1}^\infty E_n^k = \bigcap_{n=1}^\infty A\setminus E_{n}^k$$
	$$0=\lambda\qty(A\setminus \bigcup_{n=1}^\infty E_n^k) = \lambda\qty(\bigcap_{n=1}^\infty A\setminus E_{n}^k) = \lim_{n\to\infty} \lambda(A\setminus A_n^k)$$
	
	Thus for all $k$ $\lambda(A\setminus A_n^k) \to 0$. For all $\epsilon>0$ exists $n_k$ such that 
	$$n\geq n_k \Rightarrow \lambda\qty(A\setminus E_n^k) < \frac{\epsilon}{2^k}$$
	Denote $E = \bigcup_{k=1}^\infty$ (it is measurable). 
	$$\lambda(A\setminus E) = \lambda\qty(\bigcup_{k} A\setminus E_{n_k}^k) \leq \sum \lambda(A\setminus E_{n_k}^k) < \sum \frac{\epsilon}{2^k} = \epsilon$$
	
	For all $x\in E\subseteq E_{n_k}^k \subseteq E_n^k$, thus for all $k$ exists $n_k$ such that
	$$\abs{f_k(x) - f(x) } \leq \frac{1}{k}$$
	Thus $f_n \to f$ uniformly on $E$.
	\end{proof}
\end{theorem}

\paragraph{Example}
The condition $\lambda(A)<\infty$ is necessary. Take $A=\mathbb{R}$ and 
$$f_n = \chi_{[n,\infty]}$$
$f_n\to 0$ for all $x$.


\section{Lebesgue integral}
\subsection{Stage 1}
Let $s$ simple measurable function we can write
$$s(x) = \sum_{i=1}^m a_i \chi_{A_i} (x)$$
Such that $\left\{  A_i\right\}$ are measurable and disjoint and $\mathbb{R}^n = \bigcup_{i=1}^n A_i$ and $0\leq a_i \in \mathbb{R}$. We define
$$\int s\dd{\lambda} = \sum_{i=1}^m a_i \lambda(A_i)$$
where $0\cdot \infty =0$
\begin{prop}
	$\int s\dd{\lambda}$ is well-defined
	\begin{proof}
	 Directly from \ref{simpints}
	\end{proof}
\end{prop} 
\begin{prop}
$$0\leq \int s\dd{\lambda} \leq \infty$$
\end{prop} 
\begin{prop}
For all $0\leq c \in \mathbb{R}$, $\int cs\dd{\lambda} = c\int s\dd{\lambda}$
\end{prop} 
\begin{prop}
$$\int (s+t) \dd{\lambda} = \int s \dd{\lambda} + \int t \dd{\lambda}$$
\end{prop} 
\begin{prop} \label{simpints}
$s\leq t$ a.e. $\Rightarrow$ $\int s\dd{\lambda} \leq \int t \dd{\lambda}$
\begin{proof}
	
	Denote $N = \left\{ x: s(x)> t(t) \right\}$
	$$s(x) = \sum_{i=1}^{m} a_i \chi_{A_i}(x)$$
	$$t(x) = \sum_{i=1}^{k} b_i \chi_{B_i}(x)$$
	For all $i$
	$$A_i = \bigcup_{j=1}^k (A_i \cap B_j) $$
	Then
	$$\int s \dd{\lambda} = \sum a_i \lambda(A_i) $$
	$$\lambda(A_i) = \lambda\qty(\bigcup_{j=1}^k A_i \cap B_j) = \sum_{j=1}^k \lambda(A_i \cap B_j)$$
	$$\int s\dd{\lambda} = \sum_{i,j} a_i \lambda(A_i \cap B_j)$$
	$$\int t\dd{\lambda} = \sum_{i,j} b_j \lambda(A_i \cap B_j)$$
	
	For all $i,j$, $$a_i\lambda(A_i\cap B_j) \leq b_j \lambda(A_i\cap B_j)$$
	If $\lambda(A_i\cap B_j)=0$ it's obvious. Else $\exists x \in A_i\cap B_j \setminus N$, and for that $x$ $s(x) \leq t(x)$ and thus $a_i\leq b_j$
	
	and thus $\int s\dd{\lambda} \leq \int t \dd{\lambda}$.
\end{proof}
\end{prop} 
\begin{prop}
	$s= t$ a.e. $\Rightarrow$ $\int s\dd{\lambda} = \int t \dd{\lambda}$
	
	\begin{proof}
		Directly from \ref{simpints}
	\end{proof}
\end{prop} 
\begin{prop}
If $\alpha \in \mathbb{R}$ and $s'(x) = s(x+\alpha)$, then $s'$ is simple and 
$$\int s' \dd{\lambda} = \int s \dd{\lambda}$$
\end{prop} 


\subsection{Stage 2}

Let $f: \mathbb{R}^n \to [0,\infty]$ be a measurable function. We define
$$\int f \dd{\lambda } = \sup \left\{ \int s\dd{\lambda} : s<\infty, s\leq f, \text{ simple masurable} \right\}$$
\begin{prop}
	$\int f\dd{\lambda}$ is well-defined
\end{prop} 
\begin{prop}
	$$0\leq \int f\dd{\lambda} \leq \infty$$
\end{prop} 
\begin{prop}
	For all $0\leq c \in \mathbb{R}$, $\int cf\dd{\lambda} = c\int f\dd{\lambda}$
\end{prop} 
\begin{prop}
	If $f\leq g$ a.e.
	$$\int f \dd{\lambda} \leq \int g \dd{\lambda}$$
	\begin{proof}
		Denote $N = \left\{ x ; f(x) > g(x) \right\}$ and $I = \int f\dd{\lambda}$.
		
		For all $\epsilon>0$  exists $s$ such that
		$$\int s \dd{\lambda} \geq I-\epsilon$$
		$$\tilde{s} = \begin{cases}
		s(x) & x\notin N\\
		0 &  x\in N
		\end{cases}$$
		Then $\tilde{s} \leq g$.
		$$I-\epsilon = \int s \dd{\lambda}  = \int \tilde{s} \dd{\lambda} \leq \int g \dd{\lambda}$$
		Thus
		$$\int \tilde{s} \dd{\lambda} \geq I$$
	\end{proof}
\end{prop} 
\begin{prop}
If $f= g$ a.e.
$$\int f \dd{\lambda} = \int g \dd{\lambda}$$
\end{prop} 
\begin{prop}
If $f'(x)= f(x+\alpha)$ a.e.
$$\int f \dd{\lambda} = \int f' \dd{\lambda}$$
\end{prop} 


\begin{theorem}[Monotone convergence theorem]
Let $\left\{ f_n\right\}$ sequence of measurable functions such that $0\leq f_1\leq \dots $
Let $f(x) = \lim_{n\to\infty} f_n(x)$ then $\int f\dd{\lambda} = \lim_{n\to\infty} \int f_n \dd{\lambda}$.
\begin{proof}
	For all $n$
	$$\int f_n \dd{\lambda} \leq \int f\dd{\lambda}$$
	Denote $I = \lim_{n\to\infty} \infty f_n$.
	
	Assume $I<\int f\dd{\lambda}$. Then $\exists c\in \mathbb{R}$ such that
	$$I<c<\int f\dd{\lambda}$$
	From definition of $\int f\dd{\lambda}$ exists simple $t$ such that $0\leq t\leq f$ such that $\int t \dd{\lambda} > c$. Exists $0<q<1$ such that $\int t \dd{\lambda} > \frac{c}{q}$
	
	Define $s=qt$, then $c<\int s \dd{\lambda}$ and $f(x).s(x)$.
	
	Define $$E_k = \left\{ x: f_k(x) \geq s(x) \right\}$$, then
	$$E_1 \subseteq E_2 \subseteq \dots$$
	$$\mathbb{R}^n = \bigcup_{k=1}^\infty E_k$$
	
	For all $k$
	$$f_k \geq f_k \chi_{E_k} \geq s(x) \chi_{E_k}$$
	By writing $s(x) = \sum_{i=1}^m a_i \chi_{A_i}$ we get
	$$ s(x) \chi_{E_k} = \sum a_i \chi_{A_i \cap E_k}$$
	
	Then
	$$\int f_k \dd{\lambda} \geq \int s(x) \chi_{E_k} \dd{\lambda} = \sum a_i \lambda(A_i \cap E_k)$$
	
	But
	$$\lambda(A_i)  = \lim_{k\to \infty} \lambda(A_i\cap E_k)$$
	and
	$$I = \lim_{k\to \infty} \int f_k \dd{x} = \lim_{k\to \infty} \sum_{i=1}^m a_i \lambda(A_i \cap E_k) = \sum_{i=1}^m a_i \lambda(A_i) =\int s \dd{x} > c$$
	
\end{proof}
\begin{coll}
	$$\int (f+g) \dd{\lambda} = \int f\dd{\lambda} + \int g \dd{\lambda}$$
	\begin{proof}
		Find $\left\{s_k\right\} \to f$, $\left\{t_k\right\} \to g$, then
		$$\int f+g \dd{\lambda} = \lim_{k\to \infty} \int s_k+t_k \dd{\lambda}= \lim_{k\to \infty} \int s_k\dd{\lambda}+\lim_{k\to \infty} \int  t_k \dd{\lambda} = \int s\dd{\lambda} + \int t \dd{\lambda}$$
	\end{proof}
\end{coll}

\begin{coll}
	Let $\left\{ f_k \right\}$ sequence of measurable functions, $f_k\geq 0$ on $\mathbb{R}^n$ then
	$$\int \qty[\sum_{k=1}^\infty f_k ] \dd{\lambda} = \sum_{k=1}^\infty \int f_k \dd{\lambda}$$
\end{coll}
\end{theorem}
\begin{theorem}[Fatou's lemma]
	Let $\left\{ f_k \right\}_{k=1}^\infty$ a sequence of nonegative measurable functions.
	
	$$\int \liminf f_k \dd{\lambda} \leq \liminf \int f_k \dd{\lambda}$$
	
	\begin{proof}
		$$\liminf f_k(x)  =\sup\limits_m \inf\limits_{j\geq m} f_j(x)$$
		$$g_m(x) = \inf\limits_{j\geq m} f_j(x) $$
		Since $0\leq g_m \leq g_{m+!}$ and $g_m \leq f_m$,
		$$\liminf f_k(x) = \sup\limits_m g_m(x) = \lim g_m(x)$$
		$$\int \liminf f_k \dd{\lambda} =  \int \lim_{m\to \infty} g_m \dd{\lambda} = \int \lim_{m\to \infty}  g_m \dd{\lambda} \leq \liminf \int f_m \dd{\lambda}$$
	\end{proof}
\end{theorem}

\subsection{Stage 3}
\begin{definition}
	Let $f : \mathbb{R}^n \to [-\infty, \infty]$ measurable. $f$ is called integrable if $\int f^+(x) \dd{\lambda} <\infty$ and $\int f^-(x) \dd{\lambda} <\infty$ and in this case
	$$\int f\dd{\lambda} = \int f^+(x) \dd{\lambda}  - \int f^-(x) \dd{\lambda}$$
	
	Set of all integrable functions is denoted $\mathcal{L}^!(\mathbb{R}^n)$.
\end{definition}

\begin{prop}
	$$\int \abs{f} \dd{\lambda} \leq \infty \iff f\in \mathcal{L}^1$$
	\begin{proof}
		$f\in \mathcal{L}^1$, then
		$$\int \abs{f}  \dd{\lambda}  = \int \qty(f_++f_-) \dd{\lambda} = \int f_+ \dd{\lambda} + \int f_- \dd{\lambda} < \infty$$
		
		$f_+, f_- \leq \abs{f}$, then $\int f_+ \dd{\lambda} ,\int f_- \dd{\lambda} \leq  \int \abs{f} \dd{\lambda} < \infty$.
	\end{proof}
\end{prop}
\begin{prop}
$$\int \abs{f} \dd{\lambda} \geq \abs{\int f \dd{\lambda}}$$
\begin{proof}
	$$\abs{\int f \dd{\lambda} } = \abs{\int f_+ \dd{\lambda} +\int f_- \dd{\lambda}  } \leq \int f_+ \dd{\lambda} + \int f_- \dd{\lambda}  = \int \abs{f}$$
\end{proof}
\end{prop}

\begin{prop}
	If $f\in \mathcal{L}^1$ then
	$$\lambda(f^{-1}(-\infty)) = \lambda(f^{-1}(\infty)) = 0$$
	\begin{proof}
		Let $A = \lambda(f^{-1}(\infty))$ and $\lambda(A) > 0$. 
		$f_+ \geq \infty \cdot \chi_A$.
		
		Then $\int f_+ \dd{\lambda} \geq \int \infty \cdot \chi_A \dd{\lambda} = \infty$.
	\end{proof}
\end{prop}


\begin{prop}
	If $f=g$ a.e. and $f\in \mathcal{L}_1$, then $g\in \mathcal{L}_1$ and $\int f\dd{\lambda}  =\int g\dd{\lambda}$.
\end{prop}


\begin{prop}
	If $f\in \mathcal{L}_1$ and $\int \abs{f} \dd{\lambda} = 0$, then $f=0$ a.e.
	\begin{proof}
		Define 
		$$A_k = \left\{ x: \abs{f(x) } > \frac{1}{k}  \right\}$$
		$$\abs{f} \geq \abs{f} \chi_{A_k} \geq \frac{1}{k} \chi_{A_k}$$
		$$0 =\int \abs{f} \dd{\lambda} \geq \frac{1}{k} \int \chi_{A_k} $$
		Thus $\lambda(A_k) = 0$ and thus
		$$\lambda(\bigcup A_k) = 0$$
	\end{proof}
\end{prop}

\begin{prop}
	$\mathcal{L}^1$ is vector space.
	\begin{proof}
		For $a\geq 0$,
		$$af = af_+ + af_-$$
		$$\int af \dd{\lambda} = \int af_+ - \int af_- = a\int f_+ - a\int f_- = a\int f$$
		For $a<0$, we note that $(af)_\pm = -af_\mp$.
		
		Let $h=f+g$
		$$h_+-h_- = h = (f_+-f_-) + (g_+ - g_-)$$
		$$h_+ + f_- + g_- = h_- + f_+ + g_+$$
		$$\int h_+ +\int f_- + \int g_-=  \int h_- + \int f_+ + \int g_+$$
		Then $h_+ \leq f_++ g_+$ and $\int h_+ < \infty$, i.e. $h\in \mathcal{L}^1$.
	\end{proof}
\end{prop}

\begin{theorem}[Monotonic convergence theorem]
	Let $\left\{ f_n\right\}$ sequence of integrable functions such that $ f_1\leq f_2 \leq  \dots $ and $\sup\limits_n \int f_n \dd{\lambda} <\infty$
	Let $f(x) = \lim_{n\to\infty} f_n(x)$ then $\int f\dd{\lambda} = \lim_{n\to\infty} \int f_n \dd{\lambda}$.
	
	\begin{proof}
		Define $g_n = f_n-f_1$ then $g_n$ are non-negative and thus converge as we've shown.
	\end{proof}
	
\end{theorem}

\begin{theorem}[Dominated convergence theorem]
	Let $\left\{  f_n \right\}$ be a sequence of measurable functions on $\mathbb{R}^n$ such that exists $0\leq g \in \mathcal{L}^1$ such that
	$\abs{f_k(x)} \leq g_x$ a.e.
	
	Let
	$$f(x) = \lim_{k\to \infty} f_k(x)$$
	then
	$$\int f \dd{\lambda} = \lim_{k\to \infty} \int f_k \dd{\lambda}$$
	
	\begin{proof}
		
	\end{proof}
\end{theorem}
\subsection{Riemann integral}
Reminder: Riemann integral in rectangle is limit of Darboux sums: for $I_j$ subrectangle
$$U(P) = \sum M_j  \lambda(I_j)$$
$$D(P) = \sum m_j  \lambda(I_j)$$
where $m_j$ and $M_j$ are supremum and infimum on subrectangle.
$f$ is integrable if sums converge to same number.

Define
$$\tau_P(x) = \begin{cases}
M_j & x\in \inter I_j\\
M & \text{ otherwise}
\end{cases}$$
$$\sigma_P(x) = \begin{cases}
m_j & x\in \inter I_j\\
m & \text{ otherwise}
\end{cases}$$
$\sigma$, $\tau$ are simple, Lebesgue integrable, in particular measurable, and Riemann integrable and continuous a.e.

$$\int \sigma_p \dd{\lambda} = L(P) = \int \sigma_p \dd{x}$$
$$\int \tau_p \dd{\lambda} = U(P) = \int \sigma_p \dd{x}$$
and $\sigma_p \leq f \leq \tau_p$.
\begin{theorem}
	Let $f$ bounded $(\abs{f} \leq M$) on special box $I$ then 
	\begin{enumerate}
		\item $f$ is Riemann integrable iff $f$ continuous a.e.
		\item If $f$ is Riemann integrable it is measurable and 
		$$\int f \dd{\lambda} =  \int f\dd{x}$$
	\end{enumerate}
\begin{proof}
	Suppose $f$ is Riemann integrable, for each $k$ exists partition $P_k$ such that $U(P_k)-L(P_k)< \frac{1}{k}$ and $P_{k-1} \subset P_{k}$. Denote $\sigma_k$, $\tau_k$. $\int \tau_k \dd{\lambda} = U(P_k)$ and $\tau_k\geq f$.
	
	In every point of continuity of $\sigma_p$, $\sigma_k \leq f$ thus
	$$\sigma_k \leq \underline{\sigma}_k \leq \underline{f} $$
	Thus
	$$\sigma_k \leq \underline{f} \leq \bar{f} \leq \tau_k$$
	a.e. Obviously $\sigma_{k+1} \geq \sigma_k$ and $\tau_{k+1} \leq \tau_k$.
	
	Define $g =\sup \sigma_k$ and $h=\inf \tau_k$.  By definition
	$$g \leq \underline{f} \leq \bar{f} \leq h$$
	a.e., and $h-g \leq \tau_k -\sigma_k$.
	$$\int (h-g) \dd{\lambda} \leq \in t\tau_k - \sigma_k \dd{\lambda} = U(P_k) - L(P_k) < \frac{1}{k}$$
	$$\int (h-g) \dd{\lambda} = 0$$
	Thus $h=g$ a.e., and $\underline{f} = \bar{f}$ a.e., thus $f$ continuous a.e.
	
	
	$$\int\limits_I f\dd{x} \leq U(P_k) < L(P_k) + \frac{1}{k} = \int \sigma_k \dd{\lambda} +\frac{1}{k} \leq \int f \dd{\lambda} +\frac{1}{k}$$
	$$\int\limits_I f\dd{x} \geq L(P_k) > U(P_k) -+ \frac{1}{k} = \int \tau_k \dd{\lambda} - \frac{1}{k} \geq \int f \dd{\lambda} -\frac{1}{k}$$
	Thus
	$$\int f \dd{\lambda} -\frac{1}{k} \leq \int f \dd{x} \leq\int f \dd{\lambda} +\frac{1}{k}$$
	i.e., $\int f \dd{\lambda} = \int f \dd{x} $ю
	
	
	Suppose $f$ continuous a.e. denote by $P_k$ partition of each rectangle side to $2^k$ equal parts. We know that
	$$\sigma_1 \leq \sigma_2 \leq \dots \leq f$$
	For $x$ which is not on the boundary,
	$$\underline{f}(x) \leq \lim_{k\to \infty} \sigma_k(x)$$
	else there is $t$ such that
	$$\lim_{k\to \infty} \sigma_k(x)<t<\underline{f}(x)$$
	Thus exists $\delta>0$ such that $\inf\left\{  f(y) : y\in B(x,\delta)   \right\} > t$ and
	$$\lim_{k\to \infty} \sigma_k(x) <t< f(y)$$
	However we can increase $k$ such that the partition will include the whole $\delta$-neighborhood, and thus there is contradiction with the definition of the $\sigma$. Thus 
	
	$$\underline{f}(x) \leq \lim_{k\to \infty} \sigma_k(x)$$
	and similarly 
	$$\bar{f}(x) \geq \lim_{k\to \infty} \tau_k(x)$$
	a.e.
	
	Since $f$ is continuous a.e., $f=\underline{f}=\bar{f}$ a.e. Thus
	$$ \lim_{k\to \infty} \sigma_k(x) \geq f \geq \lim_{k\to \infty} \tau_k(x)$$
	a.e. However $\sigma_k \leq \tau_k$, i.e.
	$$ \lim_{k\to \infty} \sigma_k(x) = f = \lim_{k\to \infty} \tau_k(x)$$
	a.e.
	$$ f = \lim_{k\to \infty} \int \sigma_k(x) \dd{\lambda} = \lim_{k\to \infty} \int  \tau_k(x) \dd{\lambda}$$
	where we switch limit and integral by DCT with $M\chi$.
	We got that
	$$\lim U(P_k) = \lim L(P_k) $$
	and thus $f$ is Riemann integrable.
\end{proof}
\end{theorem}
\begin{prop}
	The Riemann and Lebesgue integrals are equivalent only for bounded functions on finite intervals.
\end{prop}
\begin{theorem}
	Let $T: \mathbb{R}^n \to \mathbb{R}^n$ be a linear transformation and $A\subseteq \mathbb{R}^n$. Then
	$$\lambda^*(TA) = \abs{\det{T}} \lambda^*(A)$$
	$$\lambda_*(TA) = \abs{\det{T}} \lambda_*(A)$$
	and if $A$ is measurable, $TA$ is measurable and
	$$\lambda(TA) = \abs{\det{T}} \lambda(A)$$
	\begin{proof}
		If $T$ is not invertible, then dimension of the image is less than dimension of the space, and thus 
		$$\lambda^*(TA) = \lambda_*(TA) = 0$$
		and since $\det T = 0$, we are done.
		
		
		If $T$ is not invertible, denote matrix corresponding to $T$ as $B$. We can rewrite $B$ as
		$$B=E_1E_2\dots E_n$$. 
		Assuming $A$ is special rectangle, without loss of generality it is
		$$A  =[0,b_1]\times[0,b_2]\times\dots\times [0,b_n]\times$$
		If $E_i$ multiplies row by $c$
		then
		$$EA  =[0,b_1]\times[0,b_2]\times\dots\times[0,cb_i]\times\dots\times [0,b_n]\times$$
		$$\lambda(EA) = \abs{c} \lambda(A) = \abs{\det E}\lambda(A)$$
		If $E_i$ swaps two rows the area doesn't change.
		If $E_i$ adds row to another row, we get a parallelogram whose area still equals to original.
		
		We continue step-by-step for steps of building of Lebesgue measure.
	\end{proof}
\end{theorem}

\begin{theorem}
	Let $T$ invertible linear transformation on $\mathbb{R}^n$ and $f$ function on $\mathbb{R}^n$. 
	\begin{enumerate}
		\item If $f$ is measurable $f\circ T$ is measurable
		\item If $f\geq 0$ measurable then $\int f \dd{\lambda} = \abs{\det T} \int f(Tx) \dd{\lambda}$.
		\item If $f$ is measurable then $\int f \dd{\lambda} = \abs{\det T} \int f(Tx) \dd{\lambda}$.
	\end{enumerate}
\end{theorem}
	\begin{theorem}
		Let $T$ linear transformation and $f$ function defined on $\mathbb{R}^n$ then
		\begin{enumerate}
			\item If $f$ measurable then so is $f\circ T$
			\item If $f$ is measurable and $f\geq 0$ then $\int f \dd{\lambda} = \abs{\det(T)} \int f(Tx)\dd{\lambda}$.
			\item If $f\in \mathcal{L}^1$ then $f\circ T \in L^1$.
		\end{enumerate}
	\end{theorem}
	
	\section{$L^1(\mathbb{R}^n )$ space}
	\begin{definition}
		Vector space $X$ over $\mathbb{R}$ is called norm space if exists function $\norm{\cdot} : X\to \mathbb{R}$ such that
		\begin{enumerate}
			\item $\norm{x} \geq 0$
			\item $\norm{x} = 0 \iff x=0$
			\item For $c\in \mathbb{R}$ $\norm{cx} = \abs{c} \norm{x}$
			\item $\norm{x+y} \leq \norm{x} + \norm{y}$
		\end{enumerate}
	\end{definition}
	Norm space is metric space with metric $d(x,y) = \norm{x-y}$.
	
	\begin{definition}[Convergence]
		$x_n\to x$ if $d(x_n, x) \to 0 \iff \norm{x_n-x}\to 0$.
		
		If $x_n\to x$ then $\norm{x_n} \to \norm{x}$.
	\end{definition}
	\begin{definition}[Open and closed balls]
		Open ball: $B(x,r) = \left\{ y\in X | \norm{y-x} < r \right\}$
		Closed ball: $B(x,r) = \left\{ y\in X | \norm{y-x} \leq r \right\}$
	\end{definition}
	\begin{definition}[Open set]
		$A\subset X$ is open if $\forall a\in A \exists r>0$ such that $B(a,r) \subseteq A$.
	\end{definition}
	\begin{definition}[Closed set]
		$A\subset X$ is open if $A \ni x_n \to x \Rightarrow x\in A$.
	\end{definition}
	\begin{definition}[Closure]
		$\bar{A} $, the closure of $A$:
		$$\bar{A} = \left\{ x: \exists x_n \in A , x_n \to x \right\}$$
	\end{definition}
	\begin{definition}[Dense set]
		$A$ is dense if $\bar{A} = X$.
	\end{definition}
	
	For $f\in \mathcal{L}^1(\mathbb{R}^n)$ denote
	$$\norm{f}_1 = \int \abs{f} \dd{\lambda}$$
	
	Note that this is almost norm, except that there exists non-zero functions with norm $0$. To fix it, we define
	$$N = \left\{ f\in \mathcal{L}^1 : f=0 \text{ a.e.} \right\}$$
	and
	$$L_1(\mathbb{R}^n) = \mathcal{L}^1(\mathbb{R}^n)/N$$
	i.e., we define equivalence relation, $f\sim g$ if $f=g$ a.e.
	
	Usually we'll look at members of $L_1$ as functions and not as equvivalence classes. However, it's impossible to talk about value or limit of function in point in this case. Now we can define
	$$\norm{f}_1 = \int \abs{f} \dd{\lambda}$$
	norm on $L_1(\mathbb{R}^n)$.
	
	\begin{definition}
		$C_c(\mathbb{R}^n)$ are continuous functions with compact support, i.e., $f=0$ outside of some ball.
	\end{definition}
	
	
	\begin{definition}
		$[f]\in L_1(\mathbb{R}^n)$ is continuous if there is continuous $f\in [f]$.
	\end{definition}

\section{Question 1}
$L^1_C(\mathbb{R}^n)$ - functions in $L^1$ with compact support a.e.

$\mathcal{C}^m$ - function with continuous partial derivatives up to order $m$.

$\mathcal{C}^m_c$ - function in $\mathcal{C}^m$ with constant support.

$\mathcal{C}^\infty = \bigcap_{k=1}^\infty \mathcal{C}^k$.


Convergence in $L^1$:
$$f_n\stackrel{L^1}{\to} f  \Rightarrow \int \abs{f_n-f} \dd{\lambda} \to 0$$

If $Y_1\subset Y_2$ then $\bar{Y}_1 \subset \bar{Y}_2$. Also $\bar{\bar{Y}} = \bar{Y}$.

\begin{lemma}
	If $f_n\to f$ a.e. and $\abs{f_n}<\abs{f}$ and $\left\{  f_n, f\right\} \subseteq L^1$ then $f_n\stackrel{L^1}{\to} f  $.
	\begin{proof}
		Since
		$$\int \abs{f_n-f} \dd{\lambda} \stackrel{a.e.}{\to} 0$$
		$$\abs{f_n-f} \leq \abs{f_n}+\abs{f} \leq 2\abs{f} \in L^1$$
		From dominated convergence,
		$$\int\abs{f_n-f} \dd{\lambda} = 0$$
	\end{proof} 
\end{lemma}


\begin{lemma}[Urysohn's lemma ] \label{urysohn}
	Let $B$,$C$ closed disjoint sets, then exists continuous $f:\mathbb{R}^n \to [0,1]$ such that $\eval{f}_C = 1$ and $\eval{f}_B=0$.
\end{lemma}
\begin{lemma}
	Simple function are dense in $L^1$.
	\begin{proof}
		First lets show that $L_c^1$ is dense in $L^1$. Let $f\in L^1$ and denote
		$$f_n = f\chi_{B_0(n)}$$
		Since $f_n \stackrel{a.e.}{\to} f$ and $\abs{f_n} \leq \abs{f}$ we get 
		$$L_c^1 \ni f_n \stackrel{L^1}{\to} f $$
		Thus $L_c^1$ is dense in $L^1$.
		
		Now lets show that simple functions are dense in $L_c^1$. It's enough to show that for nonnegative functions. For each function $f\in L^1_c$ exists sequence $s_k\geq 0$ such that $s_k \stackrel{a.e.}{\to} f$ and $\abs{s_k} \leq \abs{f}$ we get 
		$$s_k \stackrel{L^1}{\to} f $$
		
	\end{proof} 
\end{lemma}


\begin{lemma}
	$C_c$ is dense in $L^1$.
	\begin{proof}
		We want to show that any nonnegative simple function with compact support can be approximated with functions from $C_c$. It's enough to show that for $\chi_A$, where $A$ is closed and measurable. Suppose $A\subseteq B_0(n)$. $\forall \epsilon >0$ exists open $G$ and compact $K$ such that 
		$$K\subseteq A \subseteq G$$
		$G\subseteq B_0(n)$.
		
		From \ref{urysohn}, exists $g: \mathbb{R}^n \to [0,1]$ such that 
		$\eval{g}_{K} = 1$ and $\eval{g}_{\mathbb{R}^n \setminus G}=0$. $g\in C_c$.
		$$g(x) - \chi_A(x) = \begin{cases}
		0&x\in K\\
		0& x \in \mathbb{R}^n \setminus G\\
		\leq 1 & x\in G\setminus K
		\end{cases}$$
		$$\int \abs{g(x) - \chi_A(x)} \dd{\lambda} = \int\limits_{G\setminus K} \abs{g(x) - \chi_A(x)} \dd{\lambda} \leq \lambda(G\setminus K) < \epsilon$$
	\end{proof} 
\end{lemma}

\begin{theorem}
	Let $f\in L_1$ and $f_y(x) = f(x+y)$
	$$\norm{f_y-f} \stackrel{y\to0}{\to} 0$$
	\begin{proof}
		Let $f\in L^1$ and $\epsilon>0$. Exists $g\in C_c$ such that $\norm{f-g} <\frac{\epsilon}{3}$. Suppose $g=0$ outside of $B_0(r)$.
		
		$$\int \abs{f-g} \dd{\lambda} < \frac{\epsilon}{3} $$
		But
		$$\int \abs{f_y-g_y} \dd{\lambda} < \frac{\epsilon}{3} $$
		i.e., $\norm{f_y-g_y} <\frac{\epsilon}{3}$.
		$g$ is uniformly continuous. Thus exists $1>\delta>0$ such that if $\norm{y}<\delta$ then
		$$\abs{g(x+y)-g(x)} <\frac{\epsilon}{3\lambda(B_0(r+1)}$$
		For $\norm{x} > r+1$ 
		$$\abs{g(x+1)-g(x) } = 0$$
		Thus for $\norm{y}<\delta$
		$$\int \abs{g_y(x) -g(x)} \dd{\lambda} = \int\limits_{B_0(r+1)} \abs{g(x+y)-g(x)} \leq \lambda(B_0(r+1)) \frac{\epsilon}{3\lambda(B_0(r+1)} = \frac{\epsilon}{3}$$
		Thus
		$$\norm{f_y-f} \leq \norm{f_y-g_y} + \norm{g_y-g} + \norm{g_y-f} <\epsilon$$		
	\end{proof}
\end{theorem}

\begin{definition}
	We say that $f_n\to f$ in measure if 
	$$\lambda\qty(\left\{ x: \abs{f_n(x) - f(x)} \geq \epsilon  \right\}) = 0$$
\end{definition}

Note that $f_n\stackrel{L^1}{\to} f \Rightarrow f_n\stackrel{\text{measure}}{\to} f$.

\begin{definition}
	$\left\{x_n\right\}$ in metric space $X$ is called Cauchy sequence if for all $\epsilon>0$ exists $N$ such that for all $n,m>N$
	$$\norm{x_n-x_m} < \epsilon$$
\end{definition}

\begin{definition}
	Metric space is called complete space if each Cauchy sequence converges.
\end{definition}
\begin{definition}
	Complete normed space is called Banach space.
\end{definition}
\begin{theorem}
	$L^1$ is Banach space.
	\begin{proof}
		Let $\left\{ f_n \right\}$ Cauchy sequence in $L^1$. For all $\epsilon>0$ exists $N(\epsilon)$ such that $\norm{f_n-f_m}<\epsilon$ for $n,m>N$. We can require that $\left\{ N\qty(\frac{1}{2^k}) \right\}$ is non-decreasing sequence.
		Denote $n_k = N\qty(\frac{1}{2^k})$ and $g_k = f_{n_k}$.
		
		For all $k$:
		$$\int \abs{g_k-g_{k+1} } = \norm{g_k-g_{k+1}} = \abs{f_{n_k} - f_{n_{k+1}}} < \frac{1}{2^k}$$
		Thus
		$$\sum_{k=1}^\infty \int \abs{g_k-g_{k+1} } < \infty$$
		
		We've shown that from that we can conclude that
		$$h(x) = \sum_{k=1}^\infty (g_{k+1}-g_k)$$
		exists and $h\in L^1$.
		
		$$\int h\dd{\lambda} = \sum_{k=1}^\infty g_{k+1}-g_k \dd{\lambda}$$
		$$h(x) = \lim_{N\to \infty} \sum_{k=1}^N  g_{k+1}-g_k = \lim_{N\to \infty} g_N(x) - g_1(x)$$
		
		Thus
		$$\lim_{N\to \infty} g_N(x) = h(x) + g_1(x)$$
		
		On the other hand
		$$\sum_{k=m}^\infty g_{k+1}(x)- g_k(x) =\lim_{N\to \infty} g_N(x) - g_m(x) = g_1(x) + h(x) - g_m(x) $$
		$$\norm{g_1+h-g_m} = \norm{\sum_{k=m}^\infty g_{k+1} - g_k} \leq \sum \norm{g_k-g_{k+1}}\leq \sum_{k=m}^\infty \frac{1}{2^k} \stackrel{m\to \infty}{\to} 0$$
		
		Thus $g_m \stackrel{L_1}{\to} g_1+h$ meaning partial sequence of $\left\{f_n\right\}$ converges and thus  $\left\{f_n\right\}$ converges.
	\end{proof}

\begin{coll}
	If $f_n \stackrel{L_1}{\to}  f$ then exists subsequence $\left\{ f_{n_k} \right\}$ such that $f_n \stackrel{\text{a.e.}}{\to}  f$ 
	\begin{proof}
		$g_k = f_{n_{k}} \to g_1+h$ a.e. and $g_k = f_{n_k} \stackrel{L_1}{\to}  g_1+h$  but $g_k \stackrel{L_1}{\to}  f_1$ thus $f=g_1+h$ a.e. and thus $g_k\to f$ a.e.
	\end{proof}
\end{coll}
\end{theorem}

\begin{prop}
	In normed space if exists $\sum x_n$ then
	$$\norm{\sum x} \leq \sum \norm{x}$$
\end{prop}

Note that since there is unique completion of metric space we could define $L^1$ differently. Start with $X=C_c(\mathbb{R}^n)$ with Riemann integral as a  metric. Its completion is $L^1$ and we can define integral as a limit of Riemann integrals.

\subsection{Parameter-dependent integrals}
Let $J\subseteq \mathbb{R}$ and $f: \mathbb{R}^n\times \mathbb{R} \to [-\infty,\infty]$ and
$$f_t: x \mapsto f(x,t) \in \mathbb{L}^1(\mathbb{R}^n)$$
Denote
$$F(t) = \int f_t(x) \dd{x}$$

\begin{theorem}
	Let $f$, $F$. Suppose $h\in L^1(\mathbb{R}^n)$ such that
	$$\forall x,t \quad \abs{f(x,t)} \leq g(x)$$
	Let $t_0\in J$. Suppose for almost every $x$ $t\mapsto f(x,t)$ if continuous in $t_0$. Then $F$ is continuous in $t_0$.
	
	\begin{proof}
		We want to show that
		$$\lim F(t_n) = F(t_0)$$
		for $t_n\to t_0$.
		
		$$F(t_n) = \int f_{t_n} \dd{\lambda}$$
		$$F(t_0) = \int f_{t_0} \dd{\lambda}$$
		also, $f_{t_n} \to f_{t_0}$ a.e., from DCT we get the required.
	\end{proof}
\end{theorem}

\begin{theorem}
	Let $J$ open and assume that for almost all $x$ ($x\notin N$) $t\mapsto f_t(x)$  differentiable for all $t\in J$. Also assume that exists $h\in L^1$ such that for all $x$ and $t$ 
	$$\pdv{f}{t} \qty(x,t) \leq (x)$$
	Then $F$ is differentiable in $J$ and
	$$\dv{F}{t} \qty(t) = \int \pdv{f}{t} \qty(x,t)  \dd{\lambda(x)}$$
	\begin{proof}
		For all $t\in J$ $\pdv{f}{t} \qty(x,t) $ is measurable as a limit of measurable functions. Choose $t\in J$ and $\delta>0$ such that $t+s \in J$ for all $\abs{s}<\delta$. Define
		$$g(x,s) =\begin{cases}
		\pdv{f}{t} \qty(x,t) & x\notin N, s=0\\
		\frac{f(x,t+s)-f(x,t)}{s}& x\notin N, s\neq0\\
		0& x\in N
		\end{cases}$$
		Now
		$$\int g(x,0) \dd{\lambda(x)}= \int \pdv{f}{t} \qty(x,t) \dd{\lambda(x)}$$
		$$\int g(x,s) \dd{\lambda(x)}= \frac{F(t+s)+F(t)}{s}$$
		Thus we need
		$$\lim_{s\to 0} \int g(x,s) \dd{\lambda(x)} = \int g(x,0) \dd{\lambda(x)}$$
		Note that $s\mapsto g(x,s)$ is continuous in $s=0$ and $x\mapsto g(x,s)$ is integrable for all $s$. Also $\abs{g(x,s)} \leq h(x)$ (for $s>0$ from Lagrange) and thus from previous theorem we get the required.
	\end{proof}
\end{theorem}

\paragraph{Functions in $C^\infty_c$}
$$h(t) = \begin{cases}
0&t\leq 0\\
e^{-\frac{1}{t}}& t>0
\end{cases}$$

$h$ is differentiable infinite times
$$h^{(n)} = P_{2n} \qty(\frac{1}{t}) e^{-\frac{1}{t}}$$
	
Define
$$\tilde{\phi} (x) = h(1-x_1^2-x_2^2-\dots x_n^2) \in C^\infty(\mathbb{R}^n)$$
Note that $\tilde{\phi}(x) = 0$ if $\norm{x} \geq 1$, thus $\tilde{\phi}(x) \in C_c^\infty(\mathbb{R}^n)$ denote $C = \int \tilde{\phi} \dd{\lambda}$ and define
$$\phi(x) = \frac{1}{C} \tilde{\phi}(x)$$
We get $\phi(x) \in C_c^\infty(\mathbb{R}^n)$, $\phi\geq 0$, $\int \phi \dd{\lambda} = 1$ and $\phi(x)>0$ iff $\norm{x} <1$.
	Let $f: \mathbb{R}^n = \mathbb{R}^l\times \mathbb{R}^m \to \mathbb{R}$. Denote for $y\in \mathbb{R}^m$ and $x\in \mathbb{R}^l$ $f_y(x) = f(x,y)$


\begin{lemma}
	If the theorem is right for a sequence of functions $0\leq f_1 \leq f_2 \dots $ and $f=\lim_{j\to \infty}f_j$ then the theorem is right for $f$.
	\begin{proof}
		For all $y\in \mathbb{R}^m$ $(f_j)_y \uparrow f_y$, thus $f$ is measurable for almost all $y$:
		$$F_j(y) = \int (f_j)_y \dd{x} \uparrow\int f_y \dd{x} = F(y)$$
		$$\int\limits_{\mathbb{R}^m} F \dd{y} = \lim_{j\to \infty} \int F_j \dd{y} = \lim_{j\to \infty} \int f_j \dd{\lambda} = \int f \dd{\lambda}$$
	\end{proof}
\end{lemma}

\begin{theorem}[Fubini theorem]
Let $f: \mathbb{R}^n \to [-,\infty]$ be a measurable function. THen for almost all $y\in \mathbb{R}^m$ a function $f_y$ is measurable and thus $F(y) \int\limits_{\mathbb{R}^l} f_y(x) \dd{\lambda(x)}$ and $F$ is measurable on $\mathbb{R}^m$ and
	$$\int\limits_{\mathbb{R}^m} F(y) \dd{\lambda(y)} = \int\limits_{\mathbb{R}^n} f \dd{\lambda}$$
\end{theorem}


\section{Differentiation and integration in $\mathbb{R}$}

\begin{definition}
	Let $E\subseteq \mathbb{R}$ and $M$ set of closed intervals  of positive lengths. $M$ is called Vitali cover if for all $\epsilon>0$ and $\forall x\in E$ $\exists I\in M$ such that $x\in I$ and $\lambda(I)<\epsilon$.
\end{definition}
\begin{theorem}[Vitaly]
	Let $E\subset \mathbb{R} $, $\lambda^*(E) <\infty$ and $M$ Vitaly cover of $E$. Then either $E$ is contained in finite disjoint union $\bigcup_{k=1}^N I_K$ for $I_k\in M$ or exists $\left\{ I_k\right\}_{k=0}^\infty \subset M$ such that $I_k\cap I_j = \emptyset$ and for all $\epsilon>0$ exists $N=N(\epsilon)$ such that $\lambda^*(E\setminus \bigcup^N I_k) < \epsilon$
\end{theorem}
\begin{definition}
	
	Let $g$ defined in neighbourhood of $x_0$. Define 
	$$\liminf_{x\to x_0} g(x)  = \sup\limits_{\delta > 0} \qty(\inf \left\{ g(y): 0<\abs{y-x_0} < \delta \right\})$$
	$$\limsup_{x\to x_0} g(x)  = \inf\limits_{\delta > 0} \qty(\sup \left\{ g(y): 0<\abs{y-x_0} < \delta \right\})$$
\end{definition}

\begin{lemma}
	\begin{enumerate}
		\item  $$\liminf g \leq \limsup g$$
		\item  If $g(x) \geq M$ 
		$$\liminf g \geq M$$
		\item  If $g(x) \leq M$ 
		$$\limsup g \leq M$$
		\item  $\lim g$ exists if $\liminf g = \limsup g$
	\end{enumerate}
\end{lemma}

\begin{definition}
	$$UD f(x) = \limsup_{h\to 0} [f(x+h) - f(x)]$$
	$$LD f(x) = \liminf_{h\to 0} [f(x+h) - f(x)]$$
\end{definition}

\begin{lemma}
	Let $A\subseteq [a,b]$ such that for all $x\in A$, $0\leq LD f(x) \leq r$ then
	$$\lambda^*(f(A)) \leq r \lambda^*(A)$$
	\begin{proof}
		Let $\epsilon>0$ and $G$ open such that $A\subseteq G$ and $\lambda(G) \leq \lambda^*(A)+\epsilon$. Thus
		$$r . \sup\limits_{\delta > 0} \qty(\inf \left\{ \frac{1}{h}\qty[f(x+h)-f(x)] : 0 < \abs{h} < \delta  \right\})$$
		
		For all $x\in A$ choose $\delta_0 > 0$ such that $(x-\delta_0,x+\delta_0) \subseteq G$. For all $\delta<\delta_0$ exists $0<\abs{h}<\delta$ such that $\frac{1}{h} \qty[f(x+h) - f(x)] < r$. Denote
		$$M = \left\{ [f(x), f(x+h)] : [x,x+h]\subseteq G,\: \frac{1}{h} \qty[f(x+h) - f(x)] < r \right\}$$
		
		Then $M$ is Vitali cover of $f(A)$, since $$\lambda\qty([f(x), f(x+h)]) < \abs{h} r < \delta r$$
		
		By theorem, exists 
		$$\left\{ \qty[f(x+h) - f(x)]  \right\} \subseteq M$$
		of disjoint intervals such that
		$$\lambda^*\qty(f(A)\setminus \bigcup [f(x_n), f(x_n+h_n)]) = 0$$
		Thus
		$$\lambda^*(f(A)) \leq \lambda^*\qty(f(A)\setminus \bigcup [f(x_n), f(x_n+h_n)]) + \lambda^*\qty(\bigcup [f(x_n), f(x_n+h_n)]) \leq \sum \lambda[f(x_n), f(x_n+h_n)] = \sum \abs{[f(x_n)- f(x_n+h_n)} \leq \sum r \abs{h_n} \leq r\lambda(G) \leq r\lambda^*(A) + r\epsilon$$
	\end{proof}
\end{lemma}

\begin{lemma}
	Let $A\subseteq [a,b]$ such that for all $x\in A$ 
	$$UD f(x) > s>0$$
	then $\lambda^*(f(A)) \geq s \lambda^*(A)$
	\begin{proof}
		Let $G$ open. $f(A) \subseteq G$ and $\lambda(G) \leq \lambda^*(A)+\epsilon$.
		Since $f$ has only countable number of discontinuity points,  we can remove a null set of such points and assume $f$ is continuous on $A$.
		
		Thus, for all $x \in A$ exists $\delta_0(x) > 0$ such that 
		$$[f(x-\delta), f(x+\delta)] \subseteq G$$
		for all $0<\delta < \delta_0$ and for all $\delta_0$ exists $\abs{h}$ such that
		$0<\abs{h} < \delta$ and 
		$$s < \frac{1}{h} \qty[f(x+h) - f(x)] $$
		and $\qty[ f(x),f(x+h) ] \subseteq G$.
		
		Let 
		$$M = \left\{ [x,x+h]  \right\}$$
		which is Vitali cover.
		
		$$\lambda^*(A) \leq \lambda^*\qty(A \setminus \bigcup [x_n, x_n+h_n]) + \lambda^*\qty(\bigcup [x_n, x_n+h_n]) =\leq \lambda \bigcup [x_n, x_n+h_n] = \sum \abs{h_n} \leq \frac{1}{s} \sum \abs{f(x_n+h_n) - f(x_n) } < \frac{1}{s} \lambda(G) \leq \frac{1}{s} \lambda^*(f(A)) + \frac{1}{s} \epsilon$$
	\end{proof}

\begin{coll}
	If $f$ is monotonic on $[a,b]$ and $A = \left\{ x\in[a,b] : \: UD f(x) =\infty \right\}$ then $\lambda^*(A) = 0$.
	\begin{proof}
		For all $s>0$
		$$\infty > f(b)-f(a) = \lambda([f(a),f(b)]) = \lambda^*(f([a,b])) \geq \lambda^*(f(A) ) \geq s\lambda^*(A)$$
		Thus $\lambda^*(A) = 0$.
	\end{proof}
\end{coll}
\end{lemma}

\begin{theorem}[Lebesgue]
	If $f$ is monotoneous on $[a,b]$ it is differentiable a.e. there.
	\begin{proof}
		Assume $f$ is non-decreasing and if we replace $f$ with $f(x)+x$ it is increasing.
		From corollary, $UD f(x) < \infty$ a.e. and it has to be $UD f(x) = LD f(x)$ a.e.
		
		For all $r<s$ in $\mathbb{Q}$ define
		$$A_{r,s} = \left\{ x: LD f(x) < r <s<UD f(x) \right\}$$
		$$\bigcup A_{r,s} = \left\{ x: LD f(x) <UD f(x) \right\}$$
		It's enough to show that for any $r$, $s$ $\lambda^*(A_{r,s}) = 0$.
		
		From lemmas we get
		$$s\lambda^*(A_{r,s}) \leq \lambda^*(fA_{r,s}) \leq r \lambda^*(A_{r,s})$$
		Bur $r<s$ and thus $\lambda^*(A_{r,s}) =0$.
	\end{proof}
\begin{coll}
	For all $f\in L^1[a,b]$ the function $F(x) = \int_a^x f \dd{\lambda}$ os differentiable a.e.
	\begin{proof}
		Use $f_+$ and $f_-$.
	\end{proof}
\end{coll}
\end{theorem}

\begin{prop}
	Let $f$ non-decreasing on $[a,b]$ then $f'$ is measurable and integrable and 
	$$\int_a^b f' \dd{\lambda} \leq f(b)-f(a)$$
	\begin{proof}
		Define $f$ to be $f(b)$ on $[b,b+1]$ from Lebesgue theorem $f'$ exists a.e. If $f'$ exists, 
		$$f'(x) = \lim_{n\to \infty}\underbrace{ n\qty[f\qty(x+\frac{1}{n}) -f(x)]}_{g_n(x) \geq 0}$$
		$g_n$ is measurable and so is $f'$. By Fatou lemma
		\begin{align*}
		\int_a^b f' \dd{\lambda} \leq \liminf \int g_n \dd{\lambda} = \liminf \qty(n \int_a^b \qty[f\qty(x+\frac{1}{n}) -f(x)] \dd{\lambda}) = \liminf n \int_a^b f\qty(x+\frac{1}{n}) \dd{\lambda} -n \int_a^b  f(x) \dd{\lambda}= \liminf n \int_{a+\frac{1}{n}}^{b+\frac{1}{n}} f\qty(x) \dd{\lambda} -n \int_a^b  f(x) \dd{\lambda} =  \liminf n \int_{b}^{b+\frac{1}{n}} f\qty(x) \dd{\lambda} -n \int_a^{a+\frac{1}{n}}  f(x) \dd{\lambda} \leq f(b) - f(a)
		\end{align*}
	\end{proof}
\end{prop}
\subsection{Functions of bounded variation}
\begin{definition}
	Let $f$ real on $[a,b]$ and $T = \left\{ T_i \right\}_{i=0}^n$ partition of $[a,b]$. Denote
	$$V_f(T) = \sum_{i=1}^n \abs{f(t_i) - f(t_{i-1})}$$
	And
	$$V_a^b f = \sup\limits_T \left\{ V_F(T) \right\}$$
	variation of $f$
\end{definition}

\begin{definition}
	We say $f$ has bounded variation if $f \in BV[a,b]$. Note that
	$V_a^b f + V_b^c f = V_a^c f$.
\end{definition}

\begin{prop}
	Let $f\in BV [a,b]$ and define for all $x\in [a,b] $
	$$g(x) = V_a^x f$$
	Then $g$ is monotonous non decreasing, and if $f$ is continuous from left/right, so is $g$.
	
	\begin{proof}
		$$V_a^y f = V_a^x f + V_x^y f$$
		
		Now suppose $f$ is continuous from the left in $x_0$. Choose partition of $[a,x_0]$ such that $V_f(T) > V_a^{x_0} f - \frac{\epsilon}{2}$. From continuity from the left exists $t_{n-1}<y<x_0$ such that $\abs{f(x_0) - f(y)} <\frac{\epsilon}{2}$ Now
		$$g(y) = V_a^y f \geq \sum_{i=1}^{n-1} \abs{f(t_i) - f(t_{i-1})} + \abs{f(y)-f(t_{n-1})} > \sum_{i=1}^{n-1} \abs{f(t_i) - f(t_{i-1})} + \abs{f(y)-f(t_{n-1})} + \abs{f(x_0) - f(y)} -\frac{\epsilon}{2} \geq V_f(T) - \frac{\epsilon}{2} > g(x_0) - \frac{\epsilon}{2}$$
		
		Now let $y\leq z\leq x_0$ then $\abs{g(x_0) - g(z)} = g(x_0) - g(z) \leq  g(x_0) - g(y) < \epsilon$
	\end{proof} 
\end{prop}


\begin{theorem}
	$f\in BV[a,b]$ iff $f$ is difference of two non-decreasing monotonic functions. If $f$ is continuous we get difference of two continuous functions then $f\in BV[a,b]$ and $f$ differentiable a.e. and Reimann integrable.
	
	\begin{proof}
		$f(x) = V_a^x(f) - [V_a^xf - f(x)]$, we need $v_a^x f - f(x) = h(x)$ monotonous non-decreasing. If $x<y$
		$$f(y)-f(x) \leq \abs{f(x) -f(y)}\leq V_x^yf$$
		$$h(y)-h(x) = V_a^y - f(y) - V_a^x +f(x) = V-x^y - f(y) +f(x) \geq 0$$
	\end{proof}
\end{theorem}

\begin{prop}
	Let $f$ integrable on $[a,b]$, define $F(x) = \int_a^x f \dd{\lambda}$ then $F$ is continuous and $F\in BV[a,b]$.
	
	\begin{proof}
		Continuity of integral relative to the boundaries gives us continuity of $F$. $F\in BV[a,b]$ since $F = \int_a^x f_+ - \int_a^x f$, i.e., $F$ is a difference of two non-decreasing monotonic functions.
	\end{proof}
\end{prop}

\begin{theorem}
	Let $f$ integrable on $[a,b]$. Define
	$$F(x) = \int_a^x f \dd{\lambda}$$
	Then $F$ is differentiable a.e. and $F'=f$ a.e.
	
	\begin{proof}
		We'll proof that $\int_a^x F' \dd{\lambda} = \int_a^x f \dd{\lambda} $.
		
		Denote $M = \sup \abs{x}$. 
		$$f_n(x) = n\qty[F\qty(x+\frac{1}{n}) -F(x)] \to F'(x)$$
		
		Define $f(x) = 0$ for $x>b$, i.e. $F(x) = F(b)$ for those $x$.
		$$\abs{f_n(x)} = \abs{n\qty[F\qty(x+\frac{1}{n}) -F(x)] } = n\abs{\int_x^{x+\frac{1}{n}} f\dd{\lambda} } \leq M$$
		Since $\abs{f_n} \leq M \chi_{[a,b]} $ and $f_n\to F'$ a.e., we get
		\begin{align*}
		\int_a^c F' \dd{\lambda} = \lim_{n\to \infty} \int_a^c f_n \dd{\lambda} = \lim_{n\to \infty} n\int_a^c F\qty(x+\frac{1}{n}) - F(x) \dd{\lambda} =\lim_{n\to \infty} n\int_a^c F\qty(x+\frac{1}{n})\dd{\lambda} -n\int_a^c F(x) \dd{\lambda}=\\=\lim_{n\to \infty} n\int_{a+\frac{1}{n}}^{c+\frac{1}{n}} F\qty(x)\dd{\lambda} -n\int_a^c F(x) \dd{\lambda} = \lim_{n\to \infty} n\int_{a}^{a+\frac{1}{n}} F\qty(x)\dd{\lambda} - \lim_{n\to \infty} n\int_{c}^{c+\frac{1}{n}} F\qty(x)\dd{\lambda} =\\= F(c) - F(a) = \int_a^c f \dd{\lambda}
		\end{align*}
	\end{proof}
\end{theorem}
\begin{definition}
	$f$ is uniformly continuous on $[a,b]$ if for all $\epsilon>0$ exists $\delta>0$ such that 
	$$\abs{x-y}<\delta \Rightarrow \abs{f(x)-f(y)} <\epsilon$$
\end{definition}
\begin{definition}
	$f$ is absolutely continuous on $[a,b]$ ($f\in AC[a,b]$) if for $\left\{ (x_i,y_i) \right\}_{i=1}^m$ are disjoint open intervals in $[a,b]$ such that $\sum (y_i-x_i) <\delta$ then
	$$\abs{f(x)_i-f(y)_i} <\epsilon$$

\end{definition}
\begin{prop}
	$AC$ is subspace.
\end{prop}
\begin{prop}
$f\in AC[a,b] \Rightarrow f \in BV[a,b]$
\begin{proof}
	
\end{proof}
\end{prop}
\begin{prop}
$f\in AC[a,b]$ and $f'=0$ a.e., then $f$ is constant.
\begin{proof}
	Choose $a<c\leq b$ and show that $f(c)=f(a)$. Denote $E = \left\{ x \in (a,c) : f'(x) \right\}$ and $\lambda\qty((a,c) \setminus E) = 0$. Let $s>0$. For all $x\in E$
	$$\lim_{h\to 0} \frac{\abs{f(x+h)-f(x)}}{\abs{h}} = 0$$
	Thus for $h>0$ small enough
	$$\abs{f(x+h)-f(x)} < \frac{sh}{n-a}$$
	$$[x,x+h] \subseteq (a,c)$$
	Denote
	$$M = \left\{ [x,x+h]: x\in E, h>0 \right\}$$
	Thus is Vitaly cover (it obviously covers $[a,c]$ an the size of interval depends on $h$). 
	
	Let $\epsilon>0$ and corresponding $\delta>0$ by absolute continuity. By Vitaly theorem exists finite set of disjoint intervals $\left\{  [x_k,x_k+h_k]\right\}\subseteq M$ such that
	$$\lambda\qty(E\setminus \bigcup  [x_k,x_k+h_k]) <\delta$$
	and then
	$$\lambda\qty((a,c)\setminus \bigcup  [x_k,x_k+h_k]) <\delta$$
	Then 
	$$\sum x_{i+1} - y_i <\delta$$
	and thus
	$$\sum \abs{f(x_{i+1})-f(y_i)} <\epsilon$$
	$$\sum \abs{f(y_{i})-f(y_i)} <\sum \frac{s}{b-a} (y_k-x_k) \leq s$$
	$$\abs{f(c) - f(a) } = \abs{\sum f(x_{k+1}) - f(y_k) + \sum f(y_k) -f(x_k)} \leq \sum \abs{f(x_{k+1}) - f(y_k)} + \sum \abs{f(y_k) -f(x_k)} \leq \epsilon+s  $$
\end{proof}
\end{prop}

\begin{theorem}
	$f$ on $[a,b]$ is absolutely continuous iff exists $h\in \mathcal{L}^[a,b]$ and $c\in \mathbb{R}$ such that
	$$f(x) = c + \int_a^x h \dd{\lambda}$$
	In this case $h=f'$ a.e.
\begin{proof}
	
	$\Rightarrow$:
	$h\in \mathcal{L}^1$, $f(x) = c+\int_a^x h \dd{\lambda}$ thus we've seen that for $\lambda(E) < \delta$
	$$\int\limits_E \abs{h} \dd{\lambda} < \epsilon$$
	We'll use it to show that $\int_a^x h\dd{\lambda}$ is absolutely continuous. Let $\left\{x_i,y_i\right\}$ set of disjoint intervals such that $\lambda\qty(\bigcup^m (x_i,y_i)) <\delta$ denote $E=\bigcup^m (x_i,y_i)$, thus
	
	$$\int\limits_E \abs{h} \dd{\lambda} < \epsilon$$
	$$\sum^m \abs{f(y_i) - f(x_i)} = \sum^m \abs{\int_{x_i}^{y_i} h \dd{\lambda}} \leq \sum^m \int_{x_i}^{y_i} \abs{h} \dd{\lambda} = \int\limits_E \abs{h} \dd{\lambda} < \epsilon$$
	
	
	$\Leftarrow$:
	
	Suppose $f\in AC[a,b]$ thus $f\in BV[a,b]$ and thus $f'$ exists a.e. and $f' \in \mathcal{L}^1$. Define $g(x) = \int_a^x f' \dd{\lambda}$, by previous direction $g\in AC[a,b]$ and thus $f-g \in AC[a,b]$. We know that $g'$ exists a.e. and $g'=f'$ a.e. Thus $f'-g'=0$ a.e., i.e., $f-g = 0$.
\end{proof}

\begin{coll}
	If $f\in AC[a,b]$ then for all $a\leq x\leq b$ 
	$$\int_a^x f' \dd{\lambda} = f(x)-f(a)$$
\end{coll}
\end{theorem}

\begin{prop}
	$f\in AC[a,b]$ and $E \subseteq [a,b]$ such that $\lambda(E) = 0$ then $f(E)$ measurable and $\lambda(f(E)) = 0$.
\end{prop}
\end{document}
