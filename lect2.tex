\paragraph{Reqirements for measure in $\mathbb{R}$}

\begin{enumerate}
	\item $0\leq \lambda(E) \leq \infty$
	\item $\lambda(\emptyset) = 0$
	\item $\lambda(\qty[a_1,b_1]\cross\qty[a_2,b_2]\cross\dots\cross \qty[a_n,b_n]) = \prod_{i=1}^n (b_i-a_i)$
	\item If $A = \bigcupdot_{k=1}^\infty A_k$, then $\lambda(A) = \sum_{i=1}^\infty \lambda(A_k)$.
	\item If $C$ is acquired from $A$ by rotation or translation $\lambda(C) = \lambda(A)$.
\end{enumerate}
\paragraph{Note} In $\mathbb{R}^3$ it is impossible to define measure that fulfills those requirements eve if we replace sigma-additivity with additivity.
\paragraph{Banach-Tarski paradox} Denote $B$ -- unit ball in $\mathbb{R}^3$. We can write 
$$B = \bigcupdot_{i=1}^5 A_i$$
and find $C_i$ by rotation or translation of $A_i$ such that $\bigcupdot_{i=1}^5 C_i$ is two unit balls.

\begin{center}
	\includesvg{lect2/l2pic1.svg}
\end{center}
\subsection{Construction of $\lambda$}
\begin{definition}[Special boxes]
	Let $E$ box with edges parallel to axes:
	$$E = \qty[a_1,b_1]\cross\qty[a_2,b_2]\cross\dots\cross \qty[a_n,b_n]$$
	
	For $E$ we define 
	$$\lambda(E) = \prod_{i=1}^n (b_i-a_i)$$
\end{definition}
 
 \begin{definition}[Special polygons] is a finite union of special boxes. 
 \end{definition}
 \subparagraph{Note} Each special polygon is a finite union of special boxes with disjoint interior.
 
 Let $P$ is special polygon written as $P = \bigcap_{i=1}^k A_i$ where $A_i$ is special box and their interior is disjoint.
 $$\lambda(P) = \sum_{i=1}^k \lambda(A_i)$$
 \begin{prop}
 	The definition is independent on choice of $A_i$.
 	\begin{proof}
 		Let $P = \bigcap A_i = \bigcap B_i$.
 		
 		If we continue edges of both $A_i$ and $B_i$ we'll get net which divides $P$ into $C_i$ which refines both $A_i$ and $B_i$ and thus
 		$$\lambda(P) = \sum_i \lambda(A_i) = \sum_i \lambda(B_i) = \sum_i \lambda(C_i)$$ 
 	\end{proof}
 \end{prop}

\begin{prop}
	If $P_1$, $P_2$ are special polygons and $P_1\subseteq P_2$ then $\lambda(P_1)\leq \lambda(P_2)$.
	\begin{proof}
		Let $P_2= \bigcap A_i$ and choose the refinement which divides $P_1$. 
	\end{proof}
\end{prop}

\begin{prop}
	If $P_1$, $P_2$ are  special polygons with disjoint interior then
	$$\lambda(P_1\cup P_2) =  \lambda(P_1)+\lambda(P_2)$$
	\begin{proof}
		Find $A_i$ which divides both $P_1$ and $P_2$.
	\end{proof}
\end{prop}

\begin{prop}
	For all $x\in \mathbb{R}^n$ 
	$$\lambda(x+P) = \lambda(P)$$
\end{prop}

	\begin{proof}[Alternative proof]
For special boxes
$$\lambda(E) = \lim_{N\to \infty} \frac{1}{N^n} \abs{E \cap \frac{1}{N}\mathbb{Z}^n}$$

For $n=1$, $I=[a,b] \subseteq \mathbb{R}$. We claim
$$b-a = \lim_{N\to \infty} \frac{1}{N} \abs{E \cap \frac{1}{N}\mathbb{Z}}$$
First of all
$$b-a-1 \leq \abs{[a,b] \cap \mathbb{Z}} \leq b-a+1$$
To find $\abs{[a,b] \cap \frac{1}{2}\mathbb{Z}}$, we can use $\abs{[2a,2b] \cap \mathbb{Z}}$, which means

$$2b-2a-1 \leq \abs{E \cap \frac{1}{2}\mathbb{Z}} \leq 2b-2a+1$$

And for any $N$:
$$Nb-Na-1 \leq \abs{[a,b] \cap \frac{1}{N}\mathbb{Z}} \leq Nb-Na+1$$
$$b-a-\frac{1}{N} \leq \frac{1}{N}\abs{[a,b] \cap \frac{1}{N}\mathbb{Z}} \leq b-a+\frac{1}{N}$$
By sandwich rule, we get the equality.

We can do the same for higher dimension and for open sets, and then we can easily proof the claim.

If $P$ is special polygon and we take $\lim_{N\to \infty} \frac{1}{N^n} \abs{P \cap \frac{1}{N}\mathbb{Z}^n} = \sum \lambda(A_i) $
when $P=\bigcap A_i$

\end{proof}
\paragraph{Open sets}
\begin{definition}
	$G$ is open if $\forall x\in G$ exists ball $B(x,r)$ such that $B\subset G$. Alternatively we can replace ball with special box.
\end{definition}
Thus for any open $G\neq \emptyset$ 
$$G = \bigcup \left\{ P \text{ special polygon}  \right\}$$

And we can define
$$\lambda(G) = \sup \left\{ \lambda(P) | P\subseteq G \right\}$$

\begin{lemma}
	Let $K\subseteq \mathbb{R}^n$ compact set and $\left\{ G_i \right\}_{i\in I}$ open cover ($K\subseteq \bigcup G_i$). Then exists $\epsilon>0$ such that $\forall x\in K$ exists $i\in I$ such that $B(x,\epsilon) \subseteq G_i$.
\end{lemma}
\begin{lemma} \label{lemma_poly_inf}
	For all polygon of dimension $P$ 
	$$\lambda(P) = \inf \left\{ \lambda(G) : P\subset G  \right\}$$
	\begin{proof}
		$$P \subseteq G \Rightarrow \lambda(P) \leq \lambda(G)$$
		Infimum would give
		$$\lambda(P) \leq \inf \left\{ \lambda(G) : P\subset G  \right\}$$
		
		
		Write $P = \bigcup_{k=1}^N I_k$. Then
		$$\lambda(P)  =\sum_{k=1}^N \lambda(I_k)$$
		For $\epsilon$ find $I_k^\epsilon$ such that
		$$\begin{cases}
		\inter I_k^\epsilon \supseteq I_k\\
		\lambda(I_k^\epsilon) \leq \lambda(I_k) + \frac{\epsilon}{N}
		\end{cases}$$
		Denote $G = \bigcup_{k=1}^N \inter (I_k^\epsilon)$, then, from subadditivity
		$$\lambda(G) \leq \sum_{k=1}^N \lambda (\inter I_k^\epsilon) = \sum_{k=1}^N \lambda (I_k^\epsilon) \leq \epsilon + \sum_{k=1}^N \lambda(I_k) $$
		
		In addition,
		$$\inf \lambda(G) \leq \lambda(P) $$
	\end{proof}
\end{lemma}

\begin{prop}
	$$0 \leq \lambda(G) \leq \infty$$
	\begin{proof}
		Obvious
	\end{proof}
\end{prop}
\begin{prop}
	$$\lambda(G) = 0 \iff G =\emptyset$$
\begin{proof}
	If $G$ is not empty, exists $x\in G$ and special box around $x$ such that $P\subseteq G$and thus $\lambda(G) \leq \lambda(P) > 0$
\end{proof}
\end{prop}
\begin{prop}
	$$\lambda(\mathbb{R}^n) = \infty$$
	\begin{proof}
		Any box is subset of $\mathbb{R}^n$ thus $\lambda(\mathbb{R}^n) = \infty$
	\end{proof}
\end{prop}
\begin{prop}
	$$G_1 \subseteq G_2 \Rightarrow \lambda(G_1) \leq \lambda(G_2)$$
	\begin{proof}
		Obvious
	\end{proof}
\end{prop}
\begin{prop}
	$$\lambda\qty(\bigcup_{k=1}^\infty G_k) \leq \sum \lambda(G_k)$$
	\begin{proof}
		Let $P$ special polygon, $P\subseteq \bigcup_{k=1}^\infty G_k$. We'll show that it's possible to write 
		$$P = \bigcup_{j=1}^N I_j$$
		finite union of special boxes with disjoint interior and for each $j$ exists $k$ such that $I_j \subset G_k$. Let $\epsilon$ from lemma for $K=P$. Write $P=\bigcup_{j=1}^N = I_j$ such that diameter of each $I_j<\epsilon$. If $x_j$ is center of $I_j$, then $I_j \subseteq B(x_j,\epsilon) \subseteq G_k$.
		
		If this is possible, for such $P$ denote $$P_k =  \bigcup_{j=1}^\infty I_j | I_j \subset G_k, \forall i<k \quad I_j \not\subset G_i$$
		
		Obviously $\bigcup P_k = P$ and union is finite since for some $m$, for every $k>m$ $P_m=\emptyset$, because there is finite number of $I_j$, and also internals of $P_k$ are disjoint.
		
		Thus $\lambda(P)=\sum \lambda(P_k)\leq\sum \lambda(G_k)$.
		This is right for any $P$, thus
		$$ \lambda\qty(\bigcup (G_k)) = \sup \left\{ \lambda(P) | P\subseteq \bigcup (G_k) \right\} \leq \sum_{k=1}^\infty \lambda(G_k)$$
	\end{proof}
\end{prop}
\begin{prop}
	$$\lambda\qty(\bigcupdot_{k=1}^\infty G_k) = \sum \lambda(G_k)$$
	\begin{proof}
		Since we have sigma-subadditivity, we need only one direction of inequality:
		$$\lambda(G_k) = \sup\left\{ \lambda(P) : P\subseteq G_k \right\}$$
		For any $N$
		$$\sum_{k=1}^N \lambda(G_k) = \sup\left\{ \sum_{k=1}^N \lambda(P_k) : P_k\subseteq G_k \right\} = \sup\left\{ \lambda\qty(\bigcupdot_{k=1}^N P_k) : P_k\subseteq G_k \right\} \leq \lambda\qty(\bigcupdot_{k=1}^N G_k) \leq \lambda\qty(\bigcupdot_{k=1}^\infty G_k) $$
		i.e.,
		$$\sum_{k=1}^\infty \lambda(G_k)  \leq \lambda\qty(\bigcupdot_{k=1}^\infty G_k) $$
	\end{proof}
\end{prop}
\begin{prop}
	$$\lambda(P) = \lambda(\inter P ) =\inf \left\{ \lambda(G) : P\subseteq G\right\}$$
	\begin{proof}
		First, proof that $\lambda(P) = \lambda(\inter P)$. If $I=P$ is non-empty special box $I = [a_1,b_1]\times [a_2,b_2]\times \dots \times [a_n,b_n]$. For any $\epsilon>0$, $I_\epsilon = [a_1+\epsilon,b_1-\epsilon]\times [a_2+\epsilon,b_2-\epsilon]\times \dots \times [a_n+\epsilon,b_n-\epsilon]$. $I_\epsilon \subseteq \inter I$.
		
		That means that $\lambda(I_\epsilon) \leq \lambda(\inter I)$. Obviously, $\lambda(I_\epsilon)\to \lambda(I)$, i.e. $\lambda(I) \leq \lambda(\inter I)$.
		
		Generally, for $P = \bigcup_{k=1}^N I_k$, 
		$$\inter P \geq \bigcup_{k=1}^N \inter I_k$$
		thus
		$$\lambda(\inter P) \geq \lambda \qty(\bigcup_{k=1}^N \inter I_k) = \sum_{k=1}^N \lambda(\inter I_k) \geq \sum_{k=1}^N \lambda(I_k) = \lambda(P)$$
		For any $P$ 
		$$\lambda(\inter P) \geq \lambda P$$
		
		However
		$$\lambda(\inter P) = \sum \left\{ \lambda(Q) : Q \subseteq \inter P  \right\}$$
		$$Q \subseteq P \Rightarrow \lambda(Q) \leq \lambda(P) \Rightarrow \lambda(\inter P) \leq \lambda(P)$$
		
		Second part is obvious from Lemma \ref{lemma_poly_inf}.
	\end{proof}
\end{prop}
\begin{prop}
	$$\lambda(x+G) = \lambda(G)$$
	\begin{proof}
		Obvious since it's right for polygons
	\end{proof}
\end{prop}


